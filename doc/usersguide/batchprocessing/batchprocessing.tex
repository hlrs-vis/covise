
\begin{htmlonly}


\usepackage{html, htmllist}
\usepackage{longtable}

\bodytext{bgcolor="#ffffff" link="#0033cc" vlink="#0033cc"}

%%%==================================================	
%%%==================================================	

% #1  mark defined by \label
% #2  a linktext 
% #3  a html link 
\newcommand{covlink}[3]{\htmladdnormallink{#2}{#3} \latex{(\ref{#1})} }


\newenvironment{covimg}[4]%
{
 \begin{figure}[htp]
  \begin{center}
   \latexonly
      \includegraphics[scale=#4]{#1/pict/#2}
   \endlatexonly  
   \html{\htmladdimg[align="center"]{pict/#2.png}}
   \caption{#3}
  \end{center}
 \end{figure} 
}{} 

\newenvironment{covimg2}[3]%
{ 
 \begin{figure}[htp]
  \begin{center}
     \latexonly
       \includegraphics[scale=#3]{#1/pict/#2}   
     \endlatexonly
     \html{\htmladdimg[align="center"]{pict/#2.png}}
  \end{center}
 \end{figure} 
}{}

\definecolor{output}{rgb}{0.,0.,1.}
\definecolor{depend}{rgb}{1.,0.65,0.}
\definecolor{required}{rgb}{0.58,0.,0.83}
\definecolor{optional}{rgb}{0.,0.39,0.}

\newcommand{\addimage}[1] {\html{\htmladdimg{pict/#1.png}}}

\newcommand{\addpict}[4] {\latexonly
	     \begin{figure}[!htbp]
			  \begin{center}
   	 		  \includegraphics[scale=#1]{#2}
   	 		  \caption{#3}
		 		  \label{#4}
			  \end{center}
	 		\end{figure}
	     \endlatexonly}



\newcommand{\mapeditor}{\textbf{Map Editor}}
\newcommand{\covise}{\textbf{COVISE}}


\end{htmlonly}

%=============================================================
\startdocument
\chapter{Batch Processing in COVISE}
\label{Batch_Processing}
%=============================================================


The {\bf Batch Processing} function has been provided together with COVISE version 5.3.1., and this Appendix
provides a short introduction to (actual a presentation of) this topic.  

\section{What does Batch Processing mean?}

\begin{itemize}
\item {\bf Historically:} A term from the early age of computing: A stack of input-decks (the program) was 
brought to an operator to be processed.
\item {\bf Today:} A set of commands which covers the whole functionality of a software to be processed 
without a Graphical User Interface (GUI)
\end{itemize}

\section{Applications}

{\bf User:}
\begin{itemize}
\item  Creation of individual converters.
\item  Preparation of huge data-sets for interactive analysis (VR) 
or presentations. Typical tasks: Cutting (CropUSG), 
Sampling (Sample), Assembling (BlockCollect)...
\item Specific animations.
\end{itemize}

{\bf Developer:}
\begin{itemize}
\item Testing
\item New applications
\item Rapid prototyping
\end{itemize}

\section{Implementation}

{\bf New user interface:} the command-line \newline \newline

{\bf Language used:} PYTHON \newline \newline

{\bf Advantages of Python:}
\begin{itemize}
\item Open source {\bf (www.python.org)} and widespread in scientific computing
\item Comprehensive syntax
\item Rich on features (object orientation, huge package-library)
\item extensible
\end{itemize}
\clearpage
\section{Mapping of COVISE to PYTHON}

\begin{covimg}{batchprocessing}{mapping}{Mapping of COVISE to PYTHON}{0.7}\end{covimg}
\begin{htmlonly}
Figure 7.1: Mapping of COVISE to PYTHON
\vspace{0.5cm}
\end{htmlonly}


{\bf Rule:} Each module on the map-editor corresponds to an object 
	     in Python; the map itself translates to an object called 
	     \textit{\textbf{net}}

\clearpage

\section{How to begin?}

{\bf Run Python-interface by typing: covise -- script [filename]}

\begin{verbatim}
[gromit|SNAP] ~ > covise -- script

***************************************************************
* COVISE 5.4_a1 starting up, please be patient....            *
* Flexlm license of type STD_UI checked out...                *
*                                                             *
* Starting local request broker...                            *
* Starting user interface....                                 *
*  ******* COVISE PYTHON INTERFACE ********                   *
* ...done initialization                                      *
***************************************************************
  COVISE PYTHON INTERFACE ready

covise>
\end{verbatim}

{\it Note: Using the Python Interpreter provided by your Vendor}
\newline

On some Linux distributions incompatibilities between existing libraries and 
the python interpreter included in covise may occur. In those cases  
using  the python interpreter provided by your linux distributor can resolve 
the problem. Known are incompatibilities at SuSE 9 and Mandrake 9.0 systems. 
\newline

In order to circumvent those problems do:

\begin{itemize}
\item Find out if python is already installed on your system:  
\begin{verbatim} rpm -qa | grep python\end{verbatim}
  should show you that the package is installed. If not, install python 
  according to the rules of your distribution. The packages are currently 
  included in all known linux distributions.
\item Find the path to your python binary by typing `which python` 
\item Set the environment variable COVISE\_LOCAL\_PYTHON to the result of the 
  previous command (/usr/bin/python in many cases).
\end{itemize}

After entering covise --script    you should receive the following output:

\begin{verbatim}
	***************************************************************
	* COVISE 5.3.2 starting up, please be patient....             *
	* Flexlm license of type STD_UI checked out...                *
	*                                                             *
	* Starting local request broker...                            *
	* Starting user interface....                                 *
	* using local python interpreter  /usr/bin/python
	*  ******* COVISE PYTHON INTERFACE ********                   *
	* ...done initialization                                      *
	***************************************************************
		  COVISE PYTHON INTERFACE ready

	covise>
\end{verbatim}

On some systems you might obtain a warning like:\newline

\begin{verbatim}
RuntimeWarning: Python C API version mismatch for module _covise: This Python 
has API version 1012, module _covise has version 1011
\end{verbatim}


In most cases you can ignore this warning but nevertheless it is recommended to 
check the proper functionality by converting an example COVISE map with the 
tool map\_converter:

\begin{verbatim}map\_converter -P -o converted.py $COVISEDIR/net/tutorial/channel.net\end{verbatim} 

and run the resulting python-script  in COVISE-python by typing 
the command
 
\begin{verbatim}covise --script convertedNet.py\end{verbatim}
 
In case problems occur due to 
version incompatibilities of covise and the version of python provided by 
your linux-distributor please contact support@visenso.de.  

\clearpage

\section{Python syntax}

\subsection{Not COVISE related:}

\begin{verbatim}
covise> a=3                        
covise> b=14
\end{verbatim}
	{\it --------- Assignment}
\begin{verbatim}
covise> print a+b		
17
\end{verbatim}
	{\it --------- Output}
\begin{verbatim}
covise> i=0
covise> for i in range(0,3):      
...     print i
...
0
1
2
\end{verbatim}
	{\it --------- Loop}
\begin{verbatim}
covise> s="hello World"		
covise> print s
Hello World
\end{verbatim}
	{\it --------- String}


\vspace{0.5cm}
{\bf Comprehensive tutorial:}\newline
	http://www.python.org/doc/2.2p1/tut/tut.html

\clearpage

\subsection{COVISE-related:}

\begin{verbatim}
covise> myNet=net()
\end{verbatim}
   	{\it --------- Create a net object - implicit after opening the map-editor}
\begin{verbatim}
covise> rin=RWCovise()
\end{verbatim}
	{\it --------- Create a RWCovise object}
\begin{verbatim}
covise> myNet.add( rin )
\end{verbatim}
	{\it --------- Add module to the net - drag RWCovise module to the visual programming area (VPA)}
\begin{verbatim}
covise> rin.showParams()		
  set_grid_path( x )
\end{verbatim}  
  	{\it --------- Utility function}
\begin{verbatim}
covise> rin.showPorts()
mesh_in
mesh
\end{verbatim}
\begin{verbatim}
covise> rin.set_grid_path("share/covise/example-data/COVISE/airbag.covise")
\end{verbatim}	
	{\it --------- Set parameter}
\begin{verbatim}
covise> render=Renderer()
\end{verbatim}		
	{\it --------- Create Renderer module}
\begin{verbatim}
covise> myNet.add( render )
\end{verbatim}		
	{\it --------- Add Renderer to net - drag Renderer to the VPA}
\begin{verbatim}
covise> myNet.connect(rin, "mesh", render, "RenderData")
\end{verbatim}	
	{\it --------- Create connection between modules}
\begin{verbatim}
covise> runMap()
\end{verbatim}			
	{\it --------- Guess what--} (-;
\vspace{0.5cm}
 
\subsection{Mapping rules: COVISE - Python}

\begin{itemize}
\item The Visual Programming Area is represented by the object net()
\item Each COVISE-module is represented by a Python object of 
the same name: Renderer \latexonly{$\rightarrow$}\endlatexonly \begin{htmlonly}->\end{htmlonly}Renderer()
\item Each parameter of a COVISE-module is mapped by a 
member-function with the prefix set\_   
\end{itemize}

{\bf Example:}

\begin{verbatim}
covise> re=ReadEnsightNT()
covise> re.showParams()
  set_case_file( x )
  set_data for sdata1( x )
  set_data for sdata2( x )
  set_data for vdata1( x )
  set_data for vdata2( x )
  set_choose_parts( x )
  set_repair connectivity( x )
\end{verbatim}

{\it Note:}\newline
{\it Due to technical reasons you MUST set a parameter after you have added the module to the
network!}

\subsection{Details of the Python net() object:}

\begin{verbatim}
myNet.add( module )			
\end{verbatim}
	{\it --------- Add module to net}
\begin{verbatim}
[ mynet.remove( module ) ]		
\end{verbatim}
	{\it --------- Remove module from net}
\begin{verbatim}
myNet.save( filename )			
\end{verbatim}
	{\it --------- Save module into a COVISE net file}
\begin{verbatim}
myNet.connect( module1, portName1, module2, prtName2 )	
\end{verbatim}
	{\it --------- Connect two modules}
\begin{verbatim}
myNet.finishedBarrier() 		
\end{verbatim}
	\textit{\textbf{ ---------- Wait until all modules have finished their work !!}}
\vspace{0.5cm}

\subsection{Details of the Python module objects:}

\begin{itemize}
\item Parameter methods
\item Utility methods 
\end{itemize}

\begin{verbatim}
module.showParams()

module.showPorts()
\end{verbatim}

{\it Note:}\newline
{\it Due to technical reasons you MUST set a parameter after you have added the module to the
network!}
\vspace{0.5cm}

\section{One step forward - one step back}

COVISE-net file to Python:

\begin{verbatim}
    map_converter -P -o myFile.py myFile.net
\end{verbatim}
   
Example airbag.net:

\begin{verbatim}
#
#   converted with COVISE (C) map_converter
#   from /home/ralfm_te/covise_snap/net/examples/Airbag.net
#
# create global net
#
theNet = net()
#
# MODULE: RWCovise
#
RWCovise_1 = RWCovise()
theNet.add( RWCovise_1 )
#
# set parameter values
#
RWCovise_1.set_grid_path( "share/covise/example-data/COVISE/airbag.covise" )
#
# MODULE: Renderer
#
Renderer_1 = Renderer()
theNet.add( Renderer_1 )
#
# CONNECTIONS
#
theNet.connect( RWCovise_1, "mesh", Renderer_1, "RenderData" )
#
# uncomment the following line if you want your script to be executed after loading
#
#runMap()
#
# uncomment the following line if you want exit the COVISE-Python interface
#
#sys.exit()
\end{verbatim}
\vspace{0.5cm}
COVISE-net file to Python:
	use the save( fileName ) method of the net() object


