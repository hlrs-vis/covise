
\begin{htmlonly}

\usepackage{html, htmllist}
\usepackage{longtable}

\bodytext{bgcolor="#ffffff" link="#0033cc" vlink="#0033cc"}

%%%==================================================	
%%%==================================================	

% #1  mark defined by \label
% #2  a linktext 
% #3  a html link 
\newcommand{covlink}[3]{\htmladdnormallink{#2}{#3} \latex{(\ref{#1})} }


\newenvironment{covimg}[4]%
{
 \begin{figure}[htp]
  \begin{center}
   \latexonly
      \includegraphics[scale=#4]{#1/pict/#2}
   \endlatexonly  
   \html{\htmladdimg[align="center"]{pict/#2.png}}
   \caption{#3}
  \end{center}
 \end{figure} 
}{} 

\newenvironment{covimg2}[3]%
{ 
 \begin{figure}[htp]
  \begin{center}
     \latexonly
       \includegraphics[scale=#3]{#1/pict/#2}   
     \endlatexonly
     \html{\htmladdimg[align="center"]{pict/#2.png}}
  \end{center}
 \end{figure} 
}{}

\definecolor{output}{rgb}{0.,0.,1.}
\definecolor{depend}{rgb}{1.,0.65,0.}
\definecolor{required}{rgb}{0.58,0.,0.83}
\definecolor{optional}{rgb}{0.,0.39,0.}

\newcommand{\addimage}[1] {\html{\htmladdimg{pict/#1.png}}}

\newcommand{\addpict}[4] {\latexonly
	     \begin{figure}[!htbp]
			  \begin{center}
   	 		  \includegraphics[scale=#1]{#2}
   	 		  \caption{#3}
		 		  \label{#4}
			  \end{center}
	 		\end{figure}
	     \endlatexonly}



\end{htmlonly}



%=============================================================
\startdocument
\chapter{COVISE Data Objects}
\label{CoviseDataObjects}
%=============================================================
\index{Data Objects}

\section{Available Data Object Types}
\latexonly
\index{Data Object Types}
\endlatexonly

Before the access for reading input objects is explained, we give in this 
section a short description of all available data object classes the programmer
can use.

\begin{longtable}{|p{7cm}|p{7cm}|}
\hline
   {\bf Data type} & {\bf Description} \\
\hline\hline
	coDistributedObject \index{Distributed Object} 
	                          & Base class for all data object types \\
\hline
	coDoAbstractStructuredGrid & Abstract base class for structured grids \\
\hline
	coDoUniformGrid \newline 
	coDoRectilinearGrid \newline                        
	coDoStructuredGrid      & Structured Grid Types \\
\hline
	coDoUnstructuredGrid     & Unstructured Grid Types \\
\hline
	coDoCoordinates & Abstract base class for geometry data \\
\hline
	coDoPoints \newline
	coDoLines\newline
	coDoTriangleStrips\newline
	coDoPolygons\newline
        coDoSpheres &   Geometry data types \endhead
\hline
	coDoFloat & Scalar data \\
\hline
	coDoVec2 \newline
	coDoVec3 & Vector data \\
\hline
	coDoTensor  &   Tensor data \\
\hline
	coDoRGBA &  Packed colour values \\
\hline
	coDoSet & Container type for grouping of data objects \\
\hline
   coDoGeometry & Renderer geometry container data type (as created by Collect
module) \\

\hline
\end{longtable}
{\bf        Table 2: COVISE data types}
\vspace{1cm}


The data types have been designed to meet the requirements of common scientific 
visualization problems and are similar to data models of widespread
visualization packages.

 
There are five main groups of data in the COVISE data model: 

\begin{enumerate}
\item Structured grids: They have implicit connectivity information. Addressing 
is  typically done by using 3 indices in x, y and z direction. The most simple 
structured grid is an equidistant orthogonal grid, i.~e.\ a collection of cubes 
or bricks. Also non-equidistant and deformed structured grid types are supported.
\item Unstructured grids: They are constructed of primitive elements by explicit 
definition of the element vertices of each single element.
\item Geometric Objects: They contain geometry element descriptions in various 
formats, which can be displayed by a renderer. They can be combined with color, 
normal and texture information using the Collect module.
\item Data  objects: They contain data elements, which usually reside either on
unstructured or structured grids, but also the other geometric elements are
possible.
\item Set Containers: They allow grouping of multiple objects, typically of the same 
kind. Thus, either multi-part objects or time-series of objects can be described. 
Containers can be used recursively.
\end{enumerate}

Each of these groups can be divided in sub-groups as shown in the table above. 


\section{Two important warnings}

\begin{covimgpath}{CoviseDataObjects}{Warning3}{0.7}\end{covimgpath}
 

\section{The Base Class for Distributed Objects}
\index{Distributed Object|see{coDistributedObject}}
\index{Data Object|see{coDistributedObject}}

This class is the base class of all available data object classes. It offers 
some basic functionality to all object classes:

\begin{longtable}{|p{4cm}|p{10cm}|}
\hline
\multicolumn{2}{|p{13.5cm}|}{\bf const char *getName()} \\
\hline
{Description:}   
              & {get the name of an object} \\
\hline
{Return value:}  & {object name} \endhead
\hline
\end{longtable}
\index{coDistributedObject!getName}

\begin{longtable}{|p{4cm}|p{10cm}|}
\hline
\multicolumn{2}{|p{13.5cm}|}{\bf const char *getType()} \\
\hline
{Description:}   
              & {get the type of an object} \\
\hline
{Return value:}  & {object type} \endhead
\hline
\end{longtable}
\index{coDistributedObject!getType}

\begin{longtable}{|p{4cm}|p{10cm}|}
\hline
\multicolumn{2}{|p{13.5cm}|}{\bf int isType(const char *reqType)} \\
\hline
{Description:}   
            & {check whether a an object has a certain type} \\
\hline
{Return value:}  & {object type} \endhead
\hline
\end{longtable}
\index{coDistributedObject!isType}

\begin{longtable}{|p{4cm}|p{10cm}|}
\hline
\multicolumn{2}{|p{13.5cm}|}{\bf char coDistributedObject::objectOk()} \\
\hline
{Description:}   
          & {Check whether object was created correctly} \\
\hline
{COVISE states:} & {Compute} \\
\hline
{Return value:}  
                    & {NONZERO on success, =0 on error} \endhead
\hline
\end{longtable}
\index{coDistributedObject!objectOk}

The coDistributedObject class is generally only used as the base class for all 
object classes. Direct usage of coDistributedObject class objects is usually
not necessary, except for:

\begin{itemize}
\item receiving data at an input port: A port can deliver different kinds of 
objects, so it returns a pointer to the base class. The user can then request 
the type with getType and then safely up-cast it to the real data type.
\item building/retrieving sets: The data container {\tt coDoSet} can hold 
groups of arbitrary data objects, which are represented by base class pointers. 
\end{itemize}


\section{Object types}

To identify the type of an object, the getType or isType functions can be used. 
This function works with a unique 6-digit string, documented in the
following table

\begin{longtable}{|p{6cm}|p{6cm}|}
\hline
   {\bf Data Object Type} & {\bf Type name} \\
\hline\hline
	coDoUniformGrid & UNIGRD \\
\hline
	coDoRectilinearGrid & RCTGRD \\
\hline
	coDoStructuredGrid & STRGRD \\
\hline
	coDoUnstructuredGrid & UNSGRD \\
\hline
	coDoPoints & POINTS \\		
\hline
	coDoLines &  LINES \\
\hline
	coDoPolygons & POLYGN \\
\hline
	coDoTriangleStrips & TRIANG \\
\hline
	coDoSpheres & SPHERE \\		
\hline
	coDoFloat & USTSDT \\
\hline
	coDoVec3 & USTVDT \\
\hline
	coDoTensor & USTTDT \endhead	
\hline
\end{longtable}
{\bf        Table 2: COVISE Type Identifiers}
\vspace{1cm}

A typical code fragment from the beginning of the {\tt compute()} callback 
function:

\begin{verbatim}
myApplication::compute(const char * /*port*/)     // ... module is executed
{
   // get the grid object from the port and check, 
   // whether it was received at all
   coDistributedObject *gridObj = p_grid->getCurrentObject();
   if (!gridObj)
   {
      sendError("Did not receive object at port %s",
                 p_grid->getName());
      return;
   }
 
   // if this is an unstructured grid...
   if ( coDoUnstructuredGrid *unsGrd = dynamic_cast<coDoUnstructuredGrid *>(gridObj) )
   {
      ... handle unstructured grid
   }
 
   // it is not an unstructured grid, maybe a structured grid...
   else if ( coDoStructuredGrid *strGrd = dynamic_cast<coDoStructuredGrid *>(gridObj) )
   {
      ... handle structured grid
   }
 
   // if it is not a case we can handle, we have to submit an error
   else
   {
      sendError("Illegal object type at port %s : %s",
                              p_grid->getName(), grid->get_type());
      return;
   }
}
\end{verbatim}
 


The code fragment first checks whether an object was created and then forks 
into different branches dependent on the received object type. dynamic\_cast
is used to check if an object is of a certain type and to up-cast the 
object pointer delivered by the port.

\section{Attributes}

Object attributes can be added to all objects derived from the coDistributedObject 
class. Attributes are only meaningful to modules which know about the
existence of a certain attribute. Attributes can be used to pack additional 
information (only character data is possible) into an data object.  Attributes
are created by:


\begin{longtable}{|p{4cm}|p{2.5cm}|p{7cm}|}
\hline
\multicolumn{3}{|p{13.5cm}|}{\bf addAttribute(const char *name, const char *value)} \\
\hline
{Description:}  
           & \multicolumn{2}{p{9.5cm}|}{attach a single attribute to an object} \\
\hline
\multicolumn{1}{|r|}{IN:} & {name} 
                          & {attribute name}\\
\hline
{Return}  & \multicolumn{2}{p{9.5cm}|}{attribute value} \endhead
\hline
\end{longtable}
\index{coDistributedObject!addAttribute}

\begin{longtable}{|p{4cm}|p{2.5cm}|p{7cm}|}
\hline
\multicolumn{3}{|p{13.5cm}|}{\bf addAttributes (int num,\newline
           const char * const* names, const char * const* values)}\\
\hline
{Description:}  
           & \multicolumn{2}{p{9.5cm}|}{attach multiple attributes to an object} \\
\hline
\multicolumn{1}{|r|}{IN:} & {name} 
                          & {attribute name}\\
\hline
\multicolumn{1}{|r|}{IN:} & {values} 
                          & {attribute values}\\
\hline
{Return}  & \multicolumn{2}{p{9.5cm}|}{attribute value} \endhead
\hline
\end{longtable}
\index{coDistributedObject!addAttributes}

The difficult-looking `const' declarations are necessary to allow constant 
fields to be assigned to this argument. Any two-dimensional char array or
list pointer to a list of char pointers can be assigned to this parameter. 

This is how attributes are retrieved from an object:


\begin{longtable}{|p{4cm}|p{2.5cm}|p{7cm}|}
\hline
\multicolumn{3}{|p{13.5cm}|}{\bf const char *getAttribute(const char *name)} \\
\hline
{Description:}  
           & \multicolumn{2}{p{9.5cm}|}{retrieve attribute values} \\
\hline
\multicolumn{1}{|r|}{IN:} & {name} 
                          & {attribute name}\\
\hline
{Return}  
  & \multicolumn{2}{p{9.5cm}|}{attribute value, NULL if attribute does not exist} \endhead
\hline
\end{longtable}
\index{coDistributedObject!getAttribute}

\begin{longtable}{|p{4cm}|p{2.5cm}|p{7cm}|}
\hline
\multicolumn{3}{|p{13.5cm}|}{\bf int getAllAttributes(const char ***names, 
const char ***content)}\\
\hline
{Description:}  
           & \multicolumn{2}{p{9.5cm}|}{retrieve all attribute values} \\
\hline
\multicolumn{1}{|r|}{OUT:} & {names} 
                          & {attribute name}\\
\hline
\multicolumn{1}{|r|}{OUT:} & {contents} 
                          & {attribute contents}\endhead
\hline
\end{longtable}
\index{coDistributedObject!getAllAttributes}

\begin{longtable}{|p{4cm}|p{2.5cm}|p{7cm}|}
\hline
\multicolumn{3}{|p{13.5cm}|}{\bf void copyAllAttributes(coDistributedObject  *src)} \\
\hline
{Description:}  
      & \multicolumn{2}{p{9.5cm}|}{copy all attribute values of object `src' to this object} \\
\hline
\multicolumn{1}{|r|}{IN:} & {src} 
                          & {the source distributed object}\endhead
\hline
\end{longtable}
\index{coDistributedObject!copyAllAttributes}

 

\section{Structured Grid Types}
\index{Structured Grid}

A structured grid can be either a regular orthogonal grid with constant spacing 
(coDoUniformGrid) or a regular orthogonal grid with variable spacing
(coDoRectilinerGrid) or a curvilinear grid (coDoStructuredGrid). Typical for all 
these grids is, that the grid is constructed by intersecting lines.
Furthermore it is not necessary to have a special connection list of the 
neighbouring points (contrary to unstructured grids) because each point can be
addressed by using the indices of  the computational domain.

\subsection{Abstract Structured Grid Base Class}
\index{Structured Grid!Abstract base class|see{coDoGeneralStructuredGrid}}

For easier handling of all these similiar structured grid types, they are
derived from the common abstract base class coDoGeneralStructuredGrid,
which provides the following methods:

\latexonly
\begin{longtable}{|p{4cm}|p{2.5cm}|p{7cm}|}
\hline
\multicolumn{3}{|p{13.5cm}|}{\bf void coDoGeneralStructuredGrid::getGridSize \newline
           (int *x\_size, int *y\_size, int *z\_size)}\\
\hline
{Description:}  
           & \multicolumn{2}{p{9.5cm}|}{get the grid dimensions} \\
\hline
{COVISE states:}  
           & \multicolumn{2}{p{9.5cm}|}{Compute} \\
\hline
\multicolumn{1}{|r|}{IN:} & \multicolumn{1}{p{3cm}|}{\{x$\mid$y$\mid$z\}\_size} 
                          & \multicolumn{1}{p{5cm}|}{ptr to variables in which 
			  dimensions should be stored}\endhead
\hline
\end{longtable}
\index{coDoGeneralStructuredGrid!getGridSize}
\endlatexonly

\begin{htmlonly}
\begin{longtable}{|p{4cm}|p{2.5cm}|p{7cm}|}
\hline
\multicolumn{3}{|p{13.5cm}|}{\bf void coDoGeneralStructuredGrid::getGridSize \newline
           (int *x\_size, int *y\_size, int *z\_size)}\\
\hline
{Description:}  
           & \multicolumn{2}{p{9.5cm}|}{get the grid dimensions} \\
\hline
{COVISE states:}  
           & \multicolumn{2}{p{9.5cm}|}{Compute} \\
\hline
\multicolumn{1}{|r|}{IN:} & \multicolumn{1}{p{3cm}|}{\{x|y|z\}\_size} 
                          & \multicolumn{1}{p{5cm}|}{ptr to variables in which 
			  dimensions should be stored}\endhead
\hline
\end{longtable}
\end{htmlonly}

\latexonly
\begin{longtable}{|p{4cm}|p{2.5cm}|p{7cm}|}
\hline
\multicolumn{3}{|p{13.5cm}|}{\bf void getPointCoordinates \newline
(int i, float *x\_coord, int j,  float *y\_coord,  int k, float *z\_coord)}\\
\hline
{Description:}  
           & \multicolumn{2}{p{9.5cm}|}{get the coordinates of a single grid point} \\
\hline
{COVISE states:}  
           & \multicolumn{2}{p{9.5cm}|}{Compute} \\
\hline
\multicolumn{1}{|r|}{IN:} & \multicolumn{1}{p{3cm}|}{i,j,k} 
                          & \multicolumn{1}{p{5cm}|}{coordinate index in 
			     \{x$\mid$y$\mid$z\} direction}\\
\hline
\multicolumn{1}{|r|}{OUT:} & \multicolumn{1}{p{3cm}|}{\{x$\mid$y$\mid$z\}\_ccord} 
                          & \multicolumn{1}{p{5cm}|}{ptr to variables in 
			  which cordinates should be stored}\endhead
\hline
\end{longtable}
\index{coDoGeneralStructuredGrid!getPointCoordinates}
\endlatexonly

\begin{htmlonly}
\begin{longtable}{|p{4cm}|p{2.5cm}|p{7cm}|}
\hline
\multicolumn{3}{|p{13.5cm}|}{\bf void getPointCoordinates \newline
(int i, float *x\_coord, int j,  float *y\_coord,  int k, float *z\_coord)}\\
\hline
{Description:}  
           & \multicolumn{2}{p{9.5cm}|}{get the coordinates of a single grid point} \\
\hline
{COVISE states:}  
           & \multicolumn{2}{p{9.5cm}|}{Compute} \\
\hline
\multicolumn{1}{|r|}{IN:} & \multicolumn{1}{p{3cm}|}{i,j,k} 
                          & \multicolumn{1}{p{5cm}|}{coordinate index in 
			     \{x|y|z\} direction}\\
\hline
\multicolumn{1}{|r|}{OUT:} & \multicolumn{1}{p{3cm}|}{\{x|y|z\}\_ccord} 
                          & \multicolumn{1}{p{5cm}|}{ptr to variables in 
			  which cordinates should be stored}\endhead
\hline
\end{longtable}
\end{htmlonly}


\subsection{Uniform Grid}
\index{Uniform Grid|see{coDoUniformGrid}}
%\vspace*{1cm}
%{\Large Uniform Grid} 
%\vspace*{0.5cm}

A uniform grid is described by 3 sets of equidistantly split coordinate ranges. 
The interpretation of a uniform grid may vary with the used coordinate
system. COVISE currently only supports Cartesian coordinates, so any uniform 
grid handed over to a COVISE module is interpreted as shown in the
following drawing: 
\latexonly

\endlatexonly

\begin{covimg}{CoviseDataObjects}{Uniform_Grid}{Uniform Grid}{0.7}\end{covimg}
\begin{htmlonly}
Figure: Uniform Grid
\end{htmlonly}
\vspace*{1cm}

Uniform grids are seldom, since the spatial resolution is constant, while most 
real-world applications require varying spatial resolutions. 

To create a new uniform grid object in shared memory, the 
constructor \index{coDoUniformGrid!constructor}is used:

\latexonly
\begin{longtable}{|p{4cm}|p{2.5cm}|p{7cm}|}
\hline
\multicolumn{3}{|p{13.5cm}|}{\bf coDoUniformGrid(const char *name,\newline
           int     x\_size,  int     y\_size,  int     z\_size,\newline
	   float  x\_min,    float   x\_max,   float  y\_min,\newline
	   float  y\_max,    float   z\_min,   float  z\_max }\\
\hline
{Description:}  
       & \multicolumn{2}{p{9.5cm}|}{create a new uniform grid by number of steps} \\
\hline
{COVISE states:}  
           & \multicolumn{2}{p{9.5cm}|}{Compute} \\
\hline
\multicolumn{1}{|r|}{IN:} & \multicolumn{1}{p{3cm}|}{name} 
                          & \multicolumn{1}{p{5cm}|}{new object name}\\
\hline
\multicolumn{1}{|r|}{IN:} & \multicolumn{1}{p{3cm}|}{\{x$\mid$y$\mid$z\}\_size} 
                          & \multicolumn{1}{p{5cm}|}{grid dimensions\newline 
			  in \{x$\mid$y$\mid$z\} direction}\\
\hline
\multicolumn{1}{|r|}{IN:} & \multicolumn{1}{p{3cm}|}{\{x$\mid$y$\mid$z\}\_\{min/max\}} 
                          & \multicolumn{1}{p{5cm}|}{minimum/maximum coordinate 
			  in \{x$\mid$y$\mid$z\} direction}\endhead
\hline
\end{longtable}
\endlatexonly

\begin{htmlonly}
\begin{longtable}{|p{4cm}|p{2.5cm}|p{7cm}|}
\hline
\multicolumn{3}{|p{13.5cm}|}{\bf coDoUniformGrid(const char *name,\newline
           int     x\_size,  int     y\_size,  int     z\_size,\newline
	   float  x\_min,    float   x\_max,   float  y\_min,\newline
	   float  y\_max,    float   z\_min,   float  z\_max }\\
\hline
{Description:}  
       & \multicolumn{2}{p{9.5cm}|}{create a new uniform grid by number of steps} \\
\hline
{COVISE states:}  
           & \multicolumn{2}{p{9.5cm}|}{Compute} \\
\hline
\multicolumn{1}{|r|}{IN:} & \multicolumn{1}{p{3cm}|}{name} 
                          & \multicolumn{1}{p{5cm}|}{new object name}\\
\hline
\multicolumn{1}{|r|}{IN:} & \multicolumn{1}{p{3cm}|}{\{x|y|z\}\_size} 
                          & \multicolumn{1}{p{5cm}|}{grid dimensions\newline 
			  in \{x|y|z\} direction}\\
\hline
\multicolumn{1}{|r|}{IN:} & \multicolumn{1}{p{3cm}|}{\{x|y|z\}\_\{min/max\}} 
                          & \multicolumn{1}{p{5cm}|}{minimum/maximum coordinate 
			  in \{x|y|z\} direction}\endhead
\hline
\end{longtable}
\end{htmlonly}

The new object name is usually taken from the output port's 
{\tt getObjName()} function, except for building part objects in a {\tt coDoSet}
container, as explained later.

Earlier API versions also contained a constructor, which retrieved the object 
from shared memory. This constructor is not necessary any more, since
the {\tt getCurrentObject()} method of the input port automatically retrieves all objects 
and hands over an object pointer.

The following functions extract the information of a received uniform grid:

\latexonly
\begin{longtable}{|p{4cm}|p{2.5cm}|p{7cm}|}
\hline
\multicolumn{3}{|p{13.5cm}|}{\bf void coDoUniformGrid::getGridSize \newline
           (int *x\_size, int *y\_size, int *z\_size)}\\
\hline
{Description:}  
           & \multicolumn{2}{p{9.5cm}|}{get the grid dimensions} \\
\hline
{COVISE states:}  
           & \multicolumn{2}{p{9.5cm}|}{Compute} \\
\hline
\multicolumn{1}{|r|}{IN:} & \multicolumn{1}{p{3cm}|}{\{x$\mid$y$\mid$z\}\_size} 
                          & \multicolumn{1}{p{5cm}|}{ptr to variables in which 
			  dimensions should be stored}\endhead
\hline
\end{longtable}
\index{coDoUniformGrid!getGridSize}
\endlatexonly

\begin{htmlonly}
\begin{longtable}{|p{4cm}|p{2.5cm}|p{7cm}|}
\hline
\multicolumn{3}{|p{13.5cm}|}{\bf void coDoUniformGrid::getGridSize \newline
           (int *x\_size, int *y\_size, int *z\_size)}\\
\hline
{Description:}  
           & \multicolumn{2}{p{9.5cm}|}{get the grid dimensions} \\
\hline
{COVISE states:}  
           & \multicolumn{2}{p{9.5cm}|}{Compute} \\
\hline
\multicolumn{1}{|r|}{IN:} & \multicolumn{1}{p{3cm}|}{\{x|y|z\}\_size} 
                          & \multicolumn{1}{p{5cm}|}{ptr to variables in which 
			  dimensions should be stored}\endhead
\hline
\end{longtable}
\end{htmlonly}

\latexonly
\begin{longtable}{|p{4cm}|p{2.5cm}|p{7cm}|}
\hline
\multicolumn{3}{|p{13.5cm}|}{\bf void getPointCoordinates \newline
(int i, float *x\_coord, int j,  float *y\_coord,  int k, float *z\_coord)}\\
\hline
{Description:}  
           & \multicolumn{2}{p{9.5cm}|}{get the coordinates of a single grid point} \\
\hline
{COVISE states:}  
           & \multicolumn{2}{p{9.5cm}|}{Compute} \\
\hline
\multicolumn{1}{|r|}{IN:} & \multicolumn{1}{p{3cm}|}{i,j,k} 
                          & \multicolumn{1}{p{5cm}|}{coordinate index in 
			     \{x$\mid$y$\mid$z\} direction}\\
\hline
\multicolumn{1}{|r|}{OUT:} & \multicolumn{1}{p{3cm}|}{\{x$\mid$y$\mid$z\}\_ccord} 
                          & \multicolumn{1}{p{5cm}|}{ptr to variables in 
			  which cordinates should be stored}\endhead
\hline
\end{longtable}
\index{coDoUniformGrid!getPointCoordinates}
\endlatexonly

\begin{htmlonly}
\begin{longtable}{|p{4cm}|p{2.5cm}|p{7cm}|}
\hline
\multicolumn{3}{|p{13.5cm}|}{\bf void getPointCoordinates \newline
(int i, float *x\_coord, int j,  float *y\_coord,  int k, float *z\_coord)}\\
\hline
{Description:}  
           & \multicolumn{2}{p{9.5cm}|}{get the coordinates of a single grid point} \\
\hline
{COVISE states:}  
           & \multicolumn{2}{p{9.5cm}|}{Compute} \\
\hline
\multicolumn{1}{|r|}{IN:} & \multicolumn{1}{p{3cm}|}{i,j,k} 
                          & \multicolumn{1}{p{5cm}|}{coordinate index in 
			     \{x|y|z\} direction}\\
\hline
\multicolumn{1}{|r|}{OUT:} & \multicolumn{1}{p{3cm}|}{\{x|y|z\}\_ccord} 
                          & \multicolumn{1}{p{5cm}|}{ptr to variables in 
			  which cordinates should be stored}\endhead
\hline
\end{longtable}
\end{htmlonly}

\latexonly
\begin{longtable}{|p{4cm}|p{2.5cm}|p{7cm}|}
\hline
\multicolumn{3}{|p{13.5cm}|}{\bf void getDelta(float *dx, float *dy, float *dz)}\\
\hline
{Description:}  
           & \multicolumn{2}{p{9.5cm}|}{get the type of an object} \\
\hline
{COVISE states:}  
           & \multicolumn{2}{p{9.5cm}|}{Compute} \\
\hline
\multicolumn{1}{|r|}{OUT:} & \multicolumn{1}{p{3cm}|}{dx,dy,dz} 
                          & \multicolumn{1}{p{5cm}|}{spacing in 
			     \{x$\mid$y$\mid$z\} direction}\endhead
\hline
{Return value}  & \multicolumn{2}{p{9.5cm}|}{none} \\
\hline
\end{longtable}
\index{coDoUniformGrid!getDelta}
\endlatexonly

\begin{htmlonly}
\begin{longtable}{|p{4cm}|p{2.5cm}|p{7cm}|}
\hline
\multicolumn{3}{|p{13.5cm}|}{\bf void getDelta(float *dx, float *dy, float *dz)}\\
\hline
{Description:}  
           & \multicolumn{2}{p{9.5cm}|}{get the type of an object} \\
\hline
{COVISE states:}  
           & \multicolumn{2}{p{9.5cm}|}{Compute} \\
\hline
\multicolumn{1}{|r|}{OUT:} & \multicolumn{1}{p{3cm}|}{dx,dy,dz} 
                          & \multicolumn{1}{p{5cm}|}{spacing in 
			     \{x|y|z\} direction}\endhead
\hline
{Return value}  & \multicolumn{2}{p{9.5cm}|}{none} \\
\hline
\end{longtable}
\end{htmlonly}

\latexonly
\begin{longtable}{|p{4cm}|p{3cm}|p{5cm}|}
\hline
\multicolumn{3}{|p{13.5cm}|}{\bf void getMinMax(  float   *x\_min,  float   *x\_max,\newline
           float   *y\_min,  float   *y\_max,\newline
	   float   *z\_min,  float   *z\_max)}\\
\hline
{Description:}  
           & \multicolumn{2}{p{9.5cm}|}{get the coordinates of a single grid point} \\
\hline
\multicolumn{1}{|r|}{IN:} & \multicolumn{1}{p{3cm}|}{\{x$\mid$y$\mid$z\}\_\{min/max\}} 
                          & \multicolumn{1}{p{5cm}|}{minimum/maximum coordinate 
			  in \{x$\mid$y$\mid$z\} direction}\endhead
\hline
\end{longtable}
\index{coDoUniformGrid!getDelta}
\endlatexonly

\begin{htmlonly}
\begin{longtable}{|p{4cm}|p{3cm}|p{5cm}|}
\hline
\multicolumn{3}{|p{13.5cm}|}{\bf void getMinMax(  float   *x\_min,  float   *x\_max,\newline
           float   *y\_min,  float   *y\_max,\newline
	   float   *z\_min,  float   *z\_max)}\\
\hline
{Description:}  
           & \multicolumn{2}{p{9.5cm}|}{get the coordinates of a single grid point} \\
\hline
\multicolumn{1}{|r|}{IN:} & \multicolumn{1}{p{3cm}|}{\{x|y|z\}\_\{min/max\}} 
                          & \multicolumn{1}{p{5cm}|}{minimum/maximum coordinate 
			  in \{x|y|z\} direction}\endhead
\hline
\end{longtable}
\end{htmlonly}

\subsection{Rectilinear Grid}
\index{Rectilinear Grid|see{coDoRectilinearGrid}}
%\vspace*{1cm}
%{\Large Rectilinear Grid}
%\vspace*{0.5cm}

A rectilinear grid is similar to a uniform grid, except that the spacing 
along each coordinate axis is non-uniform. The coordinates are the coordinates 
on the x, y and z-axis.


\begin{covimg}{CoviseDataObjects}{Rectilinear_Grid}{Rectilinear Grid}{0.7}\end{covimg}
\begin{htmlonly}
Figure: Rectilinear Grid
\end{htmlonly}
\vspace*{1cm}

A new rectilinear grid can be constructed using two different 
constructors \index{coDoRectilinearGrid!constructor}:

\latexonly	     
\begin{longtable}{|p{4cm}|p{2.5cm}|p{7cm}|}
\hline
\multicolumn{3}{|p{13.5cm}|}{\bf  coDoRectilinearGrid(const char *name, int x\_size, int y\_size, int z\_size)}\\
\hline
{Description:}  
           & \multicolumn{2}{p{9.5cm}|}{Create a rectilinear grid} \\
\hline
\multicolumn{1}{|r|}{IN:} & \multicolumn{1}{p{3cm}|}{name} 
                          & \multicolumn{1}{p{5cm}|}{name of rectilinear grid object}\\
\hline
\multicolumn{1}{|r|}{IN:} & \multicolumn{1}{p{3cm}|}{\{x$\mid$y$\mid$z\}\_size} 
                          & \multicolumn{1}{p{5cm}|}{dimension in 
			  \{x$\mid$y$\mid$z\}-direction}\endhead
\hline
\end{longtable}
\endlatexonly

\begin{htmlonly}
\begin{longtable}{|p{4cm}|p{2.5cm}|p{7cm}|}
\hline
\multicolumn{3}{|p{13.5cm}|}{\bf  coDoRectilinearGrid(const char *name, int x\_size, int y\_size, int z\_size)}\\
\hline
{Description:}  
           & \multicolumn{2}{p{9.5cm}|}{Create a rectilinear grid} \\
\hline
\multicolumn{1}{|r|}{IN:} & \multicolumn{1}{p{3cm}|}{name} 
                          & \multicolumn{1}{p{5cm}|}{name of rectilinear grid object}\\
\hline
\multicolumn{1}{|r|}{IN:} & \multicolumn{1}{p{3cm}|}{\{x|y|z\}\_size} 
                          & \multicolumn{1}{p{5cm}|}{dimension in 
			  \{x|y|z\}-direction}\endhead
\hline
\end{longtable}
\end{htmlonly}

This is the usual constructor: it allocates the fields in shared memory but does 
not set any values. The user can then request pointers to the fields in
shared memory and create the coordinate fields directly without need for an 
additional copy in the local memory space.

\latexonly
\begin{longtable}{|p{4cm}|p{2.5cm}|p{7cm}|}
\hline
\multicolumn{3}{|p{13.5cm}|}{\bf  coDoRectilinearGrid(const char *name, int x, int y, int z, 
float *x\_coord, float *y\_coord, float *z\_coord)}\\
\hline
{Description:}  
           & \multicolumn{2}{p{9.5cm}|}{Create a rectilinear grid with given set of coords} \\
\hline
{Parameters} & \multicolumn{1}{p{3cm}|}{name:} 
                          & \multicolumn{1}{p{5cm}|}{name of rectilinear grid object}\\
\hline
\multicolumn{1}{|r|}{IN:} & \multicolumn{1}{p{3cm}|}{\{x$\mid$y$\mid$z\}} 
                          & \multicolumn{1}{p{5cm}|}{size in 
			  \{x$\mid$y$\mid$z\}-direction}\\
\hline
\multicolumn{1}{|r|}{IN:} & \multicolumn{1}{p{3cm}|}{\{x$\mid$y$\mid$z\}\_coord} 
                          & \multicolumn{1}{p{5cm}|}{cordinates in 
			  \{x$\mid$y$\mid$z\}-direction}\endhead
\hline
\end{longtable}
\endlatexonly

\begin{htmlonly}
\begin{longtable}{|p{4cm}|p{2.5cm}|p{7cm}|}
\hline
\multicolumn{3}{|p{13.5cm}|}{\bf  coDoRectilinearGrid(const char *name, int x, int y, int z, 
float *x\_coord, float *y\_coord, float *z\_coord)}\\
\hline
{Description:}  
           & \multicolumn{2}{p{9.5cm}|}{Create a rectilinear grid with given set of coords} \\
\hline
{Parameters} & \multicolumn{1}{p{3cm}|}{name:} 
                          & \multicolumn{1}{p{5cm}|}{name of rectilinear grid object}\\
\hline
\multicolumn{1}{|r|}{IN:} & \multicolumn{1}{p{3cm}|}{\{x|y|z\}} 
                          & \multicolumn{1}{p{5cm}|}{size in 
			  \{x|y|z\}-direction}\\
\hline
\multicolumn{1}{|r|}{IN:} & \multicolumn{1}{p{3cm}|}{\{x|y|z\}\_coord} 
                          & \multicolumn{1}{p{5cm}|}{cordinates in 
			  \{x|y|z\}-direction}\endhead
\hline
\end{longtable}
\end{htmlonly}

If the field of coordinates exists in memory, the object can be created in 
shared memory. This is exactly like creating the object, retrieving the field
pointers and copying the three coordinate fields.

The following routines give information about a rectangular grid object:

\latexonly
\begin{longtable}{|p{4cm}|p{2.5cm}|p{7cm}|}
\hline
\multicolumn{3}{|p{13.5cm}|}{\bf void getGridSize \newline
           (int *x\_dim, int *y\_dim, int *z\_dim)}\\
\hline
{Description:}  
           & \multicolumn{2}{p{9.5cm}|}{Get dimensions of the grid} \\
\hline
\multicolumn{1}{|r|}{OUT:} & \multicolumn{1}{p{3cm}|}{\{x$\mid$y$\mid$z\}\_dim} 
                          & \multicolumn{1}{p{5cm}|}{Pointer to int variables 
			  to return \{x$\mid$y$\mid$z\}-sizes at index i}\endhead
\hline
\end{longtable}
\index{coDoRectilinearGrid!getGridSize}
\endlatexonly

\begin{htmlonly}
\begin{longtable}{|p{4cm}|p{2.5cm}|p{7cm}|}
\hline
\multicolumn{3}{|p{13.5cm}|}{\bf void getGridSize \newline
           (int *x\_dim, int *y\_dim, int *z\_dim)}\\
\hline
{Description:}  
           & \multicolumn{2}{p{9.5cm}|}{Get dimensions of the grid} \\
\hline
\multicolumn{1}{|r|}{OUT:} & \multicolumn{1}{p{3cm}|}{\{x|y|z\}\_dim} 
                          & \multicolumn{1}{p{5cm}|}{Pointer to int variables 
			  to return \{x|y|z\}-sizes at index i}\endhead
\hline
\end{longtable}
\end{htmlonly}

\latexonly
\begin{longtable}{|p{4cm}|p{2.5cm}|p{7cm}|}
\hline
\multicolumn{3}{|p{13.5cm}|}{\bf void getPointCoordinates \newline
(int i, float *x\_coord, int j,  float *y\_coord,  int k, float *z\_coord)}\\
\hline
{Description:}  
           & \multicolumn{2}{p{9.5cm}|}{get the coordinates of a single grid point} \\
\hline
\multicolumn{1}{|r|}{IN:} & \multicolumn{1}{p{3cm}|}{\{i$\mid$j$\mid$k\}} 
                          & \multicolumn{1}{p{5cm}|}{location in 
			     \{x$\mid$y$\mid$z\} direction}\\
\hline
\multicolumn{1}{|r|}{OUT:} & \multicolumn{1}{p{3cm}|}{\{x$\mid$y$\mid$z\}\_coord} 
                          & \multicolumn{1}{p{5cm}|}{Pointer to float variables to
			  return \{x$\mid$y$\mid$z\}-cordinates at 
			  index \{i$\mid$j$\mid$k\}}\endhead
\hline
\end{longtable}
\index{coDoRectilinearGrid!getPointCoordinates}
\endlatexonly

\begin{htmlonly}
\begin{longtable}{|p{4cm}|p{2.5cm}|p{7cm}|}
\hline
\multicolumn{3}{|p{13.5cm}|}{\bf void getPointCoordinates \newline
(int i, float *x\_coord, int j,  float *y\_coord,  int k, float *z\_coord)}\\
\hline
{Description:}  
           & \multicolumn{2}{p{9.5cm}|}{get the coordinates of a single grid point} \\
\hline
\multicolumn{1}{|r|}{IN:} & \multicolumn{1}{p{3cm}|}{\{i|j|k\}} 
                          & \multicolumn{1}{p{5cm}|}{location in 
			     \{x|y|z\} direction}\\
\hline
\multicolumn{1}{|r|}{OUT:} & \multicolumn{1}{p{3cm}|}{\{x|y|z\}\_coord} 
                          & \multicolumn{1}{p{5cm}|}{Pointer to float variables to
			  return \{x|y|z\}-cordinates at 
			  index \{i|j|k\}}\endhead
\hline
\end{longtable}
\end{htmlonly}

The last function gives direct access to the shared memory storage for the 
coordinate arrays. \emph{Warning: coordinate indices are not checked, so writing
beyond the bounds of the allocated array usually crashes the system.}

\latexonly 
\begin{longtable}{|p{4cm}|p{2.5cm}|p{7cm}|}
\hline
\multicolumn{3}{|p{13.5cm}|}{\bf void getAddresses(float **x\_start, float **y\_start, 
                       float **z\_start)}\\
\hline
{Description:}  
           & \multicolumn{2}{p{9.5cm}|}{Get coordinate array pointers} \\
\hline
\multicolumn{1}{|r|}{OUT:} & \multicolumn{1}{p{3cm}|}{\{x$\mid$y$\mid$z\}\_start} 
                          & \multicolumn{1}{p{5cm}|}{Pointer to (float *) variable, 
			  which is set to start of \{x$\mid$y$\mid$z\} coordinate
			  field}\endhead
\hline
\end{longtable} 
\index{coDoRectilinearGrid!getAddresses}
\endlatexonly

\begin{htmlonly}
\begin{longtable}{|p{4cm}|p{2.5cm}|p{7cm}|}
\hline
\multicolumn{3}{|p{13.5cm}|}{\bf void getAddresses(float **x\_start, float **y\_start, 
                       float **z\_start)}\\
\hline
{Description:}  
           & \multicolumn{2}{p{9.5cm}|}{Get coordinate array pointers} \\
\hline
\multicolumn{1}{|r|}{OUT:} & \multicolumn{1}{p{3cm}|}{\{x|y|z\}\_start} 
                          & \multicolumn{1}{p{5cm}|}{Pointer to (float *) variable, 
			  which is set to start of \{x|y|z\} coordinate
			  field}\endhead
\hline
\end{longtable} 
\end{htmlonly}

\subsection{Structured Grid}
\index{Structured Grid|see{coDoStructuredGrid}}
%\vspace*{1cm} 
%{\Large Structured Grid}
%\vspace*{0.5cm}

A structured grid is an arbitrarily deformed hexahedral grid, which still has a 
primitive structure of i x j x k hexahedra. All vertex coordinates are stored
independently, but the connectivity is still implicit.


\begin{covimg}{CoviseDataObjects}
		    {Structured_Grid}{Structured Grid}{0.7}\end{covimg}
\begin{htmlonly}
Figure : Structured Grid
\end{htmlonly} 
\vspace*{1cm}


Like the rectilinear grid, the structured grid can be 
constructed \index{coDoStructuredGrid!constructor} with or 
without pre-setting the coordinate arrays:


\latexonly
\begin{longtable}{|p{4cm}|p{2.5cm}|p{7cm}|}
\hline
\multicolumn{3}{|p{13.5cm}|}{\bf coDoStructuredGrid(const char *name, int x\_nv, int y\_nv, int z\_nv)  }\\
\hline
{Description:}  
           & \multicolumn{2}{p{9.5cm}|}{Create a structured grid} \\
\hline
\multicolumn{1}{|r|}{IN:} & \multicolumn{1}{p{3cm}|}{name} 
                          & \multicolumn{1}{p{5cm}|}{name of structured grid object}\\
\hline
\multicolumn{1}{|r|}{IN:} & \multicolumn{1}{p{3cm}|}{\{x$\mid$y$\mid$z\}\_nv} 
                          & \multicolumn{1}{p{5cm}|}{number of vertices in 
			  \{x$\mid$y$\mid$z\}-direction}\endhead
\hline
\end{longtable}
\endlatexonly

\begin{htmlonly}
\begin{longtable}{|p{4cm}|p{2.5cm}|p{7cm}|}
\hline
\multicolumn{3}{|p{13.5cm}|}{\bf coDoStructuredGrid(const char *name, int x\_nv, int y\_nv, int z\_nv)  }\\
\hline
{Description:}  
           & \multicolumn{2}{p{9.5cm}|}{Create a structured grid} \\
\hline
\multicolumn{1}{|r|}{IN:} & \multicolumn{1}{p{3cm}|}{name} 
                          & \multicolumn{1}{p{5cm}|}{name of structured grid object}\\
\hline
\multicolumn{1}{|r|}{IN:} & \multicolumn{1}{p{3cm}|}{\{x|y|z\}\_nv} 
                          & \multicolumn{1}{p{5cm}|}{number of vertices in 
			  \{x|y|z\}-direction}\endhead
\hline
\end{longtable}
\end{htmlonly}

\latexonly
\begin{longtable}{|p{4cm}|p{2.5cm}|p{7cm}|}
\hline
\multicolumn{3}{|p{13.5cm}|}{\bf coDoStructuredGrid(char *name, int x\_nv, int y\_nv, int z\_nv \newline 
         float *x\_coord, float *y\_coord, float *z\_coord)}\\
\hline
{Description:}  
           & \multicolumn{2}{p{9.5cm}|}{Create a structured grid} \\
\hline
{Parameters} & \multicolumn{1}{p{3cm}|}{name:} 
                          & \multicolumn{1}{p{5cm}|}{name of structured grid object}\\
\hline
\multicolumn{1}{|r|}{IN:} & \multicolumn{1}{p{3cm}|}{\{x$\mid$y$\mid$z\}\_coord} 
                          & \multicolumn{1}{p{5cm}|}{\{x$\mid$y$\mid$z\}-coordinates
			   (Field size: ixjxk)}\\
\hline
\multicolumn{1}{|r|}{IN:} & \multicolumn{1}{p{3cm}|}{\{x$\mid$y$\mid$z\}\_nv} 
                          & \multicolumn{1}{p{5cm}|}{number of vertices in
			  \{x$\mid$y$\mid$z\}-direction}\endhead
\hline
\end{longtable}
\endlatexonly

\begin{htmlonly}
\begin{longtable}{|p{4cm}|p{2.5cm}|p{7cm}|}
\hline
\multicolumn{3}{|p{13.5cm}|}{\bf coDoStructuredGrid(char *name, int x\_nv, int y\_nv, int z\_nv \newline 
         float *x\_coord, float *y\_coord, float *z\_coord)}\\
\hline
{Description:}  
           & \multicolumn{2}{p{9.5cm}|}{Create a structured grid} \\
\hline
{Parameters} & \multicolumn{1}{p{3cm}|}{name:} 
                          & \multicolumn{1}{p{5cm}|}{name of structured grid object}\\
\hline
\multicolumn{1}{|r|}{IN:} & \multicolumn{1}{p{3cm}|}{\{x|y|z\}\_coord} 
                          & \multicolumn{1}{p{5cm}|}{\{x|y|z\}-coordinates
			   (Field size: ixjxk)}\\
\hline
\multicolumn{1}{|r|}{IN:} & \multicolumn{1}{p{3cm}|}{\{x|y|z\}\_nv} 
                          & \multicolumn{1}{p{5cm}|}{number of vertices in
			  \{x|y|z\}-direction}\endhead
\hline
\end{longtable}
\end{htmlonly}

The size and the sequence of the 3 coordinate arrays is arr[x\_nv][y\_nv][z\_nv].

All other functions are equivalent to the methods for rectilinear grids:

\latexonly
\begin{longtable}{|p{4cm}|p{2.5cm}|p{7cm}|}
\hline
\multicolumn{3}{|p{13.5cm}|}{\bf void getPointCoordinates \newline
(int i, float *x\_coord, int j,  float *y\_coord,  int k, float *z\_coord)}\\
\hline
{Description:}  
           & \multicolumn{2}{p{9.5cm}|}{get the coordinates of selected point} \\
\hline
\multicolumn{1}{|r|}{IN:} & \multicolumn{1}{p{3cm}|}{\{i$\mid$j$\mid$k\}} 
                          & \multicolumn{1}{p{5cm}|}{index in 
			     \{x$\mid$y$\mid$z\}-direction}\\
\hline
\multicolumn{1}{|r|}{OUT:} & \multicolumn{1}{p{3cm}|}{\{x$\mid$y$\mid$z\}\_coord} 
                          & \multicolumn{1}{p{5cm}|}{Pointer to variable
			  for requested cordinate}\endhead
\hline
\end{longtable}
\index{coDoStructuredGrid!getPointCoordinates}
\endlatexonly

\begin{htmlonly}
\begin{longtable}{|p{4cm}|p{2.5cm}|p{7cm}|}
\hline
\multicolumn{3}{|p{13.5cm}|}{\bf void getPointCoordinates \newline
(int i, float *x\_coord, int j,  float *y\_coord,  int k, float *z\_coord)}\\
\hline
{Description:}  
           & \multicolumn{2}{p{9.5cm}|}{get the coordinates of selected point} \\
\hline
\multicolumn{1}{|r|}{IN:} & \multicolumn{1}{p{3cm}|}{\{i|j|k\}} 
                          & \multicolumn{1}{p{5cm}|}{index in 
			     \{x|y|z\}-direction}\\
\hline
\multicolumn{1}{|r|}{OUT:} & \multicolumn{1}{p{3cm}|}{\{x|y|z\}\_coord} 
                          & \multicolumn{1}{p{5cm}|}{Pointer to variable
			  for requested cordinate}\endhead
\hline
\end{longtable}
\end{htmlonly}

\latexonly
\begin{longtable}{|p{4cm}|p{2.5cm}|p{7cm}|}
\hline
\multicolumn{3}{|p{13.5cm}|}{\bf void getGridSize(int *x\_dim, int *y\_dim, int *z\_dim)}\\
\hline
{Description:}  
           & \multicolumn{2}{p{9.5cm}|}{get field dimensions} \\
\hline
\multicolumn{1}{|r|}{OUT:} & \multicolumn{1}{p{3cm}|}{\{x$\mid$y$\mid$z\}\_dim} 
                          & \multicolumn{1}{p{5cm}|}{Pointer to variable 
			  for size in \{x$\mid$y$\mid$z\}-direction}\endhead
\hline
\end{longtable}
\index{coDoStructuredGrid!getGridSize}
\endlatexonly

\begin{htmlonly}
\begin{longtable}{|p{4cm}|p{2.5cm}|p{7cm}|}
\hline
\multicolumn{3}{|p{13.5cm}|}{\bf void getGridSize(int *x\_dim, int *y\_dim, int *z\_dim)}\\
\hline
{Description:}  
           & \multicolumn{2}{p{9.5cm}|}{get field dimensions} \\
\hline
\multicolumn{1}{|r|}{OUT:} & \multicolumn{1}{p{3cm}|}{\{x|y|z\}\_dim} 
                          & \multicolumn{1}{p{5cm}|}{Pointer to variable 
			  for size in \{x|y|z\}-direction}\endhead
\hline
\end{longtable}
\end{htmlonly}

\latexonly
\begin{longtable}{|p{4cm}|p{2.5cm}|p{7cm}|}
\hline
\multicolumn{3}{|p{13.5cm}|}{\bf void getAddresses(float **x\_start, float **y\_start, 
                       float **z\_start)}\\
\hline
{Description} & 
\multicolumn{2}{p{9.5cm}|}{get coordinate array pointers} \\
\hline
\multicolumn{1}{|r|}{OUT:} & \multicolumn{1}{p{3cm}|}{\{x$\mid$y$\mid$z\}\_start} 
                          & \multicolumn{1}{p{5cm}|}{Pointer to (float *) var
			  to return starting address of vertex
			  \{x$\mid$y$\mid$z\} coordinate array}\endhead
\hline
\end{longtable} 
\index{coDoStructuredGrid!getAddresses}
\endlatexonly

\begin{htmlonly}
\begin{longtable}{|p{4cm}|p{2.5cm}|p{7cm}|}
\hline
\multicolumn{3}{|p{13.5cm}|}{\bf void getAddresses(float **x\_start, float **y\_start, 
                       float **z\_start)}\\
\hline
{Description} & 
\multicolumn{2}{p{9.5cm}|}{get coordinate array pointers} \\
\hline
\multicolumn{1}{|r|}{OUT:} & \multicolumn{1}{p{3cm}|}{\{x|y|z\}\_start} 
                          & \multicolumn{1}{p{5cm}|}{Pointer to (float *) var
			  to return starting address of vertex
			  \{x|y|z\} coordinate array}\endhead
\hline
\end{longtable} 
\end{htmlonly}

COVISE's structured grid type corresponds to "fields" in AVS which covers 
"uniform", "rectilinear" and "irregular" "AVS-grids". Referring to the IBM
Data Explorer the structured grid type corresponds to "regular", "deformed 
regular", and "irregular" grids.



\section{Unstructured Grid Types}
\index{Unstructured Grid|see{coDoUnstructuredGrid}}

Unstructured grids are grids composed of explicitly described basic elements. 
Such kind of grids is often used in Computational Fluid Dynamics (CFD)
and Structural Analysis with Finite Elements Methods (FEM).

The basic elements for COVISE's unstructured grids are:

\begin{covimg}{CoviseDataObjects}
	   {Usg_Base_Elements}{Unstructured Grid Base Element}{0.7}\end{covimg}
\begin{htmlonly}
Figure: Unstructured Grid Base Elements
\end{htmlonly}
\vspace*{1cm}

All unstructured grids are collections of these basic elements. To describe the grid, multiple lists are maintained:

\begin{itemize}
\item {\bf Type list}: It contains the type of a certain element. This cannot be 
determined implicitly, since tetrahedra and quads have the same number of vertices. 
\item {\bf Element list}: It contains the index into the connectivity list, where 
the description of a certain element begins.
\item {\bf Connectivity list}: It lists the indices of the vertices belonging to 
each element.
\item \latexonly{\bf X$\mid$Y$\mid$Z Coordinate list}\endlatexonly
\begin{htmlonly}{\bf X|Y|Z Coordinate list}\end{htmlonly}: It contains the vertex coordinates.
\end{itemize}

If the type list does not exist, the types are assumed by the relation between the 
sizes of the connectivity and the element list. Not all modules correctly implement 
the non-typed unstructured grids, so using these is strongly discouraged.

The correlation of the different lists is shown in the following drawing:

\begin{covimg}{CoviseDataObjects}
		    {Usg_Format}{Unstructured Grid format}{0.7}\end{covimg}
\begin{htmlonly}
Figure: Unstructured Grid format
\end{htmlonly}

An example for an unstructured Grid with three cells:

\begin{covimg}{CoviseDataObjects}
		    {Usg_Example}{Unstructured Grid example}{0.8}\end{covimg}
\begin{htmlonly}
Figure: Unstructured Grid example
\end{htmlonly}
\vspace*{1cm}


An unstructured grid is usually constructed using the following 
constructor \index{coDoUnstructuredGrid!constructor}:


\begin{longtable}{|p{4cm}|p{2.5cm}|p{7cm}|}
\hline
\multicolumn{3}{|p{13.5cm}|}{\bf coDoUnstructuredGrid\newline
              (const char *name, int nelem, int nconn, int ncoord, int ht)}\\
\hline
{Description:}  
       & \multicolumn{2}{p{9.5cm}|}{Create an unstructured grid} \\
\hline
\multicolumn{1}{|r|}{IN:} & {name} 
                          & {name of USG object}\\
\hline
\multicolumn{1}{|r|}{IN:} & {nelem} 
                          & {number of elements}\\
\hline
\multicolumn{1}{|r|}{IN:} & {nconn} 
                          & {number of connectivities}\\
\hline
\multicolumn{1}{|r|}{IN:} & {ncoord} 
                          & {number of coordinates}\\
\hline
\multicolumn{1}{|r|}{IN:} & {ht} 
                          & {whether the type list exists}\endhead
\hline
\end{longtable}

After constructing the object, pointers to the arrays are retrieved using the 
{\tt getAddresses} method and the lists are then filled directly in shared memory.

The other constructor allows to create the object with already prepared fields. 
This is usually a waste of memory, since a copy has to be held which is not 
necessary if the USG is allocated before.

\latexonly
\begin{longtable}{|p{4cm}|p{2.5cm}|p{7cm}|}
\hline
\multicolumn{3}{|p{13.5cm}|}{\bf coDoUnstructuredGrid(const char *name,\newline
              int nelem, int nconn, int ncoord, int *el, int *cl,\newline
	      float *x\_coord, float *y\_coord, float *z\_coord,\newline
	      int *tl)}\\
\hline
{Description:}  
       & \multicolumn{2}{p{9.5cm}|}{Create an unstructured grid} \\
\hline
\multicolumn{1}{|r|}{IN:} & {name} 
                          & {name of USG object}\\
\hline
\multicolumn{1}{|r|}{IN:} & {nelem} 
                          & {number of elements}\\
\hline
\multicolumn{1}{|r|}{IN:} & {nconn} 
                          & {number of connectivities}\\
\hline
\multicolumn{1}{|r|}{IN:} & {ncoord} 
                          & {number of coordinates}\\
\hline
\multicolumn{1}{|r|}{IN:} & {el} 
                          & {element list}\\
\hline
\multicolumn{1}{|r|}{IN:} & {cl} 
                          & {coordinate list}\\
\hline
\multicolumn{1}{|r|}{IN:} & {tl} 
                          & {type list}\\
\hline
\multicolumn{1}{|r|}{IN:} & {\{x$\mid$y$\mid$z\}\_coord} 
                          & {array of \{x$\mid$y$\mid$z\} 
			  coordinates}\endhead  
\hline
\end{longtable}
\endlatexonly

\begin{htmlonly}
\begin{longtable}{|p{4cm}|p{2.5cm}|p{7cm}|}
\hline
\multicolumn{3}{|p{13.5cm}|}{\bf coDoUnstructuredGrid(const char *name,\newline
              int nelem, int nconn, int ncoord, int *el, int *cl,\newline
	      float *x\_coord, float *y\_coord, float *z\_coord,\newline
	      int *tl)}\\
\hline
{Description:}  
       & \multicolumn{2}{p{9.5cm}|}{Create an unstructured grid} \\
\hline
\multicolumn{1}{|r|}{IN:} & {name} 
                          & {name of USG object}\\
\hline
\multicolumn{1}{|r|}{IN:} & {nelem} 
                          & {number of elements}\\
\hline
\multicolumn{1}{|r|}{IN:} & {nconn} 
                          & {number of connectivities}\\
\hline
\multicolumn{1}{|r|}{IN:} & {ncoord} 
                          & {number of coordinates}\\
\hline
\multicolumn{1}{|r|}{IN:} & {el} 
                          & {element list}\\
\hline
\multicolumn{1}{|r|}{IN:} & {cl} 
                          & {coordinate list}\\
\hline
\multicolumn{1}{|r|}{IN:} & {tl} 
                          & {type list}\\
\hline
\multicolumn{1}{|r|}{IN:} & {\{x|y|z\}\_coord} 
                          & {array of \{x|y|z\} 
			  coordinates}\endhead 
\hline
\end{longtable}
\end{htmlonly}

There are additional constructors for unstructured grids, which can be found in 
the header file, but they will not be supported for future use. 


The following functions retrieve all information from the USG and give access 
to its internal fields. 

 
\begin{longtable}{|p{4cm}|p{2.5cm}|p{7cm}|}
\hline
\multicolumn{3}{|p{13.5cm}|}{\bf void getGridSize(int *numEl, int *numConn, 
   int *numCoord)}\\
\hline
{Description:}  
       & \multicolumn{2}{p{9.5cm}|}{get field dimensions} \\
\hline
\multicolumn{1}{|r|}{OUT:} & {numEl} 
                           & {number of elements}\\
\hline
\multicolumn{1}{|r|}{OUT:} & {numConn} 
                           & {number of connectivity list entries}\\
\hline
\multicolumn{1}{|r|}{OUT:} & {numCoord} 
                           & {number of coordinates}\endhead
\hline
\end{longtable}
\index{coDoUnstructuredGrid!getGridSize}

\latexonly
\begin{longtable}{|p{4cm}|p{2.5cm}|p{7cm}|}
\hline
\multicolumn{3}{|p{13.5cm}|}{\bf void getAddresses(int **elem, int **conn,\newline
              float **x\_coord, float **y\_coord, float **z\_coord)}\\
\hline
{Description:}  
       & \multicolumn{2}{p{9.5cm}|}{Get pointers to internal lists of USG type} \\
\hline
\multicolumn{1}{|r|}{OUT:} & {elem} 
                           & {element list}\\
\hline
\multicolumn{1}{|r|}{OUT:} & {conn} 
                           & {connectivity list}\\
\hline
\multicolumn{1}{|r|}{OUT:}
          & {\{x$\mid$y$\mid$z\}\_list} 
          & {\{x$\mid$y$\mid$z\}-coordinate arrays}\endhead
\hline
\end{longtable}
\index{coDoUnstructuredGrid!getAddresses}
\endlatexonly

\begin{htmlonly}
\begin{longtable}{|p{4cm}|p{2.5cm}|p{7cm}|}
\hline
\multicolumn{3}{|p{13.5cm}|}{\bf void getAddresses(int **elem, int **conn,\newline
              float **x\_coord, float **y\_coord, float **z\_coord)}\\
\hline
{Description:}  
       & \multicolumn{2}{p{9.5cm}|}{Get pointers to internal lists of USG type} \\
\hline
\multicolumn{1}{|r|}{OUT:} & {elem} 
                           & {element list}\\
\hline
\multicolumn{1}{|r|}{OUT:} & {conn} 
                           & {connectivity list}\\
\hline
\multicolumn{1}{|r|}{OUT:}
          & {\{x|y|z\}\_list} 
          & {\{x|y|z\}-coordinate arrays}\endhead
\hline
\end{longtable}
\end{htmlonly}

\begin{longtable}{|p{4cm}|p{2.5cm}|p{7cm}|}
\hline
\multicolumn{3}{|p{13.5cm}|}{\bf void getTypeList(int **tList)}\\
\hline
{Description:}  
       & \multicolumn{2}{p{9.5cm}|}{Get pointers to internal lists od USG type} \\
\hline
\multicolumn{1}{|r|}{OUT:} & {tList} 
                           & {type list}\endhead
\hline
\end{longtable}
\index{coDoUnstructuredGrid!getTypeList}

\begin{longtable}{|p{4cm}|p{10cm}|}
\hline
\multicolumn{2}{|p{13.5cm}|}{\bf int hasTypeList()}\\
\hline
{Description:}  
       & {Check whether a type list exists} \\
\hline
{Return value:}  
       & {=0 : no typelist, else typelist exists} \endhead
\hline
\end{longtable}
\index{coDoUnstructuredGrid!hasTypeList}

The following command creates a coordinate-to-cell mapping. 
\emph{Caution: The fields allocated by this procedure have to be deleted by the user 
with} {\tt delete []} \emph{command.}


\begin{longtable}{|p{4cm}|p{2.5cm}|p{7cm}|}
\hline
\multicolumn{3}{|p{13.5cm}|}{\bf void getNeighborList(int *n, int **elemList, int **vStart)}\\
\hline
{Description:}  
       & \multicolumn{2}{p{9.5cm}|}{get backward mapping} \\
\hline
\multicolumn{1}{|r|}{OUT:}
                & {n} 
                & {number of entries in elemList}\\
\hline
\multicolumn{1}{|r|}{OUT:} 
                & {elemList} 
                & {
                                       [n] : elements containing specific
                                       vertex }\endhead
\hline
\multicolumn{1}{|r|}{OUT:} 
                & {vStart} 
                & {[numvert+1] :\newline 
		                       Entries in elemList 
                                       start at \newline 
				       vStart[vertexNo] and end \newline
                                       before vStart[vertexNo+1]}\\
\hline
\end{longtable}
\index{coDoUnstructuredGrid!getNeighborList}

E.g.the cells containing Point \#7 are {\tt *elemList[vStart[7]...(vStart[8]-1)]}

Example:

\begin{covimg}{CoviseDataObjects} {Elem_Neighbor}{Element Neighbor List}{0.7}\end{covimg}
\begin{htmlonly}
Figure: Element Neighbor List
\end{htmlonly}
\vspace*{1cm}
\clearpage

\section{Data Types}
\index{Data Objects}

The mapping between computational and coordinate space is either direct and 
implicit or explicit. In COVISE explicit data mapping has been chosen.
That means in the three dimensional case, all three coordinates at each grid point 
are stored plus possibly additional scalar values at each of these grid points. 
A vector, e.g. a three dimensional vector, will be treated and stored as three 
scalars at each grid point.

There are several different data types available: scalar data, 2D and 3D vector data,
tensor data, and packed RGBA data. 
All are quite similar, except that the 2D and 3D vector types contain 2 or 3 arrays
respectively instead of 1 array. The splitting into several different fields instead
of 1 large field with x-y-z packed data allows larger fields on systems with restricted
shared memory chunk sizes.

Data contains no information of the underlying grid. Instead, a linear array is saved. 

The user must compare the number of elements in the list, which can be requested 
in all sub-types of unstructured data, and then compare against the number of cells 
and coordinates to decide, whether cell- or point-based data was given.

\subsection{Scalar Data}
\index{Scalar data|see{coDoFloat}}
%\vspace*{1cm}
%{\Large Scalar Data}
%\vspace*{0.5cm}

\index{coDoFloat!constructor}
\begin{longtable}{|p{4cm}|p{2.5cm}|p{7cm}|}
\hline
\multicolumn{3}{|p{13.5cm}|}
{\bf coDoFloat(const char *name, int num\_values)}\\
\hline
\multicolumn{3}{|p{13.5cm}|}
{\bf coDoFloat(const char *name, int num\_values, float *scalar\_data)}\\
\hline
{Description:}  
       & \multicolumn{2}{p{9.5cm}|}{Create an object for scalar 
                                  data without and with setting elements} \\
\hline
\multicolumn{1}{|r|}{IN:} & {name} 
                   & {name of data object}\\
\hline
\multicolumn{1}{|r|}{IN:} & {num\_values} 
      & {length of data array}\\
\hline
\multicolumn{1}{|r|}{IN:} & {scalar\_data} 
      & {data values}\endhead
\hline
\end{longtable}

As for all other objects, it is recommended to create an empty object and 
fill in data to prevent double storage.


\begin{longtable}{|p{4cm}|p{2.5cm}|p{7cm}|}
\hline
\multicolumn{3}{|p{13.5cm}|}{\bf void getAddress(float **start)}\\
\hline
{Description:}  
       & \multicolumn{2}{p{9.5cm}|}{get pointer to object data} \\
\hline
\multicolumn{1}{|r|}{OUT:} & {start} 
                     & {pointer to starting address} \endhead
\hline
\end{longtable}
\index{coDoFloat!getAddress}

\begin{longtable}{|p{4cm}|p{2.5cm}|p{7cm}|}
\hline
\multicolumn{3}{|p{13.5cm}|}
{\bf void getPointValue(int pos, float *value)}\\
\hline
{Description:}  
       & \multicolumn{2}{p{9.5cm}|}{get single data value}\\
\hline
\multicolumn{1}{|r|}{IN:}  & {pos} 
     & {index}\\
\hline
\multicolumn{1}{|r|}{OUT:} & {value} 
                           & {data value}\endhead
\hline
\end{longtable}
\index{coDoFloat!getPointValue}

\begin{longtable}{|p{4cm}|p{10cm}|}
\hline
\multicolumn{2}{|p{13.5cm}|}{\bf int getNumPoints()}\\
\hline
{Description:}  
       & {get number of data values saved}\\
\hline
{Return value:}  
       & {number of elements}\endhead
\hline
\end{longtable}
\index{coDoFloat!getNumPoints}

\subsection{2D Vector Data}
\index{2D Vector data|see{coDoVec2}}
%\vspace*{1cm}
%{\Large 2D Data}
%\vspace*{0.5cm}

\latexonly
\begin{longtable}{|p{4cm}|p{2.5cm}|p{7cm}|}
\hline
\multicolumn{3}{|p{13.5cm}|}
{\bf coDoVec2(const char *name, int num\_values)}\\
\hline
\multicolumn{3}{|p{13.5cm}|}
{\bf coDoVec2(const char *name, int num\_values,\newline
              float *data1, float *data2)}\\
\hline
{Description:}  
       & \multicolumn{2}{p{9.5cm}|}{Create an object for 2D vector data 
                                  without and with setting elements} \\
\hline
\multicolumn{1}{|r|}{IN:} & {name} 
                   & {name of data object}\\
\hline
\multicolumn{1}{|r|}{IN:} & {num\_values} 
      & {length of data array}\\
\hline
\multicolumn{1}{|r|}{IN:} & {data\{1$\mid$2\}} 
      & {data value arrays}\endhead
\hline
\end{longtable}
\index{coDoVec2!constructor}
\endlatexonly

\begin{htmlonly}
\begin{longtable}{|p{4cm}|p{2.5cm}|p{7cm}|}
\hline
\multicolumn{3}{|p{13.5cm}|}
{\bf coDoVec2(const char *name, int num\_values)}\\
\hline
\multicolumn{3}{|p{13.5cm}|}
{\bf coDoVec2(const char *name, int num\_values,\newline
              float *data1, float *data2)}\\
\hline
{Description:}  
       & \multicolumn{2}{p{9.5cm}|}{Create an object for 2D vector data 
                                  without and with setting elements} \\
\hline
\multicolumn{1}{|r|}{IN:} & {name} 
                   & {name of data object}\\
\hline
\multicolumn{1}{|r|}{IN:} & {num\_values} 
      & {length of data array}\\
\hline
\multicolumn{1}{|r|}{IN:} & {data\{1|2\}} 
      & {data value arrays}\endhead
\hline
\end{longtable}
\end{htmlonly}

\latexonly
\begin{longtable}{|p{4cm}|p{2.5cm}|p{7cm}|}
\hline
\multicolumn{3}{|p{13.5cm}|}{\bf void getAddresses(float **data1, float **data2)}\endhead
\hline
{Description:}  
       & \multicolumn{2}{p{9.5cm}|}{get pointers to object data fields} \\
\hline
\multicolumn{1}{|r|}{OUT:} & {data\{1$\mid$2\}} 
      & {pointer to starting address in shared memory} \\
\hline
\end{longtable}
\index{coDoVec2!getAddresses}
\endlatexonly

\begin{htmlonly}
\begin{longtable}{|p{4cm}|p{2.5cm}|p{7cm}|}
\hline
\multicolumn{3}{|p{13.5cm}|}{\bf void getAddresses(float **data1, float **data2)}\endhead
\hline
{Description:}  
       & \multicolumn{2}{p{9.5cm}|}{get pointers to object data fields} \\
\hline
\multicolumn{1}{|r|}{OUT:} & {data\{1|2\}} 
      & {pointer to starting address in shared memory} \\
\hline
\end{longtable}
\index{coDoVec2!getAddresses}
\end{htmlonly}

\latexonly
\begin{longtable}{|p{4cm}|p{2.5cm}|p{7cm}|}
\hline
\multicolumn{3}{|p{13.5cm}|}
{\bf void getPointValue(int pos, float *val1, float *val2)}\\
\hline
{Description:}  
       & \multicolumn{2}{p{9.5cm}|}{get single data value}\\
\hline
\multicolumn{1}{|r|}{IN:}  & {pos} 
     & {index}\\
\hline
\multicolumn{1}{|r|}{OUT:} & {val\{1$\mid$2\}} 
                           & {data values}\endhead
\hline
\end{longtable}
\index{coDoVec2!getPointValue}
\endlatexonly

\begin{htmlonly}
\begin{longtable}{|p{4cm}|p{2.5cm}|p{7cm}|}
\hline
\multicolumn{3}{|p{13.5cm}|}
{\bf void getPointValue(int pos, float *val1, float *val2)}\\
\hline
{Description:}  
       & \multicolumn{2}{p{9.5cm}|}{get single data value}\\
\hline
\multicolumn{1}{|r|}{IN:}  & {pos} 
     & {index}\\
\hline
\multicolumn{1}{|r|}{OUT:} & {val\{1|2\}} 
                           & {data values}\endhead
\hline
\end{longtable}
\end{htmlonly}

\subsection{3D Vector Data}
\index{3D Vector data|see{coDoVec3}}
%\vspace*{1cm}
%{\Large Unstructured Vector Data}
%\vspace*{0.5cm}

\latexonly
\begin{longtable}{|p{4cm}|p{2.5cm}|p{7cm}|}
\hline
\multicolumn{3}{|p{13.5cm}|}
{\bf coDoVec3(const char *name, int num\_values)}\\
\hline
\multicolumn{3}{|p{13.5cm}|}
{\bf coDoVec3(const char *name, int num\_values,\newline
              float *data1, float *data2, float *data2)}\\
\hline
{Description:}  
       & \multicolumn{2}{p{9.5cm}|}{Create an object for 3D vector data 
                                  without and with setting elements} \\
\hline
\multicolumn{1}{|r|}{IN:} & {name} 
                   & {name of data object}\\
\hline
\multicolumn{1}{|r|}{IN:} & {num\_values} 
      & {length of data array}\\
\hline
\multicolumn{1}{|r|}{IN:} & {data\{1$\mid$2$\mid$3\}} 
      & {data value arrays}\endhead
\hline
\end{longtable}
\index{coDoVec3!constructor}
\endlatexonly

\begin{htmlonly}
\begin{longtable}{|p{4cm}|p{2.5cm}|p{7cm}|}
\hline
\multicolumn{3}{|p{13.5cm}|}
{\bf coDoVec3(const char *name, int num\_values}\\
\hline
\multicolumn{3}{|p{13.5cm}|}
{\bf coDoVec3(const char *name, int num\_values,\newline
              float *data1, float *data2, float *data2)}\\
\hline
{Description:}  
       & \multicolumn{2}{p{9.5cm}|}{Create an object for 3D vector data 
                                  without and with setting elements} \\
\hline
\multicolumn{1}{|r|}{IN:} & {name} 
                   & {name of data object}\\
\hline
\multicolumn{1}{|r|}{IN:} & {num\_values} 
      & {length of data array}\\
\hline
\multicolumn{1}{|r|}{IN:} & {data\{1|2|3\}} 
      & {data value arrays}\endhead
\hline
\end{longtable}
\end{htmlonly}

\latexonly
\begin{longtable}{|p{4cm}|p{2.5cm}|p{7cm}|}
\hline
\multicolumn{3}{|p{13.5cm}|}
{\bf void getAddresses(float **data1, float **data2, float **data3)}\\
\hline
{Description:}  
       & \multicolumn{2}{p{9.5cm}|}{get pointers to object data fields} \\
\hline
\multicolumn{1}{|r|}{OUT:} & {data\{1$\mid$2$\mid$3\}} 
                     & {pointer to starting address 
		                                  in shared memory} \endhead
\hline
\end{longtable}
\index{coDoVec3!getAddresses}
\endlatexonly

\begin{htmlonly}
\begin{longtable}{|p{4cm}|p{2.5cm}|p{7cm}|}
\hline
\multicolumn{3}{|p{13.5cm}|}
{\bf void getAddresses(float **data1, float **data2, float **data3)}\\
\hline
{Description:}  
       & \multicolumn{2}{p{9.5cm}|}{get pointers to object data fields} \\
\hline
\multicolumn{1}{|r|}{OUT:} & {data\{1|2|3\}} 
                     & {pointer to starting address 
		                                  in shared memory} \endhead
\hline
\end{longtable}
\end{htmlonly}

\latexonly
\begin{longtable}{|p{4cm}|p{2.5cm}|p{7cm}|}
\hline
\multicolumn{3}{|p{13.5cm}|}
{\bf void getPointValue(int pos, float *val1, float *val2, float *val3)}\\
\hline
{Description:}  
       & \multicolumn{2}{p{9.5cm}|}{get single data value}\\
\hline
\multicolumn{1}{|r|}{IN:}  & {pos} 
     & {index}\\
\hline
\multicolumn{1}{|r|}{OUT:} & {val\{1$\mid$2$\mid$3\}} 
                           & {data values}\endhead
\hline
\end{longtable}
\index{coDoVec3!getPointValue}
\endlatexonly

\begin{htmlonly}
\begin{longtable}{|p{4cm}|p{2.5cm}|p{7cm}|}
\hline
\multicolumn{3}{|p{13.5cm}|}
{\bf void getPointValue(int pos, float *val1, float *val2, float *val3)}\\
\hline
{Description:}  
       & \multicolumn{2}{p{9.5cm}|}{get single data value}\\
\hline
\multicolumn{1}{|r|}{IN:}  & {pos} 
     & {index}\\
\hline
\multicolumn{1}{|r|}{OUT:} & {val\{1|2|3\}} 
                           & {data values}\endhead
\hline
\end{longtable}
\end{htmlonly}

\subsection{Tensor Data}
\index{Tensor data|see{coDoTensor}}
%\vspace*{1cm}
%{\Large Tensor data}
%\vspace*{0.5cm}

\latexonly
\begin{longtable}{|p{4cm}|p{2.5cm}|p{7cm}|}
\hline
\multicolumn{3}{|p{13.5cm}|}
{\bf coDoTensor(const char *name, int num\_values, TensorType ttype)}\\
\hline
\multicolumn{3}{|p{13.5cm}|}
{\bf coDoTensor(const char *name, int num\_values, float *tensor\_data, TensorType ttype)}\\
\hline
{Description:}  
       & \multicolumn{2}{p{9.5cm}|}{Create an object for tensor 
                                  data without and with setting elements} \\
\hline
\multicolumn{1}{|r|}{IN:} & {name} 
                   & {name of data object}\\
\hline
\multicolumn{1}{|r|}{IN:} & {num\_values} 
      & {length of data array}\\
\hline
\multicolumn{1}{|r|}{IN:} & {tensor\_data} 
      & {data values}\\
\hline
\multicolumn{1}{|r|}{IN:} & {ttype} 
      & {kind of tensor}\endhead
\hline
\end{longtable}
\index{coDoTensor!constructor}
\endlatexonly

As for all other objects, it is recommended to create an empty object and 
fill in data to prevent double storage. The kinds of tensors are summarised here:

\begin{longtable}{|p{4cm}|p{2.5cm}|p{7cm}|} 
\hline
{Tensor type} & 
{Space \newline dimension} &
{Sequence of independent components} \\
\hline
\multicolumn{1}{|r|}{S2D:}  & {2}
      & {XX, YY, XY} \\
\hline
\multicolumn{1}{|r|}{F2D:}  & {2}
      & {XX, XY, YX, YY} \\
\hline
\multicolumn{1}{|r|}{S3D:}  & {3}
      & {XX, YY, ZZ, XY, YZ, ZX} \\
\hline
\multicolumn{1}{|r|}{F3D:}  & {3}
      & {XX, XY, XZ, YX, YY, YZ, ZX, ZY, ZZ} \\
\hline
\end{longtable}

The numerical information for this object is kept in a single
array with as many floats as the number of tensors (number of
points in space for which a tensor is described) multiplied
by the number of independent components for the tensor type at issue.

\begin{longtable}{|p{4cm}|p{2.5cm}|p{7cm}|}
\hline
\multicolumn{3}{|p{13.5cm}|}{\bf void getAddress(float **start)}\\
\hline
{Description:}  
       & \multicolumn{2}{p{9.5cm}|}{get pointer to object data} \\
\hline
\multicolumn{1}{|r|}{OUT:} & {start} 
                     & {pointer to starting address} \endhead
\hline
\end{longtable}
\index{coDoTensor!getAddress}

\begin{longtable}{|p{4cm}|p{2.5cm}|p{7cm}|}
\hline
\multicolumn{3}{|p{13.5cm}|}
{\bf void getPointValue(int pos, float *value)}\\
\hline
{Description:}  
       & \multicolumn{2}{p{9.5cm}|}{get tensor components for an index}\\
\hline
\multicolumn{1}{|r|}{IN:}  & {pos} 
     & {index}\\
\hline
\multicolumn{1}{|r|}{OUT:} & {value} 
                           & {array with tensor components}\endhead
\hline
\end{longtable}
\index{coDoTensor!getPointValue}

\begin{longtable}{|p{4cm}|p{10cm}|}
\hline
\multicolumn{2}{|p{13.5cm}|}{\bf int getNumPoints()}\\
\hline
{Description:}  
       & {get number of data values saved}\\
\hline
{Return value:}  
       & {number of tensor elements}\endhead
\hline
\end{longtable}
\index{coDoTensor!getNumPoints}

\begin{longtable}{|p{4cm}|p{10cm}|}
\hline
\multicolumn{2}{|p{13.5cm}|}{\bf coDoTensor::TensorType getTensorType()}\\
\hline
{Description:}  
       & {get tensor type}\\
\hline
{Return value:}  
       & {Either UNKNOWN, S2D, F2D, S3D or F3D}\endhead
\hline
\end{longtable}
\index{coDoTensor!getTensorType}

\subsection{Packed RGBA Data}
\index{Packed RGBA data|see{coDoRGBA}}
\index{RGBA data|see{coDoRGBA}}
%\vspace*{1cm}
%{\Large Packed RGBA Data}
%\vspace*{0.5cm}

Packed RGBA data consists of one 4-byte word per value, containing RGBA color 
and opacity values.

\begin{longtable}{|p{4cm}|p{2.5cm}|p{7cm}|}
\hline
\multicolumn{3}{|p{13.5cm}|}
{\bf coDoRGBA (const char *name, int num\_values)}\\
\hline
\multicolumn{3}{|p{13.5cm}|}
{\bf coDoRGBA (const char *name, int num\_values, int *packedColors)}\\
\hline
{Description:}  
       & \multicolumn{2}{p{9.5cm}|}{Create an object for packed RGBA values} \\
\hline
\multicolumn{1}{|r|}{IN:} & {name} 
                   & {name of data object}\\
\hline
\multicolumn{1}{|r|}{IN:} & {num\_values} 
      & {length of data array}\\
\hline
\multicolumn{1}{|r|}{IN:} & {packedColors} 
      & {data array}\endhead
\hline
\end{longtable}
\index{coDoRGBA!constructor}

\begin{longtable}{|p{4cm}|p{2.5cm}|p{7cm}|}
\hline
\multicolumn{3}{|p{13.5cm}|}{\bf void getAddress(int **color\_field)}\\
\hline
{Description:}  
       & \multicolumn{2}{p{9.5cm}|}{get pointers to object data fields} \\
\hline
\multicolumn{1}{|r|}{OUT:} & {color\_field} 
                     & {pointer to starting address} \endhead
\hline
\end{longtable}
\index{coDoRGBA!getAddress}


\begin{longtable}{|p{4cm}|p{2.5cm}|p{7cm}|}
\hline
\multicolumn{3}{|p{13.5cm}|}
{\bf void getPointValue(int pos, int *value)}\\
\hline
{Description:}  
       & \multicolumn{2}{p{9.5cm}|}{access single element}\\
\hline
\multicolumn{1}{|r|}{IN:}  & {pos} 
     & {index}\\
\hline
\multicolumn{1}{|r|}{OUT:} & {value} 
                           & {RGBA value casted to int}\endhead
\hline
\end{longtable}
\index{coDoRGBA!getPointValue}

\begin{longtable}{|p{4cm}|p{2.5cm}|p{7cm}|}
\hline
\multicolumn{3}{|p{13.5cm}|}
{\bf int setFloatRGBA(int pos, float r, float g, float b, float alpha)}\\
\hline
\multicolumn{3}{|p{13.5cm}|}
{\bf int setIntRGBA(int pos, int r, int g, int b, int alpha)}\\
\hline
{Description:}  
       & \multicolumn{2}{p{9.5cm}|}{set a single value} \\
\hline
\multicolumn{1}{|r|}{IN:} & {pos} 
                   & {-}\\
\hline
\multicolumn{1}{|r|}{IN:} & {r,g,b} 
      & {RGBA float [0..1] or int [0..255]}\endhead
\hline
\end{longtable}
\index{coDoRGBA!setFloatRGBA}
\index{coDoRGBA!setIntRGBA}

\begin{longtable}{|p{4cm}|p{2.5cm}|p{7cm}|}
\hline
\multicolumn{3}{|p{13.5cm}|}
{\bf int getFloatRGBA(int pos, float *r, float *g, float *b, float *alpha)}\\
\hline
\multicolumn{3}{|p{13.5cm}|}
{\bf int getIntRGBA(int pos, int *r, int *g, int *b, int *alpha)}\\
\hline
{Description:}  
       & \multicolumn{2}{p{9.5cm}|}{get a single value} \\
\hline
\multicolumn{1}{|r|}{IN:} & {pos} 
                   & {-}\\
\hline
\multicolumn{1}{|r|}{OUT:} & {r,g,b} 
      & {RGBA value , either float [0..1] 
                                   or int [0..255]}\\
\hline
{Return value:}  
       & \multicolumn{2}{p{9.5cm}|}{nothing (always 1)} \endhead
\hline       
\end{longtable}
\index{coDoRGBA!getFloatRGBA}
\index{coDoRGBA!getIntRGBA}


\section{Geometry data types}
\index{Geometry data}


Geometry data is the kind of data that can be displayed in Renderers. It describes 
a geometrical object consisting of points and connectivity information of the 
geometrical primitives.

\subsection{Geometry Container Class}
\index{Geometry container|see{coDoGeometry}}
%\vspace*{1cm}
%{\Large Geometry container class}
%\vspace*{0.5cm}

The additional container object type {\tt coDoGeometry} is used to combine colors, 
normals and textures into geometrical data types. Usually, this is done with the
"Collect" module, which automatically combines all its input objects into the 
{\tt coDoGeometry} container. 

Users should normally not create {\tt coDoGeometry} objects but instead create 
the single part objects and then let the "Collect" module combine it. This allows 
to add filters, e.g. cropping or simplification, to be attached after the module.

Each coDoGeometry container must contain a geometrical object, while all other parts 
of the container are optional.


\begin{longtable}{|p{4cm}|p{2.5cm}|p{7cm}|}
\hline
\multicolumn{3}{|p{13.5cm}|}
{\bf coDoGeometry (const char *name, coDistributedObject *geometry\_object)}\\
\hline
{Description:}  
 & \multicolumn{2}{p{9.5cm}|}{Create a container around the given 
                            geometrical object} \\
\hline
\multicolumn{1}{|r|}{IN:} & {name} 
                   & {object name}\\
\hline
\multicolumn{1}{|r|}{IN:} & {geometry\_object} 
      & {geometrical object}\endhead
\hline
\end{longtable}
\index{coDoGeometry!constructor}

This routine attaches the given object \emph(not a copy) to the Container. See under 
`Advanced features' what to do for re-using received data objects.

All other parts of the Container are volatile and can be attached to the container 
after its creation. There are no default objects, so if nothing is attached, a null 
object will be returned.

These methods attach additional objects to the container:

\begin{longtable}{|p{4cm}|p{2.5cm}|p{7cm}|}
\hline
\multicolumn{3}{|p{13.5cm}|}
{\bf void setColors(int attrib, coDistributedObject *colObj}\\
\hline
\multicolumn{3}{|p{13.5cm}|}
{\bf void setNormals(int attrib, coDistributedObject *normObj)}\\
\hline
{Description:}  
       & \multicolumn{2}{p{9.5cm}|}{set a single value} \\
\hline
\multicolumn{1}{|r|}{IN:} & {attrib} 
        & {OVERALL, PER\_FACE or PER\_VERTEX 
	                             how data is to be applied to 
				     geometrical objects}\\
\hline
\multicolumn{1}{|r|}{IN:} & {colObj} 
      & {Colors: Either a RGAB object containing
                                   packed colors or a structured or 
				   vector data field containing 
				   RGB valued [0..1]}\\
\hline
\multicolumn{1}{|r|}{IN:} & {normObj} 
      & {Normals: Either a structured or 
                                   vector data field containing 
				   normal vectors}\endhead
\hline
\end{longtable}
\index{coDoGeometry!setColors}
\index{coDoGeometry!setNormals}

\begin{longtable}{|p{4cm}|p{2.5cm}|p{7cm}|}
\hline
\multicolumn{3}{|p{13.5cm}|}
{\bf void coDoGeometry::setTexture(int attrib, coDistributedObject *object)}\\
\hline
{Description:}  
       & \multicolumn{2}{p{9.5cm}|}{attach colors and normals to a 
                                  geometry container} \\
\hline
\multicolumn{1}{|r|}{IN:} & {attrib} 
        & {for future use}\\
\hline
\multicolumn{1}{|r|}{IN:} & {object} 
        & {Texture object containing texture}\endhead
\hline
\end{longtable}
\index{coDoGeometry!setTexture}

The part objects can be retrieved from the container using:
 

\begin{longtable}{|p{4cm}|p{10cm}|}
\hline
\multicolumn{2}{|p{13.5cm}|}
{\bf coDistributedObject *getGeometry()}\\
\hline
\multicolumn{2}{|p{13.5cm}|}
{\bf coDistributedObject *getColors()}\\
\hline
\multicolumn{2}{|p{13.5cm}|}
{\bf coDistributedObject *getNormals()}\\
\hline
\multicolumn{2}{|p{13.5cm}|}
{\bf coDistributedObject *getTexture()}\\
\hline
{Description:}  
       & {get the geometry object from the container} \\
\hline
{Return value:}  
       & {pointer to requested object} \endhead
\hline       
\end{longtable}
\index{coDoGeometry!getGeometry}
\index{coDoGeometry!getColors}
\index{coDoGeometry!getNormals}
\index{coDoGeometry!getTexture}

The attributes of the part objects are retrieved using:


\begin{longtable}{|p{4cm}|p{10cm}|}
\hline
\multicolumn{2}{|p{13.5cm}|}
{\bf int getColorAttributes() int getNormalAttributes() int getTextureAttributes()}\\
\hline
{Description:}  
       & {get object attributes} \\
\hline
{Return value:}  
       & {attribute} \endhead
\hline       
\end{longtable}
\index{coDoGeometry!getColorAttributes}
\index{coDoGeometry!getNormalAttributes}
\index{coDoGeometry!getTextureAttributes}

\subsection{Points}
\index{Point data|see{coDoPoints}}
%\vspace*{1cm}					       
%{\Large Points}
%\vspace*{0.5cm}

A point object is simply a list of triples of data in 3D space. The representation 
in the renderer is a single point. To emphasize points, the "Sphere" module can 
convert points to spheres consisting of multiple triangles together with the 
appropriate normals or to sphere impostors.
					       
\latexonly	       
\begin{longtable}{|p{4cm}|p{2.5cm}|p{7cm}|}
\hline
\multicolumn{3}{|p{13.5cm}|}
{\bf coDoPoints(const char *name, int numPoints)}\\
\hline
\multicolumn{3}{|p{13.5cm}|}
{\bf coDoPoints(const char *name, int numPoints,\newline
  float *x\_coord, float *y\_coord, float *z\_coord)}\\
\hline
{Description:}  
           & \multicolumn{2}{p{9.5cm}|}{create a point object without 
	                              or with initial values} \\
\hline
\multicolumn{1}{|r|}{IN:} & {name} 
                          & {object name}\\
\hline
\multicolumn{1}{|r|}{IN:} & {numPoints} 
                          & {number of points}\\
\hline
\multicolumn{1}{|r|}{IN:} 
           & {\{x$\mid$y$\mid$z\}\_coord} 
           & {point 3D coordinates}\endhead
\hline
\end{longtable}
\index{coDoPoints!constructor}
\endlatexonly

\begin{htmlonly}
\begin{longtable}{|p{4cm}|p{2.5cm}|p{7cm}|}
\hline
\multicolumn{3}{|p{13.5cm}|}
{\bf coDoPoints(const char *name, int numPoints)}\\
\hline
\multicolumn{3}{|p{13.5cm}|}
{\bf coDoPoints(const char *name, int numPoints,\newline
  float *x\_coord, float *y\_coord, float *z\_coord)}\\
\hline
{Description:}  
           & \multicolumn{2}{p{9.5cm}|}{create a point object without 
	                              or with initial values} \\
\hline
\multicolumn{1}{|r|}{IN:} & {name} 
                          & {object name}\\
\hline
\multicolumn{1}{|r|}{IN:} & {numPoints} 
                          & {number of points}\\
\hline
\multicolumn{1}{|r|}{IN:} 
           & {\{x|y|z\}\_coord} 
           & {point 3D coordinates}\endhead
\hline
\end{longtable}
\end{htmlonly}

\latexonly
\begin{longtable}{|p{4cm}|p{2.5cm}|p{7cm}|}
\hline
\multicolumn{3}{|p{13.5cm}|}
{\bf void getAddresses(float **x\_start, float **y\_start, float**z\_start)}\\
\hline
{Description:}  
   & \multicolumn{2}{p{9.5cm}|}{get access to coordinate fields in shared memory} \\
\hline
\multicolumn{1}{|r|}{OUT:} 
           & {\{x$\mid$y$\mid$z\}\_start} 
           & {Start of \{x$\mid$y$\mid$z\} 
	                              cordinates field}\endhead
\hline
\end{longtable}
\index{coDoPoints!getAddresses}
\endlatexonly

\begin{htmlonly}
\begin{longtable}{|p{4cm}|p{2.5cm}|p{7cm}|}
\hline
\multicolumn{3}{|p{13.5cm}|}
{\bf void getAddresses(float **x\_start, float **y\_start, float**z\_start)}\\
\hline
{Description:}  
   & \multicolumn{2}{p{9.5cm}|}{get access to coordinate fields in shared memory} \\
\hline
\multicolumn{1}{|r|}{OUT:} 
           & {\{x|y|z\}\_start} 
           & {Start of \{x|y|z\} 
	                              cordinates field}\endhead
\hline
\end{longtable}
\end{htmlonly}

\begin{longtable}{|p{4cm}|p{10cm}|}
\hline
\multicolumn{2}{|p{13.5cm}|}{\bf int getNumPoints()}\\
\hline
{Description:}  
           & {get number of points in object} \\
\hline
{Return value:}  
           & {number of points} \endhead
\hline
\end{longtable}
\index{coDoPoints!getNumPoints}

\begin{longtable}{|p{4cm}|p{2.5cm}|p{7cm}|}
\hline
\multicolumn{3}{|p{13.5cm}|}{\bf void getPointValue(int pos, float \&value[3])}\\
\hline
{Description:}  
           & \multicolumn{2}{p{9.5cm}|}{get single point} \\
\hline
\multicolumn{1}{|r|}{IN:} & {pos} 
                          & {point index}\\
\hline
\multicolumn{1}{|r|}{OUT:} & {value} 
                          & {point coordinates}\endhead
\hline
\end{longtable}
\index{coDoPoints!getPointValue}

Point objects are used e.g. for particle positions.

An attribute "POINTSIZE" can be attached to the object : a value of "0" results in 
the default pointsize in the renderer, while higher values than "1" can result in 
larger display points, if the hardware and the renderer support this. 

\subsection{Lines}
\index{Line data|see{coDoLines}}
%\vspace*{1cm}
%{\Large Lines} 
%\vspace*{0.5cm}

A coDoLines object holds the points and the connectivity information for a set 
of  line segments in 3D space.

The storage format of coDoLines is similar to coDoUnstructuredGrid. The connectivity 
information is stored in two lists: the first list holds the indices of the points 
that belong to each line segment, while the second list holds the index in 
the first list, where each segment starts. 

This could also be done in one list by ending each segment with an index of -1, 
but the amount of storage needed for this would be nearly the same.
Additionally the access to a line segment in the middle of the list will be much 
faster in the two-list-case.


\begin{covimg}{CoviseDataObjects}
	  {Line_Structure}{Lists for defining a line structure}{0.7}\end{covimg}
\begin{htmlonly}
Figure: Lists for defining a line structure
\end{htmlonly}
\vspace*{1cm}

To better understand the structure, an example is given, which constructs 
a coDoLines object with 3 lines using 10 points for 14 corners:

\begin{covimg}{CoviseDataObjects}
		    {Line_Example}{Example for Lines}{0.7}\end{covimg}
\begin{htmlonly}
Figure: Example for Lines
\end{htmlonly}
\vspace*{1cm}

The resulting lists would be:

\begin{htmlonly}
\begin{verbatim}
num_lines   = 3         Line list:     [ 0 6 10 ]

num_corners = 13        Corner list:   [ 4 0 1 2 3 6 1 6 5 1 8 9 2 7 ]

num_coord   = 10        x_coord        [ x0, x1, ...x9]   Point x-coordinates
                        y_coord        [ y0, y1, ...y9]   Point y-coordinates
                        x_coord        [ z0, z1, ...z9]   Point z-coordinates
\end{verbatim}
\end{htmlonly}

\latexonly
\begin{longtable}{p{2cm} p{1cm} p{3cm} p{3cm} p{3.5cm}}
\multicolumn{1}{p{2cm}}{num\_lines}  
      &  \multicolumn{1}{p{1cm}}{= 3} 
	             &  \multicolumn{1}{p{3cm}}{Line list:}
	                   &  \multicolumn{2}{p{6.5cm}}{[ 0 6 10 ]} \\
			   \\
\multicolumn{1}{p{2cm}}{num\_corners}  
      &  \multicolumn{1}{p{1cm}}{= 13} 
	     &  \multicolumn{1}{p{3cm}}{Corner list:}
	        & \multicolumn{2}{p{65cm}}{[ 4 0 1 2 3 6 1 6 5 1 8 9 2 7 ]}\\
		            \\
\multicolumn{1}{p{2cm}}{num\_coord}  
      &  \multicolumn{1}{p{1cm}}{= 10} 
	   &  \multicolumn{1}{p{3cm}}{x\_coord\newline 
		                       y\_coord\newline
		                       x\_coord }
	       &  \multicolumn{1}{p{3cm}}{[ x$_0$, x$_1$, ...x$_9$]\newline
			                  [ y$_0$, y$_1$, ...y$_9$]\newline
			                  [ z$_0$, z$_1$, ...z$_9$]}
	        & \multicolumn{1}{p{3.5cm}}{Point x-coordinates\newline
		                            Point y-coordinates\newline 
		                            Point z-coordinates} \endhead
\end{longtable}
\endlatexonly

Notice that a closed line is generated by repeating the first point index at the end.

As for the unstructured grid, it is recommended to construct an empty Lines 
object of the appropriate size and then fill the list directly into shared memory. 
This both avoids double storage and the copying time.

The constructors \index{coDoLines!constructor} for Lines objects are:


\begin{longtable}{|p{4cm}|p{2.5cm}|p{7cm}|}
\hline
\multicolumn{3}{|p{13.5cm}|}
{\bf coDoLines(const char *name, int num\_points, int num\_corners, int num\_lines)}\\
\hline
{Description:}  
           & \multicolumn{2}{p{9.5cm}|}{Construct a Lines object} \\
\hline
\multicolumn{1}{|r|}{IN:} & {name} 
                          & {object name}\\
\hline
\multicolumn{1}{|r|}{IN:} & {num\_points} 
                          & {Number of points used for the lines}\\
\hline
\multicolumn{1}{|r|}{IN:} & {num\_corners} 
                          & {Sum of  corners of all lines, 
			                               including start and end point}\\
\hline
\multicolumn{1}{|r|}{IN:} & {num\_lines} 
                          & {Number of distinct lines}\endhead
\hline
\end{longtable}

\latexonly
\begin{longtable}{|p{4cm}|p{2.5cm}|p{7cm}|}
\hline
\multicolumn{3}{|p{13.5cm}|}
{\bf coDoLines(const char *name,\newline
int num\_points, float *x\_c, float *y\_c, float *z\_c,\newline
int num\_corners, int *corner\_list,\newline
int num\_lines, int *line\_list )}\\
\hline
{Description:}  
           & \multicolumn{2}{p{9.5cm}|}{Construct a Lines object 
	                              and copy data from given fields} \\
\hline
\multicolumn{1}{|r|}{IN:} & {name} 
                          & {object name}\\
\hline
\multicolumn{1}{|r|}{IN:} & {num\_points} 
                          & {Number of points 
			                               used for the lines}\\
\hline
\multicolumn{1}{|r|}{IN:} & {num\_corners} 
                          & {sum of  corners of all 
			                               lines, including 
						       start and end point }\\
\hline
\multicolumn{1}{|r|}{IN:} & {num\_lines} 
                          & {Number of distinct lines}\\
\hline
\multicolumn{1}{|r|}{IN:} 
           & {\{x$\mid$y$\mid$z\}\_c} 
           & {point coordinates}\\
\hline
\multicolumn{1}{|r|}{IN:} & {corner\_list} 
                          & {corner list}\\
\hline
\multicolumn{1}{|r|}{IN:} & {line\_list } 
                          & {line list}\endhead
\hline
\end{longtable}
\endlatexonly

\begin{htmlonly}
\begin{longtable}{|p{4cm}|p{2.5cm}|p{7cm}|}
\hline
\multicolumn{3}{|p{13.5cm}|}
{\bf coDoLines(const char *name,\newline
int num\_points, float *x\_c, float *y\_c, float *z\_c,\newline
int num\_corners, int *corner\_list,\newline
int num\_lines, int *line\_list )}\\
\hline
{Description:}  
           & \multicolumn{2}{p{9.5cm}|}{Construct a Lines object 
	                              and copy data from given fields} \\
\hline
\multicolumn{1}{|r|}{IN:} & {name} 
                          & {object name}\\
\hline
\multicolumn{1}{|r|}{IN:} & {num\_points} 
                          & {Number of points 
			                               used for the lines}\\
\hline
\multicolumn{1}{|r|}{IN:} & {num\_corners} 
                          & {sum of  corners of all 
			                               lines, including 
						       start and end point }\\
\hline
\multicolumn{1}{|r|}{IN:} & {num\_lines} 
                          & {Number of distinct lines}\\
\hline
\multicolumn{1}{|r|}{IN:} 
           & {\{x|y|z\}\_c} 
           & {point coordinates}\\
\hline
\multicolumn{1}{|r|}{IN:} & {corner\_list} 
                          & {corner list}\\
\hline
\multicolumn{1}{|r|}{IN:} & {line\_list } 
                          & {line list}\endhead
\hline
\end{longtable}
\end{htmlonly}


Access to the fields of the object is given by the following call:

\emph{Caution: boundaries are not checked!}

\latexonly
\begin{longtable}{|p{4cm}|p{2.5cm}|p{7cm}|}
\hline
\multicolumn{3}{|p{13.5cm}|}
{\bf void getAddresses(float **x\_start, float **y\_start, float **z\_start,\newline
int **corner\_list, int **line\_list)}\endhead
\hline
{Description:}  
           & \multicolumn{2}{p{9.5cm}|}{access fields in shared memory} \\
\hline
\multicolumn{1}{|r|}{OUT:} 
           & {\{x$\mid$y$\mid$z\}\_start} 
           & {Pointer to coordinate field}\\
\hline
\multicolumn{1}{|r|}{OUT:} & {corner\_list} 
                           & {Pointer to corner list}\\
\hline
\multicolumn{1}{|r|}{OUT:} & {line\_list} 
                           & {Pointer to line list}\\
\hline
\end{longtable}
\index{coDoLines!getAddresses}
\endlatexonly

\begin{htmlonly}
\begin{longtable}{|p{4cm}|p{2.5cm}|p{7cm}|}
\hline
\multicolumn{3}{|p{13.5cm}|}
{\bf void getAddresses(float **x\_start, float **y\_start, float **z\_start,\newline
int **corner\_list, int **line\_list)}\endhead
\hline
{Description:}  
           & \multicolumn{2}{p{9.5cm}|}{access fields in shared memory} \\
\hline
\multicolumn{1}{|r|}{OUT:} 
           & {\{x|y|z\}\_start} 
           & {Pointer to coordinate field}\\
\hline
\multicolumn{1}{|r|}{OUT:} & {corner\_list} 
                           & {Pointer to corner list}\\
\hline
\multicolumn{1}{|r|}{OUT:} & {line\_list} 
                           & {Pointer to line list}\\
\hline
\end{longtable}
\end{htmlonly}

Previous versions of this manual used the term "vertex" either for points within 
unstructured grids or for corners in geometry. To avoid confusion, this was changed. 
Still, the geometrical library calls use the term "vertex" for corners.


\begin{longtable}{|p{4cm}|p{10cm}|}
\hline
\multicolumn{2}{|p{13.5cm}|}{\bf int getNumPoints()}\\
\hline
{Description:}  
  & {get number of points (length of coordinate arrays)} \\
\hline
{Return value:}  
           & {number of points} \endhead
\hline
\end{longtable}
\index{coDoLines!getNumPoints}

\begin{longtable}{|p{4cm}|p{10cm}|}
\hline
\multicolumn{2}{|p{13.5cm}|}{\bf int getNumVertices()}\\
\hline
{Description:}  
     & {get number corners (length of corner list)} \\
\hline
{Return value:}  
           & {number of corners} \endhead
\hline
\end{longtable}
\index{coDoLines!getNumVertices}

\begin{longtable}{|p{4cm}|p{10cm}|}
\hline
\multicolumn{2}{|p{13.5cm}|}{\bf int getNumLines()}\\
\hline
{Description:}  
    & {get number of lines (length of line list)} \\
\hline
{Return value:}  
           & {number of lines} \endhead
\hline
\end{longtable}
\index{coDoLines!getNumLines}

\subsection{Polygons}
\index{Polygon data|see{coDoPolygons}}
%\vspace*{1cm}
%{\Large Polygons}  
%\vspace*{0.5cm}

Polygonal objects are similar to lines, except that every set of corners which 
used to define a line now defines a closed polygon. Other than lines,polygons are 
closed automatically, so the last point is automatically connected with the 
first one.

%\begin{covimg}{CoviseDataObjects}
%		    {Polygon_Example}{Polygon Example}{0.8}\end{covimg}

\begin{covimg}{CoviseDataObjects}
		    {PolygonExample}{Polygon Example}{0.6}\end{covimg}		    
\begin{htmlonly}
Figure: Polygon Example
\end{htmlonly}
\vspace*{1cm}

\begin{longtable}{|p{4cm}|p{2.5cm}|p{7cm}|}
\hline
\multicolumn{3}{|p{13.5cm}|}
{\bf coDoPolygons(const char *name,\newline
       int num\_points, int num\_corners, int num\_polygons)}\\
\hline
{Description:}  
           & \multicolumn{2}{p{9.5cm}|}{Construct a Lines object} \\
\hline
\multicolumn{1}{|r|}{IN:} & {name} 
                          & {object name}\\
\hline
\multicolumn{1}{|r|}{IN:} & {num\_points} 
                          & {Number of points used for 
			                               the lines}\\
\hline
\multicolumn{1}{|r|}{IN:} & {num\_corners} 
                          & {Sum of  corners of all lines, 
			                               including start and end point}\\
\hline
\multicolumn{1}{|r|}{IN:} & {num\_polygons} 
                          & {Number of distinct 
			                               polygons}\endhead
\hline
\end{longtable}
\index{coDoPolygons!}

\latexonly
\begin{longtable}{|p{4cm}|p{2.5cm}|p{7cm}|}
\hline
\multicolumn{3}{|p{13.5cm}|}
{\bf void getAddresses(float **x\_start, float **y\_start, float **z\_start,\newline
int **corner\_list, int **polygon\_list)}\\
\hline
{Description:}  
           & \multicolumn{2}{p{9.5cm}|}{access fields in shared memory} \\
\hline
\multicolumn{1}{|r|}{OUT:} 
           & {\{x$\mid$y$\mid$z\}\_start} 
           & {Pointer to coordinate field}\\
\hline
\multicolumn{1}{|r|}{OUT:} & {corner\_list} 
                           & {Pointer to corner list}\\
\hline
\multicolumn{1}{|r|}{OUT:} & {polygon\_list} 
                           & {Pointer to polygon list}\endhead
\hline
\end{longtable}
\index{coDoPolygons!getAddresses}
\endlatexonly

\begin{htmlonly}
\begin{longtable}{|p{4cm}|p{2.5cm}|p{7cm}|}
\hline
\multicolumn{3}{|p{13.5cm}|}
{\bf void getAddresses(float **x\_start, float **y\_start, float **z\_start,\newline
int **corner\_list, int **polygon\_list)}\\
\hline
{Description:}  
           & \multicolumn{2}{p{9.5cm}|}{access fields in shared memory} \\
\hline
\multicolumn{1}{|r|}{OUT:} 
           & {\{x|y|z\}\_start} 
           & {Pointer to coordinate field}\\
\hline
\multicolumn{1}{|r|}{OUT:} & {corner\_list} 
                           & {Pointer to corner list}\\
\hline
\multicolumn{1}{|r|}{OUT:} & {polygon\_list} 
                           & {Pointer to polygon list}\endhead
\hline
\end{longtable}
\end{htmlonly}

As described in previous cases, the recommended way to construct an object 
is to initialize it empty, get the field pointers and then fill it. Nevertheless,
a copying constructor is available.


\latexonly
\begin{longtable}{|p{4cm}|p{2.5cm}|p{7cm}|}
\hline
\multicolumn{3}{|p{13.5cm}|}
{\bf coDoPolygons(const char *name, int num\_points,\newline
float *x\_coord, float *y\_coord, float *z\_coord,\newline
int num\_corners, int *corner\_list,\newline
int num\_polygons, int *polygon\_list)}\\
\hline
{Description:}  
           & \multicolumn{2}{p{9.5cm}|}{Construct a coDoPolygons object 
	                              and copy data from given fields} \\
\hline
\multicolumn{1}{|r|}{IN:} & {name} 
                          & {object name}\\
\hline
\multicolumn{1}{|r|}{IN:} & {num\_points} 
                          & {Number of points 
			                               used for the lines}\\
\hline
\multicolumn{1}{|r|}{IN:} & {num\_corners} 
                          & {Sum of corners of all 
			                               lines, including 
						       start and end point}\\
\hline
\multicolumn{1}{|r|}{IN:} & {num\_polygons} 
                          & {Number of distinct 
			  polygons} \\
\hline
\multicolumn{1}{|r|}{IN:} 
           & {\{x$\mid$y$\mid$z\}\_c} 
           & {point coordinates}\\
\hline
\multicolumn{1}{|r|}{IN:} & {corner\_list} 
                          & {corner list}\\
\hline
\multicolumn{1}{|r|}{IN:} & {polygon\_list} 
                          & {polygon list}\endhead
\hline
\end{longtable}
\index{coDoPolygons!constructor}
\endlatexonly

\begin{htmlonly}
\begin{longtable}{|p{4cm}|p{2.5cm}|p{7cm}|}
\hline
\multicolumn{3}{|p{13.5cm}|}
{\bf coDoPolygons(const char *name, int num\_points,\newline
float *x\_coord, float *y\_coord, float *z\_coord,\newline
int num\_corners, int *corner\_list,\newline
int num\_polygons, int *polygon\_list)}\\
\hline
{Description:}  
           & \multicolumn{2}{p{9.5cm}|}{Construct a coDoPolygons object 
	                              and copy data from given fields} \\
\hline
\multicolumn{1}{|r|}{IN:} & {name} 
                          & {object name}\\
\hline
\multicolumn{1}{|r|}{IN:} & {num\_points} 
                          & {Number of points 
			                               used for the lines}\\
\hline
\multicolumn{1}{|r|}{IN:} & {num\_corners} 
                          & {Sum of corners of all 
			                               lines, including 
						       start and end point}\\
\hline
\multicolumn{1}{|r|}{IN:} & {num\_polygons} 
                          & {Number of distinct 
			  polygons} \\
\hline
\multicolumn{1}{|r|}{IN:} 
           & {\{x|y|z\}\_c} 
           & {point coordinates}\\
\hline
\multicolumn{1}{|r|}{IN:} & {corner\_list} 
                          & {corner list}\\
\hline
\multicolumn{1}{|r|}{IN:} & {polygon\_list} 
                          & {polygon list}\endhead
\hline
\end{longtable}
\end{htmlonly}

Previous versions of this manual used the term "vertex" either for points within 
unstructured grids or for corners in geometry. To avoid confusion, this
was changed. Still, the geometrical library calls use the term "vertex" for corners.


\begin{longtable}{|p{4cm}|p{10cm}|}
\hline
\multicolumn{2}{|p{13.5cm}|}{\bf int getNumPoints()}\\
\hline
{Description:}  
  & {get number of points (length of coordinate arrays)} \\
\hline
{Return value:}  
           & {number of points} \endhead
\hline
\end{longtable}
\index{coDoPolygons!getNumPoints}

\begin{longtable}{|p{4cm}|p{10cm}|}
\hline
\multicolumn{2}{|p{13.5cm}|}{\bf int getNumVertices()}\\
\hline
{Description:}  
     & {get number corners (length of corner list)} \\
\hline
{Return value:}  
           & {number of corners} \endhead
\hline
\end{longtable}
\index{coDoPolygons!getNumVertices}

\begin{longtable}{|p{4cm}|p{10cm}|}
\hline
\multicolumn{2}{|p{13.5cm}|}{\bf int getNumPolygons()}\\
\hline
{Description:}  
    & {get number of polygons (length of polygon list)} \\
\hline
{Return value:}  
           & {number of polygons}\endhead
\hline
\end{longtable}
\index{coDoPolygons!getNumLines}

\begin{longtable}{|p{4cm}|p{3.5cm}|p{6.5cm}|}
\hline
\multicolumn{3}{|p{13.5cm}|}{|p{14cm}|}{\bf get\_neighbor\_list(int *n, int **polygon\_list, int **index\_list)} \\
\hline
{Description:}  
    & \multicolumn{2}{p{10cm}|}{returns list of polygons that have the same point in common} \\
\hline
\multicolumn{1}{|r|}{OUT:} 
           & \multicolumn{1}{p{3.5cm}|}{n} 
           & \multicolumn{1}{p{6.5cm}|}{number of neigbors}\\
\hline
\multicolumn{1}{|r|}{OUT:} 
           & \multicolumn{1}{p{3.5cm}|}{polygon\_list[numconn]} 
           & \multicolumn{1}{p{6.5cm}|}{polygons that have one point in common}\endhead
\hline
\multicolumn{1}{|r|}{OUT:} 
           & \multicolumn{1}{p{3.5cm}|}{index\_list[numccod+1]} 
           & \multicolumn{1}{p{6.5cm}|}{begin of next group of polygons that have one point in common}\\
\hline
\end{longtable}
\index{coDoPolygons!get\_neighbor\_list}

Example:

Polygons of vertex i:
lnl[(lnli[i]...lnli[i+1]-1)]

\begin{covimg}{CoviseDataObjects} {Polygon_Neighbor}{Polygon Neighbor List}{0.7}\end{covimg}
\begin{htmlonly}
Figure: Polygon Neighbor List
\end{htmlonly}
\vspace*{1cm}
\clearpage

\subsection{Triangle Strips}
\index{Triangle strips|see{coDoTriangleStrips}}
%\vspace*{1cm}
%{\Large Triangle Strips}
%\vspace*{0.5cm}

Triangle strips are a special case of polygons and are rendered very efficiently 
on hardware accelerated graphics. Nevertheless, PER\_VERTEX coloring is not available 
and leads to strange colored objects. A triangle strip is stored similar to lines 
and polygons, except that a  line/polygon is now interpreted as a triangle strip.


\begin{covimg}{CoviseDataObjects}
		    {Triangle_Strip}{Single triangle strip}{0.7}\end{covimg}
\begin{htmlonly}
Figure: Single triangle strip
\end{htmlonly}
\vspace*{1cm}

Just as the coDoPolygon object contains multiple polygons, the coDoTriangleStrips object contains multiple 
strips, with a strip list pointing into a corner list which holds the corner values.


\begin{longtable}{|p{4cm}|p{2.5cm}|p{7cm}|}
\hline
\multicolumn{3}{|p{13.5cm}|}
{\bf coDoTriangleStrips(const char *name,\newline
     int num\_points, int num\_corners, int num\_strips)}\\
\hline
{Description:}  
           & \multicolumn{2}{p{9.5cm}|}{Create a Triangle Strip Object} \\
\hline
\multicolumn{1}{|r|}{IN:} & {name} 
                          & {name of triangle strips object}\\
\hline
\multicolumn{1}{|r|}{IN:} & {num\_points} 
                          & {number of coordinates (points)}\\
\hline
\multicolumn{1}{|r|}{IN:} & {num\_corners} 
                          & {length of corner list}\\
\hline
\multicolumn{1}{|r|}{IN:} & {num\_strips} 
                          & {length of strip list}\endhead
\hline
\end{longtable}
\index{coDoTriangleStrips!constructor}


\latexonly
\begin{longtable}{|p{4cm}|p{2.5cm}|p{7cm}|}
\hline
\multicolumn{3}{|p{13.5cm}|}
{\bf coDoTriangleStrips(const char *name,\newline
int num\_points, float *x\_c, float *y\_c, float *z\_c,\newline
int num\_corners, int *corner\_list,\newline
int num\_strips, int *strips\_list )}\\
\hline
{Description:}  
           & \multicolumn{2}{p{9.5cm}|}{Create a Triangle Strip Object} \\
\hline
\multicolumn{1}{|r|}{IN:} & {name, num\_*} 
                          & {as above}\\
\hline
\multicolumn{1}{|r|}{IN:} & {\{x$\mid$y$\mid$z\}\_c} 
                          & {coordinates array}\\
\hline
\multicolumn{1}{|r|}{IN:} & {corner\_list} 
                          & {corner list}\\
\hline
\multicolumn{1}{|r|}{IN:} & {strips\_list } 
                          & {strip list} \\
\hline
\end{longtable}
\endlatexonly

\begin{htmlonly}
\begin{longtable}{|p{4cm}|p{2.5cm}|p{7cm}|}
\hline
\multicolumn{3}{|p{13.5cm}|}
{\bf coDoTriangleStrips(const char *name,\newline
int num\_points, float *x\_c, float *y\_c, float *z\_c,\newline
int num\_corners, int *corner\_list,\newline
int num\_strips, int *strips\_list )}\\
\hline
{Description:}  
           & \multicolumn{2}{p{9.5cm}|}{Create a Triangle Strip Object} \\
\hline
\multicolumn{1}{|r|}{IN:} & {name, num\_*} 
                          & {as above}\\
\hline
\multicolumn{1}{|r|}{IN:} & {\{x|y|z\}\_c} 
                          & {coordinates array}\\
\hline
\multicolumn{1}{|r|}{IN:} & {corner\_list} 
                          & {corner list}\\
\hline
\multicolumn{1}{|r|}{IN:} & {strips\_list } 
                          & {strip list} \\
\hline
\end{longtable}
\end{htmlonly}

The functions {\tt getAddresses()}, {\tt getNumPoints()}, 
{\tt getNumVertices()} and {\tt getNumStrips()} work exactly like 
their coDoPolygon counterparts.

Optimal rendering performance already be reached with about 5 to 6 triangles 
per Strip.

Example:

\begin{covimg}{CoviseDataObjects}{TriangleExample}{Triangle Strip example}{0.6}\end{covimg}
	    
\begin{htmlonly}
Figure: Triangle Strip example
\end{htmlonly}




\section{Pixel Image Objects}
\index{Image data|see {coDoPixelImage}}
\index{Pixel image data|see {coDoPixelImage}}

Pixel images with different formats can be stored in the PixelImage type. 
Warning: the data saved in the pixel buffer of this type is transferred in
binary between machines and not converted.


\begin{longtable}{|p{4cm}|p{2.5cm}|p{7cm}|}
\hline
\multicolumn{3}{|p{13.5cm}|}
{\bf coDoPixelImage(const char *name,\newline
     int width, int height, unsigned form, short psize)}\\
\hline
{Description:}  
           & \multicolumn{2}{p{9.5cm}|}{create a Pixel Image} \\
\hline
\multicolumn{1}{|r|}{IN:} & {width} 
                          & {Image width}\\
\hline
\multicolumn{1}{|r|}{IN:} & {heigth} 
                          & {Image heigth}\\
\hline
\multicolumn{1}{|r|}{IN:} & {form} 
                          & {format ID}\\
\hline
\multicolumn{1}{|r|}{IN:} & {psize} 
                          & {number of bytes per pixel}\endhead
\hline
\end{longtable}
\index{coDoPixelImage!constructor}

\begin{longtable}{|p{4cm}|p{10cm}|}
\hline
\multicolumn{2}{|p{13.5cm}|}
{\bf char* getPixels()}\\
\hline
{Description:}  
           & {get a pointer to the storage area} \\
\hline
{Return value:}  
           & {pointer to buffer area} \endhead
\hline
\end{longtable}
\index{coDoPixelImage!getPixels}

All other parameters can be requested using member functions:

\begin{longtable}{p{5.5cm} p{5.5cm}}
{\tt int getWidth()}     &   	delivers image width  \\

{\tt int getHeight()}    &  	delivers image height \\

{\tt int getPixelsize()} & 	delivers image pixel size \\

{\tt int getormat()}    &  	delivers format ID \endhead
\end{longtable}
\index{coDoPixelImage!getWidth}
\index{coDoPixelImage!getHeight}
\index{coDoPixelImage!getPixelsize}
\index{coDoPixelImage!getFormat}




\subsection{2D Textures}
\index{Texture data|see {coDoTexture}}
\index{2D textures|see {coDoTexture}}
%\vspace*{1cm}
%{\Large Texture}
%\vspace*{0.5cm}

Textures are, if supported in hardware, a fast method to apply complex coloring 
to geometrical objects. 

A Texture object is created from a PixelImage, a list of vertex indices, 
a 2D field of texture coordinates  and some additional information specific to 
texture usage in OpenGL. 


\begin{longtable}{|p{4cm}|p{2.5cm}|p{7cm}|}
\hline
\multicolumn{3}{|p{13.5cm}|}
{\bf coDoTexture(const char *name, coDoPixelImage* image,\newline
     int b, int c,  int l, int nv,int* vi, int nc, float** coords);}\\
\hline
{Description:}  
           & \multicolumn{2}{p{9.5cm}|}{create a texture object} \\
\hline
\multicolumn{1}{|r|}{IN:} & {image} 
                          & {texture buffer}\\
\hline
\multicolumn{1}{|r|}{IN:} & {b} 
                          & {number of border pixels}\\
\hline
\multicolumn{1}{|r|}{IN:} & {c} 
                          & {number of components}\\
\hline
\multicolumn{1}{|r|}{IN:} & {l} 
                          & {level}\\
\hline
\multicolumn{1}{|r|}{IN:} & {nv} 
                          & {number of vertices}\\
\hline
\multicolumn{1}{|r|}{IN:} & {int* vi} 
                          & {vertex-indices}\\
\hline
\multicolumn{1}{|r|}{IN:} & {int nc} 
                          & {number of ccordinates}\\
\hline
\multicolumn{1}{|r|}{IN:} & {coords} 
                          & {x/y coordinates:\newline 
			                        Pointer to float*[2]variable}\endhead
\hline
\end{longtable}
\index{coDoTexture!constructor}

This information can also be retrieved by a module, using the following set of 
member functions:

\begin{longtable}{p{3cm} p{5.5cm}}
{\bf coDoPixelImage*}  &  {\bf getBuffer();}\\
{\bf int }             &  {\bf getBorder();}\\
{\bf int }             &  {\bf getComponents();}\\
{\bf int }             &  {\bf getLevel();}\\
{\bf int }             &  {\bf getWidth();}\\
{\bf int }             &  {\bf getHeight();}\\
{\bf int* }            &  {\bf getVertices();}\\
{\bf int }             &  {\bf getNumCoordinates();}\\
{\bf float**}          &  {\bf getCoordinates();}\\
\end{longtable}
\index{coDoTexture!getBuffer}
\index{coDoTexture!getBorder}
\index{coDoTexture!getComponents}
\index{coDoTexture!getLevel}
\index{coDoTexture!getWidth}
\index{coDoTexture!getHeight}
\index{coDoTexture!getVertices}
\index{coDoTexture!getNumCoordinates}
\index{coDoTexture!getCoordinates}
All these functions simply return the values as explained in the constructor.



\section{Text Objects}
\index{Text data|see {coDoText}}
\index{Character data|see {coDoText}}
\index{String data|see {coDoText}}

Text objects can be used to transmit binary data between modules. Furthermore the 
Inventor renderer understands a special kind of text object directly:
OpenInventor file format (or VRML 1.0 format). 


\begin{longtable}{|p{4cm}|p{2.5cm}|p{7cm}|}
\hline
\multicolumn{3}{|p{13.5cm}|}
{\bf coDoText(const char *name, int numBytes)}\\
\hline
{Description:}  
           & \multicolumn{2}{p{9.5cm}|}{create a binary object} \\
\hline
\multicolumn{1}{|r|}{IN:} & {name:} 
                          & {name of text object}\\
\hline
\multicolumn{1}{|r|}{IN:} & {numBytes} 
                          & {allocated size in bytes}\endhead
\hline
\end{longtable}
\index{coDoText!constructor}

\begin{longtable}{|p{4cm}|p{2.5cm}|p{7cm}|}
\hline
\multicolumn{3}{|p{13.5cm}|}
{\bf coDoText(const char *name, const char *value)}\\
\hline
{Description:}  
           & \multicolumn{2}{p{9.5cm}|}{create a binary objectfrom a given 
	                              string and copy string to it} \\
\hline
\multicolumn{1}{|r|}{IN:} & {name:} 
                          & {name of text object}\\
\hline
\multicolumn{1}{|r|}{IN:} & {value} 
                          & {NONZERO terminated string}\endhead
\hline
\end{longtable}


\begin{longtable}{|p{4cm}|p{10cm}|}
\hline
\multicolumn{2}{|p{13.5cm}|}
{\bf int getTextLength()}\\
\hline
{Description:}  
           & {request length of storage area} \\
\hline
{Return value:}  
           & {length of buffer in bytes} \endhead
\hline
\end{longtable}
\index{coDoText!getTextLength}

\begin{longtable}{|p{4cm}|p{2.5cm}|p{7cm}|}
\hline
\multicolumn{3}{|p{13.5cm}|}
{\bf void getAddress(char **text\_ptr)}\\
\hline
{Description:}  
           & \multicolumn{2}{p{9.5cm}|}{get buffer pointer} \\
\hline
\multicolumn{1}{|r|}{OUT:} & {text\_ptr} 
                           & {buffer pointer}\endhead
\hline
\end{longtable}
\index{coDoText!getAddress}

Text objects should not be used for generating geometry for the renderer. Instead 
the COVISE geometry object classes are recommended. In this case other existing 
modules may be able to also handle data objects.

Using Text objects for anything else than ASCII text is generally discouraged, 
since neither type conversion nor field size adjustment is done on this data.


\section{Integer Arrays}
\index{Integer arrays|see {coDoIntArr}}
\index{Arrays|see {coDoIntArr}}

Integer arrays can be used to send indices or other integer fields. Fluid dynamics 
codes often save cell types, materials and other properties per cell.
These properties can be useful to allow selection of parts, e.g. with the 
SelectUsg module. The class IntArr defines an integer array with an 
arbitrary number of dimensions. It defined them in a one dimension field, 
saving the sizes per dimension, and the data field as one large field.


\begin{longtable}{|p{4cm}|p{2.5cm}|p{7cm}|}
\hline
\multicolumn{3}{|p{13.5cm}|}
{\bf coDoIntArr(const char *objName, int numDim, const int *dimArray, const int *initdata=NULL)}\\
\hline
{Description:}  
           & \multicolumn{2}{p{9.5cm}|}{} \\
\hline
\multicolumn{1}{|r|}{IN:} & {numDim} 
                          & {number of dimensions}\\
\hline
\multicolumn{1}{|r|}{IN:} & {dimArray} 
                          & {integer array holding the 
			                             sizes for numDim dimensions}\\
\hline
\multicolumn{1}{|r|}{IN:} & {initdata} 
                          & {if given and non-zero: 
			                              initialization field}\endhead
\hline
\end{longtable}
\index{coDoIntArr!constructor}

\begin{longtable}{|p{4cm}|p{10cm}|}
\hline
\multicolumn{2}{|p{13.5cm}|}
{\bf int getNumDimensions()}\\
\hline
{Description:}  
           & {get number of dimensions}  \\
\hline
{Return value:}  
           & {number of dimensions} \endhead
\hline
\end{longtable}
\index{coDoIntArr!getNumDimensions}

\begin{longtable}{|p{4cm}|p{10cm}|}
\hline
\multicolumn{2}{|p{13.5cm}|}
{\bf int getDimension(int i)}\\
\hline
{Description:}  
           & {get size in i-dimension (i=[0..numDim-1]} \\
\hline
{Return value:}  
           & {size in i-dimension} \endhead
\hline
\end{longtable}
\index{coDoIntArr!getDimension}

\begin{longtable}{|p{4cm}|p{10cm}|}
\hline
\multicolumn{2}{|p{13.5cm}|}
{\bf int getSize()}\\
\hline
{Description:}  
           & {get overall size} \\
\hline
{Return value:}  
           & {product of all sizes} \endhead
\hline
\end{longtable}
\index{coDoIntArr!getSize}

\begin{longtable}{|p{4cm}|p{10cm}|}
\hline
\multicolumn{2}{|p{13.5cm}|}
{\bf int *getAddress()}\\
\hline
{Description:}  
           & {get base address} \\
\hline
{Return value:}  
           & {pointer to table in memory} \endhead
\hline
\end{longtable}
\index{coDoIntArr!getAddress}

\section{Container Class coDoSet}
\index{Set|see{coDoSet}}
\index{Container|see{coDoSet}}

The class coDoSet is a container for a certain number of objects of the same type. These 
objects can also be sets, thus creating hierarchies of objects.
Such hierarchies should be used cautiously since every access to a data object means a 
task-task communication with the CRB. Thus excessive usage of set hierarchies can lead to 
significant performance penalties.

There are two cases, where sets are typically used:
\begin{itemize}
\item The real-world data is structured in a hierarchical system: typically FEM data 
comes from CAD models, which have a strong hierarchical structure.
\item The data is time-dependent: in this case there is one object per time step. Time 
dependent data is created by attaching an attribute to the set declaring this set to be 
a time series. A time set is simply any set with an attribute "TIMESTEP" and the
value "0 <last number>" attached to the
container object.
\end{itemize}

The creation of a coDoSet containing n objects consists of the following steps:

\begin{enumerate}

\item Creation of an array for the object pointers: {\it one element larger for termination}
{\tt coDistributedObject **objects = new coDistributedObject* [n+1]}

\item Creation of n objects of any kind and assigning it to {\tt objects[i]}. The 
class {\tt coDistributedObject} is the base class of all object classes, thus it can be 
assigned without casting. All object names {\it must} be different. As a convention, 
the name returned from the port is taken and `\_i' is added using the current index i.

\item Assignment of {\tt objects[n] = NULL} to terminate the list.

\item Creation of the set object with {\tt objects} as the parameter.

\item Deletion of all part objects with {\tt delete objects[i]}. 

\item Deletion of the array using {\tt delete [] objects}.

\end{enumerate}
For the final level, the set object is assigned to the port, otherwise it might be 
integrated into another set and deleted after creation of the higher level
set. As a rule, any object created with {\tt `new *'} must either be deleted or 
assigned to a port.

There are several  constructors for coDoSet, but only the following is supported:

 

\begin{longtable}{|p{4cm}|p{2.5cm}|p{7cm}|}
\hline
\multicolumn{3}{|p{13.5cm}|}
{\bf coDoSet(const char *name, coDistributedObject **elements)}\\
\hline
{Description:}  
           & \multicolumn{2}{p{9.5cm}|}{Create a set} \\
\hline
\multicolumn{1}{|r|}{IN:} & {name} 
                          & {name of the set 
			                               container to fill}\\
\hline
\multicolumn{1}{|r|}{IN:} & {elements} 
                          & {null-terminated array 
			                             of COVISE data objects}\endhead
\hline
\end{longtable}
\index{coDoSet!constructor}

To retrieve data from a set, there are two possibilities:


\begin{longtable}{|p{4cm}|p{2.5cm}|p{7cm}|}
\hline
\multicolumn{3}{|p{13.5cm}|}
{\bf coDistributedObject * const *getAllElements(int *no)}\\
\hline
{Description:}  
           & \multicolumn{2}{p{9.5cm}|}{get all elements of a set} \\
\hline
\multicolumn{1}{|r|}{OUT:} & {no} 
                          & {number of elements in the set}\\
\hline
{Return value:}  
           & \multicolumn{2}{p{9.5cm}|}{Array of object pointers, must be deleted 
	                              with delete[]} \endhead
\hline
\end{longtable}
\index{coDoSet!getAllElements}

\begin{longtable}{|p{4cm}|p{2.5cm}|p{7cm}|}
\hline
\multicolumn{3}{|p{13.5cm}|}
{\bf coDistributedObject *getElement(int no)}\\
\hline
{Description:}  
           & \multicolumn{2}{p{9.5cm}|}{get all elements of a set} \\
\hline
\multicolumn{1}{|r|}{IN:} & {no} 
                          & {number of elements in the set}\\
\hline
{Return value:}  
           & \multicolumn{2}{p{9.5cm}|}{pointer to set object} \endhead
\hline
\end{longtable}
\index{coDoSet!getElement}

