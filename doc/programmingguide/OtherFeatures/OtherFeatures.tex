
\begin{htmlonly}

\usepackage{html, htmllist}
\usepackage{longtable}

\bodytext{bgcolor="#ffffff" link="#0033cc" vlink="#0033cc"}

%%%==================================================	
%%%==================================================	

% #1  mark defined by \label
% #2  a linktext 
% #3  a html link 
\newcommand{covlink}[3]{\htmladdnormallink{#2}{#3} \latex{(\ref{#1})} }


\newenvironment{covimg}[4]%
{
 \begin{figure}[htp]
  \begin{center}
   \latexonly
      \includegraphics[scale=#4]{#1/pict/#2}
   \endlatexonly  
   \html{\htmladdimg[align="center"]{pict/#2.png}}
   \caption{#3}
  \end{center}
 \end{figure} 
}{} 

\newenvironment{covimg2}[3]%
{ 
 \begin{figure}[htp]
  \begin{center}
     \latexonly
       \includegraphics[scale=#3]{#1/pict/#2}   
     \endlatexonly
     \html{\htmladdimg[align="center"]{pict/#2.png}}
  \end{center}
 \end{figure} 
}{}

\definecolor{output}{rgb}{0.,0.,1.}
\definecolor{depend}{rgb}{1.,0.65,0.}
\definecolor{required}{rgb}{0.58,0.,0.83}
\definecolor{optional}{rgb}{0.,0.39,0.}

\newcommand{\addimage}[1] {\html{\htmladdimg{pict/#1.png}}}

\newcommand{\addpict}[4] {\latexonly
	     \begin{figure}[!htbp]
			  \begin{center}
   	 		  \includegraphics[scale=#1]{#2}
   	 		  \caption{#3}
		 		  \label{#4}
			  \end{center}
	 		\end{figure}
	     \endlatexonly}



\end{htmlonly}
%=============================================================
%=============================================================





%=============================================================
\startdocument
\chapter{Other Features}
\label{OtherFeatures}
%=============================================================


\section{Error, Warning and Info Messages}

The base class coModule delivers a set of methods to send messages to the users.
Info messages are displayed in the info box at the bottom of the Map Editor, 
warnings and errors in a pop-up window can be configured to pop up a window.
The difference between warnings and 
errors is that after a module has issued an error, no other modules will 
execute automatically, while warnings only pop up without any impact on the 
flow control.

The programming interface of the message commands only differs in the endings: 

\begin{longtable}{|p{7cm}|p{7cm}|}
\hline
   {\bf Function names } & {\bf Meaning} \endhead
\hline\hline
	coModule::sendInfo &  Info message \\
\hline
	coModule::sendWarning &  Warning message \\
\hline
	coModule::sendError &  Error message \\														
\hline
\end{longtable}
\index{coModule!sendInfo}
\index{coModule!sendWarning}
\index{coModule!sendError}

\begin{longtable}{|p{4cm}|p{2.5cm}|p{7cm}|}
\hline
\multicolumn{3}{|p{13.5cm}|}
{\bf static void coModule::sendWarning(const char *fmt, \dots);}\\
\hline
{Description:}  
           & \multicolumn{2}{|p{9.5cm}|}{Send messages to the user interface} \\
\hline
\multicolumn{1}{|r|}{IN:} & {fmt} 
                          & {Message format string}\endhead
\hline
\end{longtable}
\index{coModule!sendWarning}

All of the above message methods accept \texttt{printf} style arguments.

\section{Explicit flow control commands}
\latexonly
\index{Explicit flow control}
\endlatexonly

The control message passing is done internally in the module base class, usually 
none of the messages has to be sent by the user himself. Nevertheless, users 
might send own messages for explicit flow control to achieve effects that 
extend the classical data flow paradigm. Since these explicit calls violate the 
basic structure of the COVISE system, it is discouraged to use these calls. 
Explicit flow control calls can both influence stability and performance of 
the COVISE system and should be used by experienced users only and with special care for internal racing conditions.

\vspace*{1cm}
{\Large Explicitly stopping the pipeline}
\vspace*{0.5cm}

The function {\tt void Covise::send\_stop\_pipeline()} explicitly demands that 
no other modules be started automatically by the controller. It is
discouraged to use this function: {\tt sendError} accomplishes the same 
effect and also submits an error message. In all other than error cases the 
pipeline should run based on object creation and module connections. 

\vspace*{1cm}
{\Large Implicitly stopping pipeline execution}
\vspace*{0.5cm}

A module can prevent down-stream modules from being activated by not creating an 
output object.  Any module is supposed to stop if a connected port receives
no data. Using the classic module interface from earlier versions, this was 
the case when an object name was received, but no object could be retrieved
based on this name. All modules created based on coModule automatically 
implement this behavior.

\vspace*{1cm}
{\Large Self-execution}
\index{Self-execution}
\vspace*{0.5cm}

Any module callback (except compute) can tell the module that it should 
`execute' after the callback has been left. To achieve this, the function 
calls {\tt void coModule::selfExec()}\index{coModule!selfExec}. Multiple calls within a callback will 
only trigger a single start. \emph{Warning: Executing a module again while the 
succeeding module is still running may cause COVISE to crash.}

To prevent 'overrun' crashes, the module waits one second after the selfExec 
call has been submitted. This period can be configured by placing an
entry `execGracePeriod' (measured in seconds) into the 'System.HostInfo' section of the configuration files or 
with the call {\tt coModule::setExecGracePeriod(float gracePeriod)}.


\section{Re-using objects}

Usually, COVISE objects have an automatic life period: They are created by a 
module and are destroyed when they are not needed any more. Nevertheless,
COVISE makes some assumptions about the object usage:

\begin{itemize}
\item Objects read once at an input port are not used any further
\item An object created in a module is either put ONCE into an output port 
or put ONCE  into a Set or Geometry container
\end{itemize}

If a user wants to violate these assumption, he must tell the object that it 
is used more than once by incrementing its reference counter. This can be
useful to

\begin{itemize}
\item "hand through" an object from an input to an output port by adding it to a container object
\item add an input object to a container and give this to the output
\item add an object more than once to a container - e.g. for a constant part 
within a time series.
\end{itemize}

To increment the reference counter, call:


\begin{longtable}{|p{4cm}|p{10cm}|}
\hline
\multicolumn{2}{|p{13.5cm}|}
{\bf coDistributedObject::incRefCount()}\\
\hline
{Description:}  
           & {add a reference to a distributed object} \\
\hline
{Return value}  
              & {number of references} \\
\hline
\end{longtable}
\index{coDistributedObject!incRefCount}

%\section{Automatic Debugger startup}
%\latexonly
%\index{Debugger}
%\endlatexonly
%
%The module can be configured to start a debugger immediately in the coModule 
%constructor. Any debugger can be used if it is able to attach to a running
%process. To start the debugger, the environment variable {\tt CO\_DEBUGGER} 
%must be set to the full command that starts the debugger with a `\%d' dummy
%variable for the process ID. 
%
%Example: On SGI computers, the debugger is attached with 
%"cvd -pid \begin{htmlonly} <PID> \end{htmlonly} \latexonly $<$PID$>$ \endlatexonly ", so 
%the environment variable is set to:
%
%\begin{verbatim}
%      sentenv CO_DEBUGGER 'cvd -pid %d'
%\end{verbatim}
%

\section{Overriding parameter values}

You may use the class coHideParam in order to override the values of
some module parameters. For instance, you may wish to adjust the behaviour
of the module not by directly modifying the module params, but rather
let the module read their values from a string that the module
may get for instance through a Text object at an input port,
or that may be alternatively accessible as the value of some
attribute given to one of the input objects. The example below
should make it clear how this coHideParam class is used:

\underline{myModule.h}

\begin{verbatim}
class myModule
{
   private:

      coBooleanParam *p_OneBooleanParam_;
      coFloatParam *p_OneFloatScalarParam_;
      coFloatVectorParam *p_OneIntVectorParam_;

      coHideParam *h_distance_of_plane_;
      coHideParam *h_OneFloatScalarParam_;
      coHideParam *h_OneIntVectorParam_;
      std::vector<coHideParam *> hparams_; // a useful container 
                                      // for the previous
                                      // coHideParam pointers
   
}
\end{verbatim}

\underline{myModule.cpp}

\begin{verbatim}
void
myModule::postInst()
{
   // create the hiding objects here 
   // and keep pointers in the container
   hparams_.push_back(h_OneBooleanParam_ = 
                      new coHideParam(p_OneBooleanParam_));
   hparams_.push_back(h_OneFloatScalarParam_ = 
                      new coHideParam(p_OneFloatScalarParam_));
   hparams_.push_back(h_OneIntVectorParam_ = 
                      new coHideParam(p_OneIntVectorParam_));
}


myModule::myModule()   // ... build ports and parameters as usual
{
  // create params here
  p_OneBooleanParam_ = addBooleanParameter("OneBooleanParamName",
                                           "a bool param");
  p_OneFloatScalarParam_ = 
         addFloatParameter("OneFloatScalarParamName",
                                         "a float scalar param");
  p_OneIntVectorParam_ = 
         addFloatParameter("OneIntVectorParamName",
                                         "an int vector param");
 
  // and set default values
  ....
}

#include <stringstream>

myModule::compute(const char *)   
{
  // RESETTING HIDING OBJECTS
  // first of all reset all hparams_
  for(int param = 0;param<hparams_.size();++param){
     hparams_[param]->reset();
  }
  ...

  char *values;
  ...

  // OVERRIDE PARAMETER VALUES (LOADING coHideParam OBJECTS)
  // assume that after the previous lines 
  // 'values' points to the string:
  // "OneBooleanParamName 0\nOneFloatScalarParamName 3.47\n
  //  OneIntVectorParamName 2 5 3\n"
  char *oneValue = new char[strlen(values)+1];
  istringstream pvalues(values);
  while(pvalues.getline(oneValue,strlen(values)+1)){
     for(int param=0;param<hparams_.size();++param){
        hparams_[param]->load(value);
     }
  }
  delete [] value;
  ...
  ...

  // AND NOW WE USE THE coHideParam OBJECTS 
  // INSTEAD OF THE PARAMETERS
  // whenever you would normally use a port, ...
  //     float notThisWay = 
  //                   p_OneFloatScalarParam_->getValue();
  // ...you use now instead the pertinent
  // coHideParam object
  float ratherThisWay = h_OneFloatScalarParam_->getFValue(); // 3.47
  int intArray[3];
  intArray[0] = h_OneIntVectorParam_->getIValue(0); // 2
  intArray[1] = h_OneIntVectorParam_->getIValue(1); // 5
  intArray[2] = h_OneIntVectorParam_->getIValue(2); // 3
  if(h_OneBooleanParam_->getIValue()){ // the returned value is 0
     ...
  }
}

\end{verbatim}

Some remarks are here in order. Using the reset() member function
for all coHideParam objects should always be done before using
load(). After resetting, the object forgets previously assigned values,
and if a value is requested using one of the getIValue or getFValue
functions, the requested value will be that of the hidden parameter,
i.~e.\ the parameter that was indicated in the constructor
of the coHideParam object (see myModule::postInst()).

Hiding parameter values is achieved when using the load() member function
as used in the example. When the string {\sl values} contains no substring
for a given parameter, the corresponding coHideParam object will
not hide that parameter. Otherwise, it will take the values that follow
the parameter name.

If you are deriving myModule from coSimpleModule instead of
coModule, you may prefer performing the resetting and loading of
coHideParam objects in preHandleObjects and not in the compute function.

The last part concerns requesting values from these objects.
For this purpose you have the following member functions at your
disposal:

\begin{verbatim}
class coHideParam{
public:
   ...

   // get the float value 
   // when hiding coFloatParam or coFloatSliderParam
   float getFValue();

   // get the int value 
   // when hiding coChoiceParam, coBooleanParam, 
   // coInt32Param or coIntSliderParam
   int getIValue();

   // get the float value 
   // when hiding coFloatVectorParam
   float getFValue(int component);

   // get int values 
   // when hiding coFloatVectorParam
   int  getIValue(int);

   // get float values when hiding 
   // coFloatVectorParam (assuming 3 components)
   void getFValue(float &data0, float &data1, float &data2);
}
\end{verbatim}


