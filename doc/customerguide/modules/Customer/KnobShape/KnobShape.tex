\begin{htmlonly}\begin{htmlonly}
\documentclass{covise}

\usepackage{html, htmllist}
\usepackage{color}
\usepackage{graphicx}
\usepackage{longtable}
\usepackage{palatino}
\usepackage{picins}
\usepackage[colorlinks,dvips]{hyperref}

	  
\bodytext{BGCOLOR=FFFFFF LINK=#0033cc VLINK=#0033cc}


% #1  mark defined by \label
% #2  a linktext 
% #3  a html link 
\newcommand{\covlink}[3]%
{\html{\htmladdnormallink{#1}{#3}}\latex{\hyperref[#1]{#2} (\ref{#1})}}


\newenvironment{covimg}[4]%
{ \html{\htmladdimg[ALIGN=CENTER]{#2.gif}}
 
 \latexonly
 \begin{figure}[htp]
  \begin{center}
   \includegraphics[scale=#4]{#1/#2}
   \caption{#3}
  \end{center}
 \end{figure}
 \endlatexonly
}

\newenvironment{covimg2}[3]%
{ \html{\htmladdimg[ALIGN=CENTER]{#2.gif}}
 
 \latexonly
 \begin{figure}[htp]
  \begin{center}
   \includegraphics[scale=#3]{#1/#2}
  \end{center}
 \end{figure}
 \endlatexonly
}

\definecolor{output}{rgb}{0.,0.,1.}
\definecolor{depend}{rgb}{1.,0.65,0.}
\definecolor{required}{rgb}{0.58,0.,0.83}
\definecolor{optional}{rgb}{0.,0.39,0.}

\end{htmlonly}

%=============================================================
%=============================================================


%=============================================================
\startdocument
\subsection{KnobShape}
\label{KnobShape}
%=============================================================


%
% short description what the module does
%
KnobShape produces a simplified but quite realistic representation
of the geometry of a knob. 

%
% input of a module icon for example
% #1	path for eps
% #2  picture name
% #3  scale factor
\begin{covimg2}{modules/Customer/KnobShape}{KnobShape}{0.7}\end{covimg2}



%
% short information about versions 
%
%Sample is available since COVISE snap-2000-10 on all supported platforms.

%
%=============================================================
\subsubsection{Parameters}
%=============================================================
%

%\covlink{Colors}{Colors}{../../Color/Colors/Colors.html}
\begin{longtable}{|p{3cm}|p{2cm}|p{8.5cm}|}
\hline
   \bf{Name} & \bf{Type} & \bf{Description} \endhead
\hline\hline
        AbrasionDiv & Scalar & This parameter controls the number of element divisions
              used in the geometric representation of the fillet curve associated 
              with the abrasion radius.\\
\hline
        RoundingDiv & Scalar & This parameter controls the number of element divisions
              used in the geometric representation of the fillet curve associated 
              with the rounding radius.\\
\hline
\end{longtable}



%
%=============================================================
\subsubsection{Input Ports}
%=============================================================
%


\begin{longtable}{|p{2.5cm}|p{4.5cm}|p{7cm}|}
\hline
   \bf{Name} & \bf{Type} & \bf{Description} \endhead
\hline\hline
	\textcolor{required}{KnobParam} & DO\_Text & Message with
                 values for knob geometry parameters. If this object
                 has no KNOB\_SELECT attribute (the value is irrelevant),
                 dummy objects are produced for the output ports.\\
%\covlink{Transform}{Transform}{../../Tools/Transform/Transform.html}.\\
                     
                    
														
%	....
%	....

\hline
\end{longtable}
%=============================================================



%
%=============================================================
\subsubsection{Output Ports}
%=============================================================
%

 
\begin{longtable}{|p{2.5cm}|p{4.5cm}|p{7cm}|}
\hline
   \bf{Name} & \bf{Type} & \bf{Description} \endhead
\hline\hline
	\textcolor{required}{TheKnob} & DO\_Polygons & The knob geometry to be shown
                            and produced by the renderer.\\
\hline
	\textcolor{required}{TheNormals} & DO\_Unstructured\_V3D\_Data & Do not use this port.
                                 The normals have to be generated with GenNormals. \\
\hline
	\textcolor{required}{TheSameKnob} & DO\_Polygons & The same as the first output port,
                           but this is never a dummy. Even when you are not watching the
                           knob geometry, this object is always created for the benefit
                           of other modules that may require this information.\\
\hline
\end{longtable}
%=============================================================


%%=============================================================
%\subsubsection{Examples}
%%=============================================================
%%
%
%% examples for using this module
%
%%\paragraph{First example}
%%
%\begin{covimg}{modules/Tools/ImageToTexture}%
%		{ImageToTextureMap1}{covise/net/examples/ImageToTexture.net}{0.6}\end{covimg}
%
%In the first example we show a dynamic geometry. As the geometry moves the image 
%moves with it. In order to achieve this effect, we use the displacement information
%at the second port.
%
%\begin{covimg}{modules/Tools/ImageToTexture}%
%		{ImageToTextureRend1}{The image is dragged by the material motion.}{0.6}\end{covimg}
%
%\begin{covimg}{modules/Tools/ImageToTexture}%
%		{ImageToTextureMap2}{covise/net/examples/ImageToTexture2.net}{0.6}\end{covimg}
%
%In the second example we want to illustrate the effect of the parameter {\sl GroupGeometry}
%and of size adjustment. The geometry is a set with 4 DO\_Polygon object. When
%the value of {\sl GroupGeometry} is true (default), the image is mapped once onto the
%whole geometry. This effect is seen in the first renderer image. If the value
%of this parameter is false, then we get the second image. Note that here the geometry
%is used separately for each DO\_Polygon object. The third image has the default
%value for this parameter, i.e. true. If we are seeing here many eyes, it is because
%we are no longer fitting the image size to that of the geometry. In this case, we
%have manually set the image size to an inferior value, that is why we have to replicate
%the image in order to create a texture for the whole geometry.
%
%\begin{covimg}{modules/Tools/ImageToTexture}%
%		{ImageToTextureRend2_1}{GroupGeometry is true, the image size is that of the geometry.}{0.6}\end{covimg}
%
%\begin{covimg}{modules/Tools/ImageToTexture}%
%		{ImageToTextureRend2_2}{GroupGeometry is false.}{0.6}\end{covimg}
%
%\begin{covimg}{modules/Tools/ImageToTexture}%
%		{ImageToTextureRend2_3}{GroupGeometry is true, but the image size has been manually adjusted to an inferior value.}{0.6}\end{covimg}
%
%%
%%
%%The dimension of the sampled grid was 30x30x30 and the fill value of Sample
%%was set to 0.0.
%%
%%The module \covlink 
%%{CuttingSurface}{CuttingSurface}{../../Filter/CuttingSurface/CuttingSurface.html}
%% computes a cuttingsurface on the uniform grid and the module 
%%\covlink {Colors}{Colors}{../../Color/Colors/Colors.html} maps the
%%data on the surface to colors.
%%
%%The module
%%\covlink{ShowGrid}{ShowGrid}{../../Tools/ShowGrid/ShowGrid.html}
%% displays the uniform grid (in this case 3 sides of the outer surface).
%%
%%\begin{covimg2}{modules/Tools/Sample}{SampleRenderer}{0.7}\end{covimg2}
%%
%%\paragraph{Second example}
