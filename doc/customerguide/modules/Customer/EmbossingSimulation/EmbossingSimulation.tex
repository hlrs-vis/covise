\begin{htmlonly}\begin{htmlonly}
\documentclass{covise}

\usepackage{html, htmllist}
\usepackage{color}
\usepackage{graphicx}
\usepackage{longtable}
\usepackage{palatino}
\usepackage{picins}
\usepackage[colorlinks,dvips]{hyperref}

	  
\bodytext{BGCOLOR=FFFFFF LINK=#0033cc VLINK=#0033cc}


% #1  mark defined by \label
% #2  a linktext 
% #3  a html link 
\newcommand{\covlink}[3]%
{\html{\htmladdnormallink{#1}{#3}}\latex{\hyperref[#1]{#2} (\ref{#1})}}


\newenvironment{covimg}[4]%
{ \html{\htmladdimg[ALIGN=CENTER]{#2.gif}}
 
 \latexonly
 \begin{figure}[htp]
  \begin{center}
   \includegraphics[scale=#4]{#1/#2}
   \caption{#3}
  \end{center}
 \end{figure}
 \endlatexonly
}

\newenvironment{covimg2}[3]%
{ \html{\htmladdimg[ALIGN=CENTER]{#2.gif}}
 
 \latexonly
 \begin{figure}[htp]
  \begin{center}
   \includegraphics[scale=#3]{#1/#2}
  \end{center}
 \end{figure}
 \endlatexonly
}

\definecolor{output}{rgb}{0.,0.,1.}
\definecolor{depend}{rgb}{1.,0.65,0.}
\definecolor{required}{rgb}{0.58,0.,0.83}
\definecolor{optional}{rgb}{0.,0.39,0.}

\end{htmlonly}

%=============================================================
%=============================================================


%=============================================================
\startdocument
\subsection{EmbossingSimulation}
\label{EmbossingSimulation}
%=============================================================


%
% short description what the module does
%
EmbossingSimulation takes care that the Embossing module
may always find the data it needs. The purpose of these
module is the visualisation of a roll or sheet of paper.
This visualisation requires a geometric description of
paper bumps, which in its turn is the output of an
LS-DYNA simulation. This computation is in general
time-consuming and that is why, the output of these
simulations should be stored in a database. EmbossingSimulation
checks whether the results for the bump at issue are
already available in the database. If they are not, then
an LS-DYNA simulation is started and its output is
saved in the database. The preprocessing for the LS-DYNA
is done with ANSYS.


%
% input of a module icon for example
% #1	path for eps
% #2  picture name
% #3  scale factor
%\begin{covimg2}{modules/Customer/EmbossingSimulation}{EmbossingSimulation}{0.7}\end{covimg2}



%
% short information about versions 
%
%Sample is available since COVISE snap-2000-10 on all supported platforms.

%
%=============================================================
\subsubsection{Parameters}
%=============================================================
%

%\covlink{Colors}{Colors}{../../Color/Colors/Colors.html}

\begin{longtable}{|p{3.5cm}|p{2cm}|p{8.5cm}|}
\hline
   \bf{Name} & \bf{Type} & \bf{Description} \endhead
\hline\hline
	ndivMet & Scalar & Controls the number of divisions for the metal part.\\
\hline
	ndivPap & Scalar & Controls the number of divisions for the paper part.\\
\hline
\end{longtable}

The model for the LS-DYNA simulation has three parts: metal, paper and rubber.
The paper sheet gets pressed between metal and rubber.


%
%=============================================================
\subsubsection{Input Ports}
%=============================================================
%


\begin{longtable}{|p{3.5cm}|p{4cm}|p{7cm}|}
\hline
   \bf{Name} & \bf{Type} & \bf{Description} \endhead
\hline\hline
	\textcolor{required}{PraegeParam} & DO\_Text & 
                 This message from ControlSCA
                 contains the values for the parameters
                 that are relevant for the LS-DYNA simulation,
                 which describe
                 knob geometry, rubber type, tissue type and the
                 embossing force on the metal part.\\
\hline
	\textcolor{required}{NoppenColors} & DO\_Unstructured\_S3D\_Data
				       & 
                    This is an array created by the Design module
                    which contains design rules information.
                    The array has as many floats as knobs in
                    a basic cell. If all design rules are respected,
                    then all these numbers are 0.0. If one of these
                    numbers is greater than 0, then the module 
                    generates an error message.\\ 
%\covlink{Transform}{Transform}{../../Tools/Transform/Transform.html}.\\
                     
                    
														
%	....
%	....

\hline
\end{longtable}
%=============================================================



%
%=============================================================
\subsubsection{Output Ports}
%=============================================================
%

 
\begin{longtable}{|p{3.5cm}|p{4cm}|p{7cm}|}
\hline
   \bf{Name} & \bf{Type} & \bf{Description} \endhead
\hline\hline
	\textcolor{required}{SimulationOutcome} & DO\_Text& 
                          This is a message for Embossing. 
                          Only if the message is {\sl Done},
                          Embossing sets out to produce the
                          geometry of a whole paper sheet or roll.
                          The message signifies that Embossing 
                          will be able to find the data it needs
                          in the database. \\
%	....
%	....

\hline
\end{longtable}
%=============================================================


%%=============================================================
%\subsubsection{Examples}
%%=============================================================
%%
%
%% examples for using this module
%
%%\paragraph{First example}
%%
%\begin{covimg}{modules/Tools/ImageToTexture}%
%		{ImageToTextureMap1}{covise/net/examples/ImageToTexture.net}{0.6}\end{covimg}
%
%In the first example we show a dynamic geometry. As the geometry moves the image 
%moves with it. In order to achieve this effect, we use the displacement information
%at the second port.
%
%\begin{covimg}{modules/Tools/ImageToTexture}%
%		{ImageToTextureRend1}{The image is dragged by the material motion.}{0.6}\end{covimg}
%
%\begin{covimg}{modules/Tools/ImageToTexture}%
%		{ImageToTextureMap2}{covise/net/examples/ImageToTexture2.net}{0.6}\end{covimg}
%
%In the second example we want to illustrate the effect of the parameter {\sl GroupGeometry}
%and of size adjustment. The geometry is a set with 4 DO\_Polygon object. When
%the value of {\sl GroupGeometry} is true (default), the image is mapped once onto the
%whole geometry. This effect is seen in the first renderer image. If the value
%of this parameter is false, then we get the second image. Note that here the geometry
%is used separately for each DO\_Polygon object. The third image has the default
%value for this parameter, i.e. true. If we are seeing here many eyes, it is because
%we are no longer fitting the image size to that of the geometry. In this case, we
%have manually set the image size to an inferior value, that is why we have to replicate
%the image in order to create a texture for the whole geometry.
%
%\begin{covimg}{modules/Tools/ImageToTexture}%
%		{ImageToTextureRend2_1}{GroupGeometry is true, the image size is that of the geometry.}{0.6}\end{covimg}
%
%\begin{covimg}{modules/Tools/ImageToTexture}%
%		{ImageToTextureRend2_2}{GroupGeometry is false.}{0.6}\end{covimg}
%
%\begin{covimg}{modules/Tools/ImageToTexture}%
%		{ImageToTextureRend2_3}{GroupGeometry is true, but the image size has been manually adjusted to an inferior value.}{0.6}\end{covimg}
%
%%
%%
%%The dimension of the sampled grid was 30x30x30 and the fill value of Sample
%%was set to 0.0.
%%
%%The module \covlink 
%%{CuttingSurface}{CuttingSurface}{../../Filter/CuttingSurface/CuttingSurface.html}
%% computes a cuttingsurface on the uniform grid and the module 
%%\covlink {Colors}{Colors}{../../Color/Colors/Colors.html} maps the
%%data on the surface to colors.
%%
%%The module
%%\covlink{ShowGrid}{ShowGrid}{../../Tools/ShowGrid/ShowGrid.html}
%% displays the uniform grid (in this case 3 sides of the outer surface).
%%
%%\begin{covimg2}{modules/Tools/Sample}{SampleRenderer}{0.7}\end{covimg2}
%%
%%\paragraph{Second example}
