\begin{htmlonly}\begin{htmlonly}
\documentclass{covise}

\usepackage{html, htmllist}
\usepackage{color}
\usepackage{graphicx}
\usepackage{longtable}
\usepackage{palatino}
\usepackage{picins}
\usepackage[colorlinks,dvips]{hyperref}

	  
\bodytext{BGCOLOR=FFFFFF LINK=#0033cc VLINK=#0033cc}


% #1  mark defined by \label
% #2  a linktext 
% #3  a html link 
\newcommand{\covlink}[3]%
{\html{\htmladdnormallink{#1}{#3}}\latex{\hyperref[#1]{#2} (\ref{#1})}}


\newenvironment{covimg}[4]%
{ \html{\htmladdimg[ALIGN=CENTER]{#2.gif}}
 
 \latexonly
 \begin{figure}[htp]
  \begin{center}
   \includegraphics[scale=#4]{#1/#2}
   \caption{#3}
  \end{center}
 \end{figure}
 \endlatexonly
}

\newenvironment{covimg2}[3]%
{ \html{\htmladdimg[ALIGN=CENTER]{#2.gif}}
 
 \latexonly
 \begin{figure}[htp]
  \begin{center}
   \includegraphics[scale=#3]{#1/#2}
  \end{center}
 \end{figure}
 \endlatexonly
}

\definecolor{output}{rgb}{0.,0.,1.}
\definecolor{depend}{rgb}{1.,0.65,0.}
\definecolor{required}{rgb}{0.58,0.,0.83}
\definecolor{optional}{rgb}{0.,0.39,0.}

\end{htmlonly}

%=============================================================
%=============================================================


%=============================================================
\startdocument
\subsection{ControlSCA}
\label{ControlSCA}
%=============================================================


%
% short description what the module does
%

The main purpose of this module consists in providing
other modules with the information they need. In
order to fulfil this task, the module basically packs
the values of several related parameters in the form
of DO\_Text objects, sometimes furnished with apposite
attributes, which are interpreted by other modules.

%
% input of a module icon for example
% #1	path for eps
% #2  picture name
% #3  scale factor
\begin{covimg2}{modules/Customer/ControlSCA}{ControlSCA}{0.7}\end{covimg2}



%
% short information about versions 
%
%Sample is available since COVISE snap-2000-10 on all supported platforms.

%
%=============================================================
\subsubsection{Parameters}
%=============================================================
%

%\covlink{Colors}{Colors}{../../Color/Colors/Colors.html}

\begin{longtable}{|p{3.5cm}|p{2cm}|p{8.5cm}|}
\hline
   \bf{Name} & \bf{Type} & \bf{Description} \endhead
\hline\hline
	SheetDimensions & Choice & This parameter controls
        the size of a paper sheet.\\
\hline
	blatHoehe & Slider & This parameter controls
        the size of a paper sheet and is only relevant
        when the choice in SheetDimensions is {\sl Custom}.\\
\hline
	blatBreite & Slider & This parameter controls
        the size of a paper sheet and is only relevant
        when the choice in SheetDimensions is {\sl Custom}.\\
\hline
	whatExecute & Choice & This parameter controls
        which phase of the visulisation task is being
        computed. You have to follow an order from top to bottom.\\ 
\hline
	dbaseShape & Choice & This parameter keeps
        you informed in the KnobSelect phase about
        knob geometries that are close to your last choice,
        which you may select if you wish.
        It also tells you whether your geometry is
        available in the database.\\
\hline
	noppenHoehe & Slider & Knob height.\\
\hline
	ausrundungsRadius & Slider & Fillet radius at the base
        of the knob.\\
\hline
	abnutzungsRadius & Slider & Knob abrasion fillet radius.\\
\hline
	noppenWinkel & Slider & Knob slope angle. Zero stands for
        a knob with vertical walls.\\
\hline
	noppenForm & Choice & Knob shape.\\
\hline
	laenge1 & Slider & Top knob length in the X coordinate direction.\\
\hline
	laenge2 & Slider & Top knob length in the Y coordinate direction.\\
\hline
	free\_or\_param & Choice & You may choose between free or parametric
        design. In the first case basic cell size and knob positions
        are read from a file. In the parametric case a simple cell
        is always a rectangle with two knobs on two opposite corners.\\
\hline
	cad\_datei & Browser & File with design description when
        you have chosen the {\sl Free} option in free\_or\_param.\\
\hline
	grundZellenHoehe & Slider & You need not edit this parameter.
        Its value is automatically adjusted when you have chosen
        a free design from a file. Its value is the width of a basic cell.\\
\hline
	grundZellenBreite & Slider & You need not edit this parameter.
        Its value is automatically adjusted when you have chosen
        a free design from a file. Its value is the height of a basic cell.\\
\hline
	NumPoints & Slider & You need not edit this parameter.
        Its value is automatically adjusted when you have chosen
        a free design from a file. Its value is the number of knobs
        in a basic cell.\\
\hline
	Noppen\_1\ldots Noppen\_50 & Vector & You need not edit 
        these parameters. These are 2D vectors holding the XY
        coordinates of the knobs when you have chosen a free
        design from a file.\\
\hline
	Winkel & Slider & This is relevant in the case of parametric
        design. The angle between the diagonal of the
        rectangle defining a basic cell and the X axis is given
        by this parmeter in degrees.\\
\hline
	KnobDistance & Slider & This is relevant in the case of parametric
        design. This value is the X length of the rectangle defining
        a basic cell.\\
\hline
	DanzahlReplikationenX & Slider & Do not use this parameter.
        It is automatically adjusted for you. It contains the number
        of basic cell replications in the X direction required for
        the visualisation of a metal tool area as large as a full paper sheet.\\
\hline
	DanzahlReplikationenY & Slider & Do not use this parameter.
        It is automatically adjusted for you. It contains the number
        of basic cell replications in the Y direction required for
        the visualisation of a metal tool area as large as a full paper sheet.\\
\hline
	startEmbossing & Boolean & This parameter is relevant in
        the {\sl Embossing} phase. It must be set to true if you
        want to get a full functionality. Otherwise you will only
        be able to explore the contents of the bump database.
        Its purpose is a kind of safety against the possibility
        of starting an LS-DYNA simulation by mistake.\\
\hline
	DataBaseStrategy & Choice & This parameter is active
        during the {\sl Embossing} phase. As you modify other parameters
        which are relevant for the search in the bump database,
        this parameter keeps you informed of available bumps
        with similar values. This way you may easily choose
        one of these bumps and circumvent a time-consuming
        LS-DYNA simulation.\\
\hline
	tissueTyp & Choice & This parameter is active
        during the {\sl Embossing} phase. It allows you to specify
        the paper tissue.\\
\hline
	gummiHaerte & Slider & This parameter is active
        during the {\sl Embossing} phase. It allows you to specify
        the rubber hardness.\\
\hline
	anpressDruck & Slider & This parameter is active
        during the {\sl Embossing} phase. It allows you to specify
        the pressure applied during the embossing process.\\
\hline
	anzahlReplikationenX & Slider & Do not use this parameter.
        It is automatically adjusted for you. It contains the number
        of basic cell replications in the X direction required for
        the visualisation of a full sheet or roll.\\
\hline
	anzahlReplikationenY & Slider & Do not use this parameter.
        It is automatically adjusted for you. It contains the number
        of basic cell replications in the Y direction required for
        the visualisation of a full sheet or roll.\\
\hline
	presentationMode & Choice & You may visualise either
        a roll or a sheet.\\
\hline
	kraft & Slider & This parameter is relevant during
        the {\sl Traction} phase, but its value is ignored.\\
\hline
	Solution, \newline DOF\_Solution, \newline Derived\_Solution,\newline  
        SolidShellComponents, \newline TopBottom & Choice & This parameters 
        determine an automatic selection for the
        parameters with the same name in ReadANSYS, which gets
        this piece of information not directly, but indirectly
        through the Traction module.\\
\hline
	anzahlLinien, \newline anzahlPunkteProLinie,\newline  versatz & Slider & 
        Ignore this parameters.\\
%\hline
%	SolidShellComponents & Choice & It makes a selection for the
%        parameter with the same name in ReadANSYS, which gets
%        this piece of information not directly, but indirectly
%        through the Traction module.\\
\hline
\end{longtable}



%
%=============================================================
\subsubsection{Input Ports}
%=============================================================
%


\begin{longtable}{|p{3.5cm}|p{4cm}|p{7cm}|}
\hline
   \bf{Name} & \bf{Type} & \bf{Description} \endhead
\hline\hline
	\textcolor{required}{knobParams} & DO\_Text & 
                 This message from ControlSCA
                 contains the values for the parameters
                 that are relevant for KnobShape.\\
\hline
	\textcolor{required}{designParams} & DO\_Text & 
                    Information for Design.\\
\hline
	\textcolor{required}{praegeParams} & DO\_Text & 
                    Information for Embossing and EmbossingSimulation.\\
\hline
	\textcolor{required}{Blat} & DO\_Lines & 
                    Border lines of a paper sheet.\\
\hline
	\textcolor{required}{COVERInteractor} & DO\_Points & 
                    Connecting this port with the COVER
                    is required if you want to control this
                    module from the VR environment.\\ 
\hline
	\textcolor{required}{cutX} & DO\_Text & 
                    This port may be connected with
                    a CutGeometry module in order to cut
                    a geometry to exctly fit the X limits
                    of a paper sheet.\\
\hline
	\textcolor{required}{cutY} & DO\_Text & 
                    This port may be connected with
                    a CutGeometry module in order to cut
                    a geometry to exctly fit the Y limits
                    of a paper sheet.\\
%\covlink{Transform}{Transform}{../../Tools/Transform/Transform.html}.\\
                     
                    
														
%	....
%	....

\hline
\end{longtable}
%=============================================================



%
%=============================================================
\subsubsection{Output Ports}
%=============================================================
%

 
\begin{longtable}{|p{3.5cm}|p{4cm}|p{7cm}|}
\hline
   \bf{Name} & \bf{Type} & \bf{Description} \endhead
\hline\hline
	\textcolor{required}{Grundzelle} & DO\_Polygons & 
                   This is a rectangular polygon with 4 points.
                   This output is not intended to be shown
                   with the renderer. It is rather a
                   piece of information for Embossing, from
                   which this latter module derives the basic
                   cell size.\\
\hline
	\textcolor{required}{NoppenPositionen} & DO\_Points & 
                   This object contains the knob postions
                   in a basic cell and this information
                   is intended for the module Embossing.\\
\hline
	\textcolor{required}{NoppenColors} & DO\_Unstructured\_S3D\_Data & 
                   This is an array of as many floats as knobs in a
                   basic cell. If any of these array elements is
                   not 0, a violation of a design rule has occurred.
                   This port should be connected with the
                   corresponding input port of Embossing.\\
\hline
	\textcolor{required}{ShowGrundzelle} & DO\_Polygons & 
                   This is a rectangular polygon with 4 points,
                   and is needed for the visualisation of the design.\\
\hline
	\textcolor{required}{ShowKnobs} & DO\_Polygons & 
                   The object attached to this port describes
                   the geometry of all knobs present in a basic cell.\\
\hline
	\textcolor{required}{ShowNoppenColors} & DO\_Unstructured\_S3D\_Data & 
                   This is an array of as many floats as knobs in a
                   basic cell. If any of these array elements is
                   not 0, a violation of a design rule has occurred.
                   This object is used for the coloring of border
                   lines which are produced for each knob in a basic cell.\\
\hline
	\textcolor{required}{ShowKnobProfiles} & DO\_Lines & 
                   Do not use this port.\\
\hline
	\textcolor{required}{ShowPhysKnobProfiles} & DO\_Lines & 
                   The object attached to this port is used
                   for highlighting design rules for each
                   knob in a basic cell. These lines may be coloured
                   with the scalar field attached to ShowNoppenColors.\\
\hline
	\textcolor{required}{cutX} & DO\_Text & 
                   From this message a CutGeometry module
                   may get the necessary information in order
                   to automatically adjust its parameters. This
                   may be necessary because Design generates a
                   geometry with knob parts, which should be
                   cropped.\\
\hline
	\textcolor{required}{cutX} & DO\_Text & 
                   From this message a CutGeometry module
                   may get the necessary information in order
                   to automatically adjust its parameters. This
                   may be necessary because Design generates a
                   geometry with knob parts, which should be
                   cropped.\\
                    
%	....
%	....

\hline
\end{longtable}
%=============================================================


%%=============================================================
%\subsubsection{Examples}
%%=============================================================
%%
%
%% examples for using this module
%
%%\paragraph{First example}
%%
%\begin{covimg}{modules/Tools/ImageToTexture}%
%		{ImageToTextureMap1}{covise/net/examples/ImageToTexture.net}{0.6}\end{covimg}
%
%In the first example we show a dynamic geometry. As the geometry moves the image 
%moves with it. In order to achieve this effect, we use the displacement information
%at the second port.
%
%\begin{covimg}{modules/Tools/ImageToTexture}%
%		{ImageToTextureRend1}{The image is dragged by the material motion.}{0.6}\end{covimg}
%
%\begin{covimg}{modules/Tools/ImageToTexture}%
%		{ImageToTextureMap2}{covise/net/examples/ImageToTexture2.net}{0.6}\end{covimg}
%
%In the second example we want to illustrate the effect of the parameter {\sl GroupGeometry}
%and of size adjustment. The geometry is a set with 4 DO\_Polygon object. When
%the value of {\sl GroupGeometry} is true (default), the image is mapped once onto the
%whole geometry. This effect is seen in the first renderer image. If the value
%of this parameter is false, then we get the second image. Note that here the geometry
%is used separately for each DO\_Polygon object. The third image has the default
%value for this parameter, i.e. true. If we are seeing here many eyes, it is because
%we are no longer fitting the image size to that of the geometry. In this case, we
%have manually set the image size to an inferior value, that is why we have to replicate
%the image in order to create a texture for the whole geometry.
%
%\begin{covimg}{modules/Tools/ImageToTexture}%
%		{ImageToTextureRend2_1}{GroupGeometry is true, the image size is that of the geometry.}{0.6}\end{covimg}
%
%\begin{covimg}{modules/Tools/ImageToTexture}%
%		{ImageToTextureRend2_2}{GroupGeometry is false.}{0.6}\end{covimg}
%
%\begin{covimg}{modules/Tools/ImageToTexture}%
%		{ImageToTextureRend2_3}{GroupGeometry is true, but the image size has been manually adjusted to an inferior value.}{0.6}\end{covimg}
%
%%
%%
%%The dimension of the sampled grid was 30x30x30 and the fill value of Sample
%%was set to 0.0.
%%
%%The module \covlink 
%%{CuttingSurface}{CuttingSurface}{../../Filter/CuttingSurface/CuttingSurface.html}
%% computes a cuttingsurface on the uniform grid and the module 
%%\covlink {Colors}{Colors}{../../Color/Colors/Colors.html} maps the
%%data on the surface to colors.
%%
%%The module
%%\covlink{ShowGrid}{ShowGrid}{../../Tools/ShowGrid/ShowGrid.html}
%% displays the uniform grid (in this case 3 sides of the outer surface).
%%
%%\begin{covimg2}{modules/Tools/Sample}{SampleRenderer}{0.7}\end{covimg2}
%%
%%\paragraph{Second example}
