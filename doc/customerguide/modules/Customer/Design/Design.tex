\begin{htmlonly}\begin{htmlonly}
\documentclass{covise}

\usepackage{html, htmllist}
\usepackage{color}
\usepackage{graphicx}
\usepackage{longtable}
\usepackage{palatino}
\usepackage{picins}
\usepackage[colorlinks,dvips]{hyperref}

	  
\bodytext{BGCOLOR=FFFFFF LINK=#0033cc VLINK=#0033cc}


% #1  mark defined by \label
% #2  a linktext 
% #3  a html link 
\newcommand{\covlink}[3]%
{\html{\htmladdnormallink{#1}{#3}}\latex{\hyperref[#1]{#2} (\ref{#1})}}


\newenvironment{covimg}[4]%
{ \html{\htmladdimg[ALIGN=CENTER]{#2.gif}}
 
 \latexonly
 \begin{figure}[htp]
  \begin{center}
   \includegraphics[scale=#4]{#1/#2}
   \caption{#3}
  \end{center}
 \end{figure}
 \endlatexonly
}

\newenvironment{covimg2}[3]%
{ \html{\htmladdimg[ALIGN=CENTER]{#2.gif}}
 
 \latexonly
 \begin{figure}[htp]
  \begin{center}
   \includegraphics[scale=#3]{#1/#2}
  \end{center}
 \end{figure}
 \endlatexonly
}

\definecolor{output}{rgb}{0.,0.,1.}
\definecolor{depend}{rgb}{1.,0.65,0.}
\definecolor{required}{rgb}{0.58,0.,0.83}
\definecolor{optional}{rgb}{0.,0.39,0.}

\end{htmlonly}

%=============================================================
%=============================================================


%=============================================================
\startdocument
\subsection{Design}
\label{Design}
%=============================================================


%
% short description what the module does
%

The main purpose of this module consists in showing
an area of the metal tool used in the embossing
procedure as large as a paper sheet.

%
% input of a module icon for example
% #1	path for eps
% #2  picture name
% #3  scale factor
\begin{covimg2}{modules/Customer/Design}{Design}{0.7}\end{covimg2}



%
% short information about versions 
%
%Sample is available since COVISE snap-2000-10 on all supported platforms.

%
%=============================================================
\subsubsection{Parameters}
%=============================================================
%

%\covlink{Colors}{Colors}{../../Color/Colors/Colors.html}

\begin{longtable}{|p{2.5cm}|p{2cm}|p{8.5cm}|}
\hline
   \bf{Name} & \bf{Type} & \bf{Description} \endhead
\hline\hline
	tolerance & Scalar & This parameter is relevant when checking
        design rules. If the position of a knob is closer to the line limits
        of the basic cell than this tolerance, then this knob is
        assumed to lie exactly on that line, otherwise this knob
        would generate a spurious design rule violation.\\
\hline
\end{longtable}



%
%=============================================================
\subsubsection{Input Ports}
%=============================================================
%


\begin{longtable}{|p{2.5cm}|p{4.5cm}|p{7cm}|}
\hline
   \bf{Name} & \bf{Type} & \bf{Description} \endhead
\hline\hline
	\textcolor{required}{DesignParam} & DO\_Text & 
                 This message from ControlSCA
                 contains the values for the parameters
                 that are relevant for pattern design of a besic cell.
                 This module interprets that the user wants to
                 see a design when the object attached to this port
                 has got the attribute SHOW\_SCA\_DESIGN.\\
\hline
	\textcolor{required}{Knob} & DO\_Polygons
				       & 
                    This port is connected with the third port of
                    KnobShape. This way Design gets a mesh for a knob,
                    from which several copies are needed in the
                    mesh generation of a basic cell.\\
%\covlink{Transform}{Transform}{../../Tools/Transform/Transform.html}.\\
                     
                    
														
%	....
%	....

\hline
\end{longtable}
%=============================================================



%
%=============================================================
\subsubsection{Output Ports}
%=============================================================
%

 
\begin{longtable}{|p{3.5cm}|p{4cm}|p{7cm}|}
\hline
   \bf{Name} & \bf{Type} & \bf{Description} \endhead
\hline\hline
	\textcolor{required}{Grundzelle} & DO\_Polygons & 
                   This is a rectangular polygon with 4 points.
                   This output is not intended to be shown
                   with the renderer. It is rather a
                   piece of information for Embossing, from
                   which this latter module derives the basic
                   cell size.\\
\hline
	\textcolor{required}{NoppenPositionen} & DO\_Points & 
                   This object contains the knob postions
                   in a basic cell and this information
                   is intended for the module Embossing.\\
\hline
	\textcolor{required}{NoppenColors} & DO\_Unstructured\_S3D\_Data & 
                   This is an array of as many floats as knobs in a
                   basic cell. If any of these array elements is
                   not 0, a violation of a design rule has occurred.
                   This port should be connected with the
                   corresponding input port of Embossing.\\
\hline
	\textcolor{required}{ShowGrundzelle} & DO\_Polygons & 
                   This is a rectangular polygon with 4 points,
                   and is needed for the visualisation of the design.\\
\hline
	\textcolor{required}{ShowKnobs} & DO\_Polygons & 
                   The object attached to this port describes
                   the geometry of all knobs present in a basic cell.\\
\hline
	\textcolor{required}{ShowNoppenColors} & DO\_Unstructured\_S3D\_Data & 
                   This is an array of as many floats as knobs in a
                   basic cell. If any of these array elements is
                   not 0, a violation of a design rule has occurred.
                   This object is used for the coloring of border
                   lines which are produced for each knob in a basic cell.\\
\hline
	\textcolor{required}{ShowKnobProfiles} & DO\_Lines & 
                   Do not use this port.\\
\hline
	\textcolor{required}{ShowPhysKnobProfiles} & DO\_Lines & 
                   The object attached to this port is used
                   for highlighting design rules for each
                   knob in a basic cell. These lines may be coloured
                   with the scalar field attached to ShowNoppenColors.\\
\hline
	\textcolor{required}{cutX} & DO\_Text & 
                   From this message a CutGeometry module
                   may get the necessary information in order
                   to automatically adjust its parameters. This
                   may be necessary because Design generates a
                   geometry with knob parts, which should be
                   cropped.\\
\hline
	\textcolor{required}{cutX} & DO\_Text & 
                   From this message a CutGeometry module
                   may get the necessary information in order
                   to automatically adjust its parameters. This
                   may be necessary because Design generates a
                   geometry with knob parts, which should be
                   cropped.\\
                    
%	....
%	....

\hline
\end{longtable}
%=============================================================


%%=============================================================
%\subsubsection{Examples}
%%=============================================================
%%
%
%% examples for using this module
%
%%\paragraph{First example}
%%
%\begin{covimg}{modules/Tools/ImageToTexture}%
%		{ImageToTextureMap1}{covise/net/examples/ImageToTexture.net}{0.6}\end{covimg}
%
%In the first example we show a dynamic geometry. As the geometry moves the image 
%moves with it. In order to achieve this effect, we use the displacement information
%at the second port.
%
%\begin{covimg}{modules/Tools/ImageToTexture}%
%		{ImageToTextureRend1}{The image is dragged by the material motion.}{0.6}\end{covimg}
%
%\begin{covimg}{modules/Tools/ImageToTexture}%
%		{ImageToTextureMap2}{covise/net/examples/ImageToTexture2.net}{0.6}\end{covimg}
%
%In the second example we want to illustrate the effect of the parameter {\sl GroupGeometry}
%and of size adjustment. The geometry is a set with 4 DO\_Polygon object. When
%the value of {\sl GroupGeometry} is true (default), the image is mapped once onto the
%whole geometry. This effect is seen in the first renderer image. If the value
%of this parameter is false, then we get the second image. Note that here the geometry
%is used separately for each DO\_Polygon object. The third image has the default
%value for this parameter, i.e. true. If we are seeing here many eyes, it is because
%we are no longer fitting the image size to that of the geometry. In this case, we
%have manually set the image size to an inferior value, that is why we have to replicate
%the image in order to create a texture for the whole geometry.
%
%\begin{covimg}{modules/Tools/ImageToTexture}%
%		{ImageToTextureRend2_1}{GroupGeometry is true, the image size is that of the geometry.}{0.6}\end{covimg}
%
%\begin{covimg}{modules/Tools/ImageToTexture}%
%		{ImageToTextureRend2_2}{GroupGeometry is false.}{0.6}\end{covimg}
%
%\begin{covimg}{modules/Tools/ImageToTexture}%
%		{ImageToTextureRend2_3}{GroupGeometry is true, but the image size has been manually adjusted to an inferior value.}{0.6}\end{covimg}
%
%%
%%
%%The dimension of the sampled grid was 30x30x30 and the fill value of Sample
%%was set to 0.0.
%%
%%The module \covlink 
%%{CuttingSurface}{CuttingSurface}{../../Filter/CuttingSurface/CuttingSurface.html}
%% computes a cuttingsurface on the uniform grid and the module 
%%\covlink {Colors}{Colors}{../../Color/Colors/Colors.html} maps the
%%data on the surface to colors.
%%
%%The module
%%\covlink{ShowGrid}{ShowGrid}{../../Tools/ShowGrid/ShowGrid.html}
%% displays the uniform grid (in this case 3 sides of the outer surface).
%%
%%\begin{covimg2}{modules/Tools/Sample}{SampleRenderer}{0.7}\end{covimg2}
%%
%%\paragraph{Second example}
