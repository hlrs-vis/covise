
\begin{htmlonly}

\usepackage{html, htmllist}
\usepackage{longtable}

\bodytext{bgcolor="#ffffff" link="#0033cc" vlink="#0033cc"}

%%%==================================================	
%%%==================================================	

% #1  mark defined by \label
% #2  a linktext 
% #3  a html link 
\newcommand{covlink}[3]{\htmladdnormallink{#2}{#3} \latex{(\ref{#1})} }


\newenvironment{covimg}[4]%
{
 \begin{figure}[htp]
  \begin{center}
   \latexonly
      \includegraphics[scale=#4]{#1/pict/#2}
   \endlatexonly  
   \html{\htmladdimg[align="center"]{pict/#2.png}}
   \caption{#3}
  \end{center}
 \end{figure} 
}{} 

\newenvironment{covimg2}[3]%
{ 
 \begin{figure}[htp]
  \begin{center}
     \latexonly
       \includegraphics[scale=#3]{#1/pict/#2}   
     \endlatexonly
     \html{\htmladdimg[align="center"]{pict/#2.png}}
  \end{center}
 \end{figure} 
}{}

\definecolor{output}{rgb}{0.,0.,1.}
\definecolor{depend}{rgb}{1.,0.65,0.}
\definecolor{required}{rgb}{0.58,0.,0.83}
\definecolor{optional}{rgb}{0.,0.39,0.}

\newcommand{\addimage}[1] {\html{\htmladdimg{pict/#1.png}}}

\newcommand{\addpict}[4] {\latexonly
	     \begin{figure}[!htbp]
			  \begin{center}
   	 		  \includegraphics[scale=#1]{#2}
   	 		  \caption{#3}
		 		  \label{#4}
			  \end{center}
	 		\end{figure}
	     \endlatexonly}



\end{htmlonly}


%=============================================================
\startdocument
\chapter{Starting COVISE}
\label{StartCOVISE}
%=============================================================

\section{Introduction}

After having read this chapter you will be familiar with: 

\begin{itemize}
\item how to start COVISE 
\item the parts of the MapEditor 
\item how to load a map
\end{itemize}

COVISE is a toolkit to integrate several stages of a scientific or technical 
application such as simulations, reading of data, postprocessing and visualization. 
Each step is implemented as a module. Connecting these module forms a data flow 
network that can be executed. The user interacts with COVISE and its modules 
through a Motif based user interface.

\section{Starting COVISE}

After the installation the path to the covise executable is appended to the 
environment variable \$PATH. To start COVISE type {\bf covise} in a cshell or 
tcshell window. 

This initiates the following processes: the controller which is responsible 
for message handling, the request broker, responsible for data handling, and 
the main user interface called MapEditor.


\begin{longtable}{|p{15cm}|}
\hline
If the command {\bf covise} is not found make sure you use a tcshell or a cshell.
Otherwise have a look at the files README and INSTALL.TXT and check the environment
variable COVISE\_PATH. \newline
During the starting phase you might get the message \begin{verbatim} Timelimit in accept exceeded
\end{verbatim}
Edit the file covise.config in the covise directory and increase the timelimit 
for your machine. You find more information about {\it covise.config} in Chapter 2 in 
the {\it User's Guide}.\\	 														
\hline
\end{longtable}

COVISE can also be started with parameters. Typing  \verb/covise help/  will show you the syntax. 

\vspace{0.5cm}

The following 2 start options may be useful e.g. for demos with COVISE:
\begin{itemize}
\item {\bf -i}  start with MapEditor as icon
\item {\bf -e}  execute immediately after loading
\end{itemize}

\section{The Mapeditor}

\begin{covimg}{StartCOVISE}{mapnew}{The Mapeditor}{0.7}\end{covimg}

\begin{htmlonly}
Figure 1.1: The Mapeditor 
\vspace{0.5cm}
\end{htmlonly}

After the starting phase the MapEditor window (with Visual Programming as default) appears.

 
The MapEditor window consists of four main parts: 

\begin{itemize}
\item the Menu Bar 
\item the Tool Bar
\item one of three index cards according to the selected function
  \begin{itemize}
  \item Visual Programming - default: to build a map
  \item COVISE Objects - to look at the data structures involved
  \item Control Panel - to modify parameters
  \end{itemize} 
\item the Message Area 
\end{itemize}

A Chat Line - used for communication together with the Message Area - appears in case of
two or more users only.

The Visual Programming index card contains the Module Browser and the Visual Programming Area (canvas or
working area). 
The Menu Bar contains an item File. You can have a new file, in which you create a 
data flow network, you can save a file or you can open a file that contains a saved 
session. Such a file is called a map. It contains a list of the modules in the map and information
about the connections. 

Creating a new module network (map) is discussed in chapter 3. In this chapter you learn 
how to load and execute a saved map. COVISE comes with some example maps and the 
maps described in this tutorial. They are located in the following directories: 
\newline
{\it \$COVISEDIR/net/general/example}
\newline
{\it \$COVISEDIR/net/general/tutorial} 


\clearpage
\section{Loading a Map}

\begin{covimg}{StartCOVISE}{MenuFile}{Menu Item File}{0.7}\end{covimg}

\begin{htmlonly}
Figure 1.2: Menu Item File  
\vspace{0.5cm}
\end{htmlonly}

The menu item File Open pops up a file browser (Figure 1.2). Select the 
file {\it covise/net/tutorial/tutorial\_pressure\_1.net}. 

The loaded map appears in the working area. Each module is represented by an icon. 
A module has input data ports, represented by blue buttons on the top edge of the 
module icon and output data ports represented by blue buttons on the bottom of 
the icon. Green buttons mean optional data ports. Module icons are described in 
more detail in chapter 3. 

When the Renderer module is started as part of the module network, its window is 
opened. Initially it contains only three coordinate axis. This is explained in 
more detail in chapter 2. 

Now execute the network by selecting Execute in the Execution menu. One module 
icon after the other shows a red frame which indicates that it operates. Finally 
geometry objects appear in the render window (Figure 1.3). 

\begin{covimg}{StartCOVISE}{GeometryObjects}{Geometry Objects in the Renderer}{0.7}\end{covimg}
\begin{htmlonly}
Figure 1.3: Geometry Objects in the Renderer 
\vspace{0.5cm}
\end{htmlonly}

Have a look at Figure 1.1: the modules on the top are named {\bf RWCovise}. Their task 
is to read in data files in the COVISE data format.
The files are the output of a flow simulation. In a flow simulation the geometric 
domain occupied by the fluid is covered by a grid of a certain type. Such a grid 
consists of grid points together with their connectivity information. The 
simulation results typically contain scalar or vector data such as pressure, 
temperature or velocity attached to the grid points. Grids and attached data are 
handled as separate objects in COVISE. 

The left {\bf RWCovise} module reads the grid file tiny\_geo.covise. You select the grid 
via a file browser that pops up automatically. 

Below {\bf RWCovise} you find the module {\bf DomainSurface}. Out of the whole 3D grid it 
extracts the outer surface or bounding lines of the grid. The {\bf DomainSurface} module 
is directly connected to the Renderer module, which displays the results of 
{\bf DomainSurface}.
Figure 1.3 shows the content of the Renderer module window. The white lines are the 
bounding edges of the geometry. The displayed content is a flow channel with two 
inlets.

The second {\bf RWCovise} module reads in the pressure distribution tiny\_p.covise. 
{\bf CuttingSurface} then computes a cutting plane in the pressure data. The module 
{\bf Colors} converts the scalar data values to colors. The grid of the cutting plane
and the colors are combined into a geometry object by the {\bf Collect} module. 
{\bf Collect} is directly connected to the {\bf Renderer}, which displays the resulting
colored plane. 



