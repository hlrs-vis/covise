
\begin{htmlonly}

\usepackage{html, htmllist}
\usepackage{longtable}

\bodytext{bgcolor="#ffffff" link="#0033cc" vlink="#0033cc"}

%%%==================================================	
%%%==================================================	

% #1  mark defined by \label
% #2  a linktext 
% #3  a html link 
\newcommand{covlink}[3]{\htmladdnormallink{#2}{#3} \latex{(\ref{#1})} }


\newenvironment{covimg}[4]%
{
 \begin{figure}[htp]
  \begin{center}
   \latexonly
      \includegraphics[scale=#4]{#1/pict/#2}
   \endlatexonly  
   \html{\htmladdimg[align="center"]{pict/#2.png}}
   \caption{#3}
  \end{center}
 \end{figure} 
}{} 

\newenvironment{covimg2}[3]%
{ 
 \begin{figure}[htp]
  \begin{center}
     \latexonly
       \includegraphics[scale=#3]{#1/pict/#2}   
     \endlatexonly
     \html{\htmladdimg[align="center"]{pict/#2.png}}
  \end{center}
 \end{figure} 
}{}

\definecolor{output}{rgb}{0.,0.,1.}
\definecolor{depend}{rgb}{1.,0.65,0.}
\definecolor{required}{rgb}{0.58,0.,0.83}
\definecolor{optional}{rgb}{0.,0.39,0.}

\newcommand{\addimage}[1] {\html{\htmladdimg{pict/#1.png}}}

\newcommand{\addpict}[4] {\latexonly
	     \begin{figure}[!htbp]
			  \begin{center}
   	 		  \includegraphics[scale=#1]{#2}
   	 		  \caption{#3}
		 		  \label{#4}
			  \end{center}
	 		\end{figure}
	     \endlatexonly}



\end{htmlonly}


%=============================================================
\startdocument
\chapter{Analysis of 3D Data}
\label{Analysis_of_3D_Data}
%=============================================================

{\bf Note for the experienced user:}

This chapter has been reworked to demonstrate the new features for building maps
quicker and easier. COVISE provides {\bf Complex Modules} like TracerComp, 
CuttingSurfaceComp, or IsoSurfaceComp which include modules Colors, Collect etc.

For details please see Online Help, User's Guide, and Module Reference Guide.
 
\section{Introduction}

In this chapter you learn how to

\begin{itemize}
\item visualize the computational grid
\item visualize vector data
\item visualize scalar data
\end{itemize}

This section gives a very short overview how to use COVISE modules for the analysis 
of simulation data.

In the example the bounding geometry is a channel with two inlets. The grid was created 
with PROSTAR, the pre- and postprocessing program for STAR and the simulation was 
performed with STAR. In a previous COVISE session the grid and the simulation result 
files have been converted to COVISE format, as this format allows very compact files
and fast reading by COVISE.

In the directory {\it covise/share/covise/example-data/tutorial} you find the files 

\begin{itemize}
\item {\it tiny\_geo.covise}
\item {\it tiny\_velocity.covise}
\item {\it tiny\_p.covise}
\item {\it tiny\_te.covise}
\item {\it tiny\_vis.covise}
\end{itemize}

{\it tiny\_geo.covise} contains an unstructured grid while the other files contain data.

The following tutorial module pipelines (also called maps or nets) are located in the 
directory {\it covise/net/general/tutorial/}. In the next section only the most 
important module ports and parameters are explained. For detailed information about the 
modules a documentation in html format is available. 


\clearpage
\section{Visualizing the Computational Grid}

Assuming we don't know what the files contain it may be useful to first make the 
simulation grid visible. The following modules have this or related functionality:

{\bf ShowGrid} generates lines, points, hulls or bounding boxes for 
structured/rectilinear/uniform grids.

{\bf DomainSurface} generates the outer surface of an unstructured grid.

{\bf SimplifySurface} reduces the number of polygons representing a surface.

\vspace{1cm}
\begin{Large}{\bf tutorial\_grid\_1.net}\end{Large}
\vspace{0.5cm}

Start COVISE and open a new session. Choose the file {\it tutorial\_grid\_1.net} and 
execute the module network. Examining the module network, you see, that {\bf RWCovise} 
is from the category IO-Module. It imports the data file {\it tiny\_geo.covise} and 
generates an output data object containing an unstructured grid. The file name is 
specified as a file browser parameter. 

The module {\bf DomainSurface} from the category tools determines the outer surface 
of the grid or the edges of the outer surface. {\bf DomainSurface} receives the 
unstructured grid as input data object. It produces polygons representing the
outer surface which are available at the first output port. 

This port is connected with the module {\bf Renderer} which receives the poygons as 
input data objects. Figure 4.1 shows the map and the results displayed in the renderer window.

\begin{covimg}{Analysis_of_3D_Data}{grid1}{tutorial\_grid\_1}{0.7}\end{covimg}
\begin{htmlonly}
Figure 4.1: tutorial_grid_1
\vspace{0.5cm}
\end{htmlonly}

\vspace{1cm}
\begin{Large}{\bf tutorial\_grid\_2.net}\end{Large}
\vspace{0.5cm}

The disadvantage of the representation in Fig. 4.1 is that you can't see inside the channel. 

Load the map {\it tutorial\_grid\_2.net}. The difference to the first map is that the 
third output port of DomainSurface is connected to the Renderer. This port outputs 
data objects which contain the edges of the outer surface.

\begin{covimg}{Analysis_of_3D_Data}{grid2}{tutorial\_grid\_2}{0.7}\end{covimg}
\begin{htmlonly}
Figure 4.2: tutorial_grid_2
\vspace{0.5cm}
\end{htmlonly}


\vspace{1cm}
\begin{Large}{\bf tutorial\_grid\_3.net}\end{Large}
\vspace{0.5cm}

Another possibility to look inside the channel is to cut away parts of the geometry. 
This can be done by the module {\bf CutGeometry}. Load the map 
{\it tutorial\_grid\_3.net}. The module {\bf CutGeometry} cuts the geometry with a plane. 

The plane is defined using the parameters scalar and vertex. Vertex defines the 
orientation of the plane and also the cutting direction. Scalar positions the plane 
along its normal direction. In this example vertex is [0.0 0.0 -0.1] and
scalar is -0.12. This defines a plane which cuts away the upper surface of the channel.

\begin{covimg}{Analysis_of_3D_Data}{grid3}{tutorial\_grid\_3}{0.7}\end{covimg}
\begin{htmlonly}
Figure 4.3: tutorial_grid_3
\vspace{1cm}
\end{htmlonly}


\vspace{1cm}
\begin{Large}{\bf tutorial\_grid\_4.net}\end{Large}
\vspace{0.5cm}

If a surface contains too many polygons, it can't be displayed with an acceptable 
frame rate (at least 12 frames per second). 

For interactive handling it is necessary to reduce the number of polygons, which is 
done by the module {\bf SimplifySurface} (see {\it tutorial\_grid\_4.net}). 


\begin{covimg}{Analysis_of_3D_Data}{grid4net}{tutorial\_grid\_4.net}{0.7}\end{covimg}
\begin{htmlonly}
Figure 4.4: tutorial_grid_4
\vspace{0.5cm}
\end{htmlonly}

\begin{covimg}{Analysis_of_3D_Data}{grid4surfaces}{Original and Reduced Surface}{0.7}\end{covimg}
\begin{htmlonly}
Figure 4.5: Original and Reduced Surface
\vspace{0.5cm}
\end{htmlonly} 

Operating instructions for this example (to see how SimplifySurface works):
\begin{itemize}
\item Use the "Discrete Mesh" representation (click with the right mouse button on the
renderer window and choose DrawStyle \latexonly $>>$ \endlatexonly \begin{htmlonly}
> >\end{htmlonly} DiscreteMesh.
\item Parameter "strategy": select "Vertex Removal" (click on the SimplifySurface icon
to get the Module Information with the parameters)
\item Parameter "percent": specify the percentage of remaining triangles after
SimplifySurface.
\end{itemize}



\clearpage
\section{Visualizing Vector Data}


Typical vector data resulting from a simulation is the velocity field. The following modules have to deal with vector
data:

\vspace{0.5cm}

{\bf TracerComp} 
\begin{itemize}
\item computes streamlines on unstructured grids
\item computes streamlines on structured grids
\item computes animated particle traces 
\item contains additional functions to map output data to colors
\end{itemize}
(Historical Notes: \newline 
- In former versions of COVISE these functions had been implemented in 3 different tracers - TracerUsg, TracerStr,
and TetraTrace \newline
- If you specify e.g. animated particle traces, TracerComp, on top of standard Tracer, contains functions to show the
particles as spheres and map data as colors on that spheres - no need for additional modules Sphere,
Colors, Collect)

\vspace{0.5cm}

{\bf VectorField} (explicitly used in tutorial\_vel3.net only - otherwise contained,
implicitly,in CuttingSurfaceComp)  

computes small lines at each grid point. The direction of the lines 
represents the direction of the velocity vector and the size the amount of the velocity.

\vspace{1cm}
\begin{Large}{\bf tutorial\_vel\_1\_new.net}\end{Large}
\vspace{0.5cm}

\begin{covimg}{Analysis_of_3D_Data}{vel_1_new}{tutorial\_vel\_1\_new}{0.6}\end{covimg}
\begin{htmlonly}
Figure 4.6: tutorial_vel_1_new
\vspace{0.5cm}
\end{htmlonly}

Load the map {\it tutorial\_vel\_1\_new.net}. The second {\bf RWCovise} module reads the data 
file {\it tiny\_velocity.covise} which contains the velocity field. 

The module {\bf TracerComp} computes streamlines (Specify 'Streamlines' as parameter taskType in the Tracer).
Its first input port receives the grid 
object and the second port the data object containing the velocity field. 

The starting points of the streamlines and the number of streamlines are specified via
module parameters.startpoint1 and startpoint2 define a line. no\_of\_startpoints 
defines how many streamlines start on this line. Usually it is difficult to
determine the coordinates of the two points, if you don't know the size and position of 
the grid. In this example we choose [-0.4 0.5 0.02] for startpoint1 and [-0.4 0.3 0.02] 
for startpoint2 and draw 6 streamlines. 


\vspace{1cm}
\begin{Large}{\bf tutorial\_vel\_2\_new.net}\end{Large}
\vspace{0.5cm}



The map {\it tutorial\_vel\_2\_new.net} shows an example of animated particle traces. 

The layout of the map is the same as tutorial\_vel\_1\_new.net, but you have to specify
'particle' in the parameter taskType in TracerComp. i
The output of {\bf TracerComp} contains sets of points where each set represents the 
position of the particles at a certain timestep, together with velocity data attached 
to the points. TracerComp integrates the function to make particles more visible by displaying small 
spheres instead of points (specify radius as parameter) and to map output data 
to colors. 

%\begin{covimg2}{Analysis_of_3D_Data}{TimestepSequencer}{0.7}\end{covimg2}

As soon as the pipeline is executed, the geometry is displayed in the renderer 
window. In addition, the renderer contains a slider for selecting the timesteps of 
the animation. For a continuous animation press the play button.

\begin{covimg}{Analysis_of_3D_Data}{vel_2_new}{tutorial\_vel\_2\_new}{0.7}\end{covimg}
\begin{htmlonly}
Figure 4.7: tutorial_vel_2_new
\vspace{0.5cm}
\end{htmlonly}


\vspace{1cm}
\begin{Large}{\bf tutorial\_vel\_3.net / tutorial\_vel\_4.net}\end{Large}
\vspace{0.5cm}


The map {\it tutorial\_vel\_3.net} is an 'old-fashioned' map - have a look at it for information only
and don't use it as an example; it shows a flow visualization with small lines attached to 
the grid points which depict the magnitude and the direction of the respective 
velocities. 

The module {\bf VectorField} needs the grid and the velocity data as input data objects 
and computes lines and data attached to the lines as output objects. The data 
are mapped to colors using {\bf Colors} and {\bf Collect}. 

\begin{covimg}{Analysis_of_3D_Data}{vel_4_new}{tutorial\_vel\_4\_new}{0.7}\end{covimg}
\begin{htmlonly}
Figure 4.8: tutorial_vel_4_new
\vspace{0.5cm}
\end{htmlonly}

Displaying vector dashes at all grid points is not very useful. Instead a 
{\bf CuttingSurfaceComp} module can extract a 2D subset of the grid of which the grid 
points are then used to attach the vector dashes (see {\it tutorial\_vel\_4.net}). Note 
that the data attached to the cutting plane is determined by interpolating in the 
original grid. Since displaying vector data as colored arrows on a cutting surface is a frequently
used operation, this function has been integrated into CuttingSurfaceComp (see Fig. 4.8) and
is automatically activated in case of vector data!
{\bf Colors} is used to select a colormap for the colors on the vector dashes.   


\clearpage
\section{Visualizing Scalar Data}


Typical scalar data resulting from a simulation are e.g. pressure or temperature. Modules for visualizing scalar data
are:

{\bf CuttingSurfaceComp} computes a cutting plane, cylinder or sphere in unstructured grids.
In addition, it can map scalar data to colors.
{\bf IsoSurfaceComp} computes a surface connecting all points in space, which have the 
same scalar value. In addition, it can map scalar data to colors.
{\bf Colors} is used in this example to select a colormap (including minimum, maximum,
annotation) 

\vspace{1cm}
\begin{Large}{\bf tutorial\_pressure\_1\_new.net}\end{Large}
\vspace{0.5cm}

\begin{covimg}{Analysis_of_3D_Data}{pressure_1_new_map}{tutorial\_pressure\_1\_new}{0.7}\end{covimg}
\begin{htmlonly}
Figure 4.9: tutorial_pressure_1_new
\vspace{1cm}
\end{htmlonly}

Load the map {\it tutorial\_pressure\_1\_new.net}. Additional to the {\bf RWCovise} module 
which reads the grid we have a second {\bf RWCovise} module which reads the pressure 
data file {\it tiny\_p.covise}. 

{\bf CuttingSurfaceComp} from the category filter then extracts a surface (plane, cylinder, 
sphere). It needs the grid object and the data object containing scalar data as input 
and creates polygons or tristrips which form the cuttingsurface and data on the surface 
as output objects. In addition, the data are mapped to colors and combined into a geometry object together with the 
polygons/tristrips. 

Using the parameter \emph{option} you define the shape of the surface. Map
the parameter to the ControlPanel by clicking on the button in front of the parameter. 
Now you can see that you have the options plane, cylinder and sphere. Using the 
parameters \emph{vertex} and \emph{scalar} you specify the orientation and position 
of the surface.

For a plane the parameter \emph{vector} is the normal of the plane and the \emph{scalar}
is the distance of the plane from the origin. In this example the normal of the plane 
is (0 0 1) and the distance from the origin is 0.05.

A scalar parameter such as the distance is defined using three numbers: minimum, maximum
and current value. Map the parameter distance to the ControlPanel. The default 
interactor in the control panel is a VCR type player. You can increase or decrease the 
distance via the player buttons applying a certain increment which is specified in an 
extra field. Modify the distance parameter to get an overview of the pressure 
distribution in the channel. 

%\begin{covimg2}{Analysis_of_3D_Data}{ColorMaps}{0.7}\end{covimg2}
% part of next image?

In the render window you can also see a color bar. Color bars are produced by the 
modules Colors and ColorEdit (explicitly or implicitly, if the function of Colors is integrated
e.g. in the complex module CuttingSurfaceComp). They are only displayed in the render window if they 
are selected there. Click on the name of the data object to pop it up and click 
again to hide the colorbar.

Note: The module Colors\_2 is used to select the colormap, the minimum and maximum
values, and the annotation, e. g. "Pressure" (default: Colors). 

\begin{covimg}{Analysis_of_3D_Data}{press1out}{tutorial\_pressure\_1\_new: Results in the Renderer}{0.7}\end{covimg}
\begin{htmlonly}
Figure 4.10: tutorial_pressure_1\_new: Results in the Renderer
\vspace{0.5cm}
\end{htmlonly}


\vspace{1cm}
\begin{Large}{\bf tutorial\_pressure\_2\_new.net}\end{Large}
\vspace{0.5cm}

\begin{covimg}{Analysis_of_3D_Data}{pressure_2_new_map}{tutorial\_pressure\_2\_new}{0.7}\end{covimg}
\begin{htmlonly}
Figure 4.11: tutorial_pressure_2_new
\vspace{0.5cm}
\end{htmlonly}
                                                                                              

Another method to visualize scalar data are isosurfaces. An isosurface is a surface 
consisting of points in space, where the scalar data has the isovalue. Load the map 
{\it tutorial\_pressure\_2.net} and execute it. The blue surface connects all grid 
points where the pressure has the value 0.05.

Have a look at the map in the MapEditor window. Again the first {\bf RWCovise}
reads the grid while the second one reads the pressure data. 

                                             
\begin{covimg}{Analysis_of_3D_Data}{pressure_2_new_out}
                {tutorial\_pressure\_2\_new: Results in the Renderer}{0.7}\end{covimg}                                                                                          
\begin{htmlonly}
Figure 4.12: tutorial_pressure_2_new: Results in the Renderer
\vspace{0.5cm}
\end{htmlonly}                                                                                                                                                                                          

{\bf IsoSurfaceComp} needs the grid and the scalar data as input objects. The resulting
polygons or tristrips are provided at the first output port and the data on the surface 
at the second port. The isovalue is defined using the input parameter isovalue. 
The parameter gennormals enables the generation of normals at each polygon
vertex. 

The normals are available at the third output port.. The normals are used for the computation of the surface 
lighting. If you do not provide normals for a surface the renderer computes one normal 
per polygon face. Normals per vertex result in smoother surfaces than normals per face.

Note: Like CuttingSurfaceComp in tutorial\_pressure\_1\_new.net, IsoSurfaceComp maps the data to
colors. The Module Colors\_2 is used to select the colormap, the minimum and maximum
values, and the annotation, e. g. "Pressure" (default: Colors). 

\clearpage
\section{Summary (Example)}

\begin{Large}{\bf channel\_new.net}\end{Large}
\vspace{0.5cm}

The example {\bf channel\_new.net} below combines some of the features shown before in one map. In
addition, it shows how to get the size of your geometry object using the BoundingBox
module.
 
\begin{covimg}{Analysis_of_3D_Data}{channel_new}{channel\_new}{0.6}\end{covimg}
\begin{htmlonly}
Figure 4.13: channel_new
\vspace{0.5cm}
\end{htmlonly}

