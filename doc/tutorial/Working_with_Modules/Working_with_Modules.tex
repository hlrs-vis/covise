
\begin{htmlonly}

\usepackage{html, htmllist}
\usepackage{longtable}

\bodytext{bgcolor="#ffffff" link="#0033cc" vlink="#0033cc"}

%%%==================================================	
%%%==================================================	

% #1  mark defined by \label
% #2  a linktext 
% #3  a html link 
\newcommand{covlink}[3]{\htmladdnormallink{#2}{#3} \latex{(\ref{#1})} }


\newenvironment{covimg}[4]%
{
 \begin{figure}[htp]
  \begin{center}
   \latexonly
      \includegraphics[scale=#4]{#1/pict/#2}
   \endlatexonly  
   \html{\htmladdimg[align="center"]{pict/#2.png}}
   \caption{#3}
  \end{center}
 \end{figure} 
}{} 

\newenvironment{covimg2}[3]%
{ 
 \begin{figure}[htp]
  \begin{center}
     \latexonly
       \includegraphics[scale=#3]{#1/pict/#2}   
     \endlatexonly
     \html{\htmladdimg[align="center"]{pict/#2.png}}
  \end{center}
 \end{figure} 
}{}

\definecolor{output}{rgb}{0.,0.,1.}
\definecolor{depend}{rgb}{1.,0.65,0.}
\definecolor{required}{rgb}{0.58,0.,0.83}
\definecolor{optional}{rgb}{0.,0.39,0.}

\newcommand{\addimage}[1] {\html{\htmladdimg{pict/#1.png}}}

\newcommand{\addpict}[4] {\latexonly
	     \begin{figure}[!htbp]
			  \begin{center}
   	 		  \includegraphics[scale=#1]{#2}
   	 		  \caption{#3}
		 		  \label{#4}
			  \end{center}
	 		\end{figure}
	     \endlatexonly}



\end{htmlonly}


%=============================================================
\startdocument
\chapter{Working with Modules}
\label{Working_with_Modules}
%=============================================================

\section{Introduction}

After having read this chapter you will be familiar with:

\begin{itemize}
\item Module Categories
\item Input and Output Ports
\item COVISE Data Object Types
\item Parameters
\item Module actions
\end{itemize}

\section{Categories}

COVISE separates the processing steps of an application into separate modules. A module realizes a certain
functionality. Modules are grouped into categories. The general categories are:

\begin{itemize}
\item IO/Modules
\item Simulation
\item Mapper
\item Color
\item Tracer
\item Filter
\item Tools
\item Renderer
\end{itemize}

These categories reflect the typical steps in the analysis of 3D data e.g. coming from 
flow simulations.

\begin{covimg}{Working_with_Modules}{ModuleBrowser}{Category and Module Lists}{0.7}\end{covimg}
\begin{htmlonly}
Figure 3.1: Category and Module Lists
\vspace{0.5cm}
\end{htmlonly}


Categories and modules within them are shown in the Module Browser in the MapEditor 
window (see Figure 3.1).

If the hosts are colored (configurable in the file covise.config), modules running on 
this host appear in the same color. To start a module click on the category name and then 
drag the module name to the working area. To remove the module see the section 
\emph{Actions with Modules} later in this chapter.

Filter modules typically extract data from larger datasets. The {\bf CuttingSurface} module 
interpolates on a plane or cylinder surface data coming from a 3D data volume.

The computation of an isosurface belongs to the category Mapper, because data is mapped 
to a geometric object, that can be rendered. 

The category Tools contains many useful functionalities such as the module {\bf DomainSurface} 
that computes the outer surface of a computational grid or {\bf CutGeometry}, which cuts 
off portions of polygonal geometric objects like the ones created by {\bf DomainSurface}. 

Modules in the category Renderer display data. The 3D Inventor based {\bf Renderer} belongs
to this category as well as the 2D Plot module.

\section{Input and Output Ports}

\begin{covimg}{Working_with_Modules}{ModuleIcon}{Module Icon}{0.7}\end{covimg}
\begin{htmlonly}
Figure 3.2: Module Icon
\vspace{0.5cm}
\end{htmlonly}


Figure 3.2 shows a typical module icon. The violet and green ports at the top edge are 
input data ports, the blue and yellow ports at the bottom edge are output data ports. 
Violet input ports mean that the port has to be connected to another port, green ones 
mean that the connection is optional.

Be aware that certain green input ports can become violet - and thus have to be connected - when the corresponding
output port is connected.

Connection lines between the ports indicate the flow of data objects between the modules. 
You connect modules by clicking with the left mouse button onto the ports. To remove a 
connection between modules, double-click on the connection line.


\section{Data Types}

Data is exchanged between modules through COVISE data objects. If both modules which 
need to exchange data objects run on the same machine, the data objects are located in 
shared memory and thus allow an efficient exchange of data. If the modules run on 
different machines, the data objects are exchanged through tcp socket connections.

Input ports of one module and output ports of another module can only be connected if 
the data types are compatible. The most important data types are:

\begin{itemize}
\item Unstructured/Structured/Uniform/Rectilinear Grids
\item Unstructured/Structured Scalar Data
\item Unstructured/Structured Vector Data
\item Lines, Points, Triangle Strips, Polygons
\item Normals
\item Colors
\item Geometry
\item Sets
\end{itemize}

Geometry is a container to hold geometric primitives (lines, points, triangle strips, 
polygons), colors, and normals. A set is used as a container for other objects. Sets can 
be used to group objects or to hold different time steps of an object. 

\begin{covimg}{Working_with_Modules}{PortData}{Port Data Type}{0.7}\end{covimg}
\begin{htmlonly}
Figure 3.3: Port Data Type
\vspace{0.5cm}
\end{htmlonly}

If you click on the port with the right mouse button, the data type is displayed in the 
message window. Figure 3.3 shows the result of clicking on the output port of the 
RWCovise module.

\section{Parameters}

You see the parameter list when you click with the left mouse button on the book icon. 
The module info window pops up (Figure 3.4).


\begin{covimg}{Working_with_Modules}{ModuleInfo}{Module Info Window}{0.7}\end{covimg}
\begin{htmlonly}
Figure 3.4: Module Info Window
\vspace{0.5cm}
\end{htmlonly}

The first page of this notebook contains the parameters. Use the button with the parameter 
names on the left to attach interactors to this parameter which appear in the control panel. The next column contains the type of the parameters.
After this you see an area which allows the user to change the  parameter values. The 
content of each line depends on the parameter type.

If a module has a filebrowser as parameter, a filebrowser window automatically pops up as 
soon as the module is started.

When you add an interactor to the control panel, you usually have a choice between 
different types of interactors. To select the interactor click on the "Map to Control 
Panel Button" with the right mouse button. A small menu window pops up. 

\section{Module Actions}

Clicking with the right mouse button on the module icon pops up a menu with the module 
actions (Figure 3.5)

\begin{covimg}{Working_with_Modules}{ModuleActions}{Module Actions}{0.7}\end{covimg}
\begin{htmlonly}
Figure 3.5: Module Actions
\vspace{0.5cm}
\end{htmlonly}

The most important actions are "Execute" and "Delete".

\begin{itemize}
\item Execute 
\end{itemize}

executes the module pipeline starting from this module. It is typically more efficient 
to execute only a part of the module network after having changed some parameters instead 
of executing the whole network. I/O-modules often need a lot of time to read in large data 
files which is not necessary if you have just modified a module parameter
further down the processing chain.

\begin{itemize}
\item Delete
\end{itemize}

To remove a module from the working area choose this item.

\begin{itemize}
\item Rename
\end{itemize}

To rename a module (can also be used to label a module group).

\begin{itemize}
\item Move 
\end{itemize}

With this action an user can move this module to another host for execution. This is only 
possible, if an additional host was added to the session via "Add Host/Partner". All 
parameters and connection lines are moved too.

\begin{itemize}
\item Copy 
\end{itemize}

With this action an user can copy this module to another host for execution. This is only 
possible, if an additional host was added to the session via "Add Host/Partner". The 
current parameters values and connection lines are copied too.

\begin{itemize}
\item Help
\end{itemize} 

If you click on this item the online module documentation is loaded into the 
netscape browser. 

All actions can also be used for a group of selected modules.


