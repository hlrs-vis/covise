\section{Finite Lyapunov Exponents (FLE)}
\label{sec:FLE}


\subsection{Description}
%-----------------------------------------------------------------------------
Computes finite variants of Lyapunov exponent from velocity data. Finite Lyapunov exponents are computed at each node of the sampling grid.


\subsection{Input}
%-----------------------------------------------------------------------------
\begin{itemize}
\item
  data grid (unstructured)
  \begin{itemize}
  \item
    velocity (3-vect)
  \end{itemize}
\item
  sampling grid (unstructured) (optional)
  \begin{itemize}
  \item
    the trajectories are seeded at the nodes of this grid
  \end{itemize}
\end{itemize}


\subsection{Output}
%-----------------------------------------------------------------------------
\begin{itemize}
\item
  data grid (unstructured)
  \begin{itemize}
  \item
    FLE (scalar)
  \item
    FLE eigenval max (scalar)
  \item
    FLE eigenval med (scalar)
  \item
    FLE eigenval min (scalar)
  \item
    integration time/length (scalar)
  \item
    flow map (3-vect)
  \end{itemize}
\item
  trajectory grid (structured)
  \begin{itemize}
  \item
    trajectory data (2-vect)
    \begin{itemize}
    \item
      1. component: integration time, -1 inidcates end of trajectory
    \item
      2. component: unused
    \end{itemize}
    to be visualized using the \emph{field to lines} module
  \end{itemize}
\end{itemize}


\subsection{Parameters}
%-----------------------------------------------------------------------------
\begin{itemize}

\item
  \textbf{origin}: if the sampling grid is not provided as input, a uniform sampling grid is generated at this origin.

\item
  \textbf{cells}: if the sampling grid is not provided as input, a uniform sampling grid is generated with these dimensions.

\item
  \textbf{cell size}: if the sampling grid is not provided as input, a uniform sampling grid is generated with this voxel size.

\item
  \textbf{unsteady}: if active, path lines are used instead of streamlines. The timesteps are loaded from \emph{velocity file}.

\item
  \textbf{velocity file}: if \emph{unsteady} is active, timesteps are loaded using information from this descriptor file. The descriptor file and the corresponding mmap files are generated by the \emph{Dump CFX} module.

\item
  \textbf{start time}: if \emph{unsteady} is active, this is the point in time where pathlines are seeded.

\item
  \textbf{mode}:
  \begin{itemize}
  \item
    FTLE: finite-time Lyapunov exponent, confer the work of Haller \cite{Haller01} and e.g. \cite{Sadlo07ARidges}.
  \item
    FLLE: finite-length Lyapunov exponent according to \cite{Sadlo07ARidges}. Only meaningful for steady data.
  \item
    FSLE: finite-size Lyapunov exponent according to Aurell et al. \cite{Aurell97} and \cite{Sadlo07ARidges}.
  \item
    FTLEM: 
%maximum of finite-time Lyapunov exponent according 
please refer
to \cite{Sadlo07ARidges}.
  \item
    FTLEA:
% average of finite-time Lyapunov exponent according 
please refer
to \cite{Sadlo07ARidges}.
  \end{itemize}

\item
  \textbf{ln}: if active, the logarithm is used for computation, as in the original formulation of FTLE and FSLE. Switching it off may avoid that values obtained with short integration time dominate, refer to \cite{Sadlo07ARidges} for details.

\item
  \textbf{div T}: if active, FLE is divided by the integration time, as in the original formulation of FTLE and FSLE. Confer \cite{Sadlo07ARidges} for details.

\item
  \textbf{integration time}: prescribed time length for the trajectories. Not used in \emph{FLLE} mode.

\item
  \textbf{integration length}: prescribed length for the trajectories. Only used in \emph{FLLE} mode.

\item
  \textbf{time intervals}: number of time intervals. Unused in \emph{FTLE} and \emph{FLLE} mode. Refer to \cite{Sadlo07ARidges} for details.

\item
  \textbf{sep factor min}: minimum separation (expansion) factor. Only used in \emph{FSLE} mode. Refer to \cite{Sadlo07ARidges} for details.

\item
  \textbf{integ steps max}: maximum number of integration steps. These are high-level steps used for e.g. output of the trajectories. Internal stepping is adaptive.

\item
  \textbf{forward}: if active, trajectories are integrated in downstream direction.

\item
  \textbf{smoothing range}: smoothing range used for gradient computation. A value of 1 means ``no smoothing''. Large values distort the results.

\item
  \textbf{omit boundary cells}: experimental.

\item
  \textbf{grad neigh disabled}: if active, nodes with invalid flow map are not used for gradient computation, and if a node has because of this not enough neighbors, it is marked with FLT\_MAX, see \cite{Sadlo07ARidges} for details.

\item
  \textbf{execute}: if active, FLE is computed.

\end{itemize}


\subsection{Implementation}
%-----------------------------------------------------------------------------


\subsubsection{Version}
%.............................................................................

2007-08-24


\subsubsection{Author}
%.............................................................................

Filip Sadlo


\subsection{See Also}
%-----------------------------------------------------------------------------


\subsubsection{Related Modules}
%.............................................................................

\begin{itemize}

\item
  Dump CFX (Section~\ref{sec:dump-cfx})
\item
  Field To Lines (Section~\ref{sec:field-to-lines})
\item
  Ridge Surface (Section~\ref{sec:ridge-surface})
\end{itemize}


\subsubsection{Example Network}
%.............................................................................

\begin{itemize}

\item
  AVS: 
\item
  Covise: FLE.net
\item
  ParaView: FLE\_shared.pvs

\end{itemize}
