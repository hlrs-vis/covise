\section{News}
\label{sec:news}

\begin{itemize}

\item
  2008-09-02:
  \begin{itemize}
  \item
    ParaView 3.3: some problems with the ``Vortex Cores'', ``Ridge Surface'', ``FLE'', and ``Vortex Criteria'' modules on 64-bit systems have been fixed.
  \item
    ParaView: please note that using the ``Dump CFX'' module can interfere with other CFX readers (in current ParaView versions it seems to be impossible to have more than one reader for the same data type, although they appear in the list). Possible solutions are: disabling ``Dump CFX'' during cmake configuration and compilation, or an alternative PV\_PLUGIN\_PATH to a directory not containing ``Dump CFX''.
  \end{itemize}

\item
  2008-04-24:
  \begin{itemize}
  \item
    New module: DumpCFX: extract time dependent velocity data from CFX result files. The module generates the files needed for efficient path line integration (exploiting temporal coherence) using mmap(), as done by the FLE module.
  \item
    Support of unsteady data has moved from experimental to productive state.
  \item
    For supporting data not available in CFX format, please see the stand-alone program mentioned in the ``Dump CFX'' module and the ``Write Dump'' module.
  \end{itemize}

\item
  2008-04-17:
  \begin{itemize}
  \item
    Now supporting ParaView $\geq$ 3.3 (located in the paraview/ directory). Code for ParaView 3.0 to 3.2 is located in the paraview3.2/ directory. It is recommended to use ParaView3.3 (the CVS version) or larger, since it supports tooltips (pop-up help boxes that appear when the mouse is placed over a parameter).
  \item
    ParaView: all modules now test type of input data. No "Extract Datasets" necessary anymore for processing FLE (or any MultiBlock) output.
  \end{itemize}

\item
  2008-04-16:
  \begin{itemize}
  \item
    New modules: ReadUnstructured and WriteUnstructured: read and write unstructured data in Unstructured format. Use these for exchanging (converting) data between the visualization systems, e.g. for making use of the readers for different file formats available in the systems.
  \item
    ParaView: now supporting ParaView3. ParaView3 modules are in the paraview/ directory, ParaView2 modules are located in the paraview2/ directory.
  \item
    ParaView3: plugins went from pv3-plugins/ to paraview/plugins/.
  \item
    ParaView3: partial Win32 support.
  \item
    RidgeSurface: added spatial clipping (actually AVS version only). This can help to get rid of ridge regions by separating them using clipping and then removing them using connected component filtering (triangle count).
  \item
    Added support for rotating zones (experimental).
  \end{itemize}

\item
  2007-08-24:
  \begin{itemize}
  \item
    New module: Statistics, computes statistics of node data.
  \item
    ParaView: solved problem with non-sequential node coordinate arrays, as e.g. sometimes produced by the CFX reader.
  \item
    ParaView: now explictly warning in the case of unsupported cell types, such as polylines.
  \item
    VortexCriteria module: now computing absolute value of helicity variants instead of signed values.
  \end{itemize}

\item
  2007-08-16:
  \begin{itemize}
  \item
    Paraview: Fixed gradient computation bug at nodes belonging to non-3D (unsupported) cells. TODO: also skip non-3D cells in AVS and Covise versions, or support 2D cells in all versions.
  \item
    Fixed problem in vortex cores module with multi-set data.
  \item
    Forcing symmetric matrices, this avoids complex eigenvalues/eigenvectors due to numerics.
  \end{itemize}

\item
  2007-08-15:
  \begin{itemize}
  \item
    Added support for tensor line integration to Unstructured library.
  \item
    Added preliminary support for analytic fields to Unstructured library.
  \end{itemize}

\item
  2007-06-13:
  \begin{itemize}
  \item
    module ``ridge surfaces'' released. It extracts ridge surfaces, e.g. used for extracting LCS (Lagrangian coherent structures) from FTLE. It supports hexahedral cells (tested) and may also support tetrahedral cells (untested). Other cell types are not supported.
  \item
  \item
    ``FLE'' module:
    \begin{itemize}
    \item
      steady versions of FTLE, FLLE, and FSLE are verified.
    \end{itemize}
    Covise: added min/max functionality for parameters to all modules.
  \item
    ``vortex cores'': adapted min/max parameter limits.
  \item
    AVS: one makefile for all modules.
  \item
    Covise: one makefile for all modules.
  \item
    Paraview:
    \begin{itemize}
    \item
      one makefile for all modules in ``shared'' compilation mode.
    \item
      please keep in mind that ParaView 3.x is not yet supported.
    \end{itemize}
  \end{itemize}

\item
  2007-06-07:
  \begin{itemize}
  \item
    preliminary release of finite Lyapunov exponent (FLE) module.\\
    restrictions:
    \begin{itemize}
    \item
      no support for transient data
    \item
      only FTLE verified, other FLE variants are untested
    \item
      Paraview: no disabling of dependent parameters, output is multi-block (extract output using ``Extract Datasets'' filter)
    \end{itemize}

  \item
    module ``field to lines'' released, converting trajectory line field generated by e.g. FLE module to polylines, with subsampling functionality.
  \end{itemize}

\item
  2007-05-29: Paraview: ignore ETHUnified when integrating with CSCS modules. Follow instructions in README.txt instead.

\item
  2007-05-25: TAR distribution available.

\item
  2007-05-24:
  \begin{itemize}
  \item
  Paraview: ``internal'' and ``external'' compilation methods supported for Paraview 2.4 and Paraview 2.6, see README.txt for details.
  \item
  Paraview: entering -DVTK inside cmake is not needed anymore, it is set inside CMakeLists.txt.
  \end{itemize}

\item
  2007-05-23:
  \begin{itemize}
  \item
    Paraview: all modules are now passing (resampling) input point data to output, as a work-around for ``input to output passing requirement''-bug of Paraview 2.6. TODO: report the bug to Kitware.
  \item
    Paraview: now supporting Paraview 2.6. See README.txt for support of Paraview 2.4 (need to use appropriate version of PVLocal.pvsm.in).
  \end{itemize}

\item
  2007-05-22:
  \begin{itemize}
  \item
    Covise: modules now support time series as produced by e.g. ReadCFX.
  \item
    Covise: modules category renamed from ``Filip'' to ``Univiz''.
  \item
    Paraview: reported ``multiple input'' bug to Kitware.
  \end{itemize}

\item
  2007-05-14:
  \begin{itemize}
  \item
    Paraview: now describing how to turn on code optimization.
  \item
    ``vortex criteria'': better progress metering.
  \end{itemize}

\item
  2007-05-11: module ``vortex criteria'' released. Covise and Paraview versions still need to compute velocity gradient at each execution, even if input data has not changed (a ``nice to have''-feature in Covise and Paraview).

\item
  2007-05-07: moved everything to a single Subversion repository.

\end{itemize}

More detailed information is available by the ``svn log'' command.
