\section{Vortex Criteria}
\label{sec:vortex-criteria}


\subsection{Description}
%-----------------------------------------------------------------------------
Computes vortex criteria from velocity data. Vortex criteria are computed at each node of the data grid.
 

\subsection{Input}
%-----------------------------------------------------------------------------
\begin{itemize}
\item
  data grid (unstructured)
  \begin{itemize}
  \item
    velocity (3-vect)
  \end{itemize}
\end{itemize}


\subsection{Output}
%-----------------------------------------------------------------------------
\begin{itemize}
\item
  scalar vortex criterion
\end{itemize}


\subsection{Parameters}
%-----------------------------------------------------------------------------
\begin{itemize}

\item
  \textbf{quantity}:
  \begin{itemize}
  \item
    helicity:
absolute value of
dot product between velocity and vorticity as defined by \cite{Moffatt1969}. High absolute values indicate vortex regions.
% Note that this criterion is visualized e.g. using two isosurfaces, one at positive isolevel and one at negative isolevel, since arbitrary orientation of vorticity and velocity is possible.
  \item
    velo-norm helicity:
absolute value of
dot product between normalized velocity and not normalized vorticity. High absolute values indicate vortex regions.
% Note that this criterion is visualized e.g. using two isosurfaces, one at positive isolevel and one at negative isolevel, since arbitrary orientation of vorticity and velocity is possible.
  \item
    vorticity mag: vorticity magnitude.
  \item
    z vorticity: z-component of vorticity.
  \item
    lambda2: medium eigenvalue ($\lambda_2$) of $S^2 + \Omega^2$, where $S$ is the symmetric component of the velocity gradient and $\Omega$ is the antisymmetric component, according to \cite{JeongH95}. This medium eigenvalue has to be smaller than zero. Visualized e.g. using an isosurface at level zero. Levels below zero are not perfectly conforming to the criterion but may however give insights.
  \item
    Q: introduced by Hunt in 1988 \cite{HuntWM88} as $1/2 (||\Omega||^2 - ||S||^2)$ using the Frobenius norm $||A|| = \sqrt{trace A^T A}$. Values larger than zero indicate vortices. See e.g. \cite{JeongH95} for a discussion and comparison to $\lambda_2$.
  \item
    delta: defined by Chong et al. in 1990 \cite{ChongPC90} as $\Delta = (Q/3)^3 + (R/2)^2$ where $R = det \nabla u$ with velocity $u$. Values larger than zero indicate vortex regions. See e.g. \cite{JeongH95} for a discussion and comparison to $\lambda_2$.
  \item
    div accel: divergence of ``steady'' acceleration, more precisely $\nabla \cdot ((u \cdot \nabla) u)$ with velocity $u$. Negative values may indicate vortex regions. See \cite{GotoV04} for a discussion.
  \item
    divergence: divergence of velocity (not a vortex criterion to tell the truth).
  \end{itemize}

\item
  \textbf{smoothing range}: neighborhood range used for gradient computation. Larger values cause more smoothing. Large values distort the criteria.

\end{itemize}


\subsection{Implementation}
%-----------------------------------------------------------------------------


\subsubsection{Version}
%.............................................................................

2007-08-24


\subsubsection{Author}
%.............................................................................

Filip Sadlo


\subsection{See Also}
%-----------------------------------------------------------------------------


\subsubsection{Related Modules}
%.............................................................................
